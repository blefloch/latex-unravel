% \iffalse
%% File: unravel.dtx Copyright (C) 2013, 2015, 2018 Bruno Le Floch
%%
%% It may be distributed and/or modified under the conditions of the
%% LaTeX Project Public License (LPPL), either version 1.3c of this
%% license or (at your option) any later version.  The latest version
%% of this license is in the file
%%
%%    http://www.latex-project.org/lppl.txt
%%
%% -----------------------------------------------------------------------
%
%<*driver>
%\fi
%\iffalse
\documentclass[full]{l3doc}
\usepackage{unravel}
\begin{document}
  \DocInput{unravel.dtx}
\end{document}
%</driver>
% \fi
%
% \title{The \textsf{unravel} package: \\
%   watching TeX digest tokens\thanks{This
%     file has version number 0.2e, last revised 2018/04/04.}}
% \author{Bruno Le Floch}
% \date{2018/04/04}
%
% \maketitle
% \tableofcontents
%
% \begin{documentation}
%
% \section{\pkg{unravel} documentation}
%
% The aim of this \LaTeX{} package is to help debug complicated macros.
% This is done by letting the user step through the execution of some
% \TeX{} code, going through the details of nested expansions,
% performing assignments, as well as some simple typesetting commands.
% To use this package, one should normally run \TeX{} in a terminal.
%
% \subsection{Commands}
%
% \begin{function}{\unravel}
%   \begin{syntax}
%     \cs{unravel} \oarg{key-value list} \marg{code}
%   \end{syntax}
%   This command shows in the terminal the steps performed by \TeX{}
%   when running the \meta{code}.  By default, it pauses to let the
%   user read the description of every step: simply press |<return>| to
%   proceed.  Typing |s|\meta{integer} instead will go forward
%   \meta{integer} steps somewhat silently.  In the future it will be
%   possible to use a negative \meta{integer} to go back a few steps.
%   Typing |h| gives a list of various other possibilities.
%   The available \meta{key-value} options are described in
%   Section~\ref{sec:options}.
% \end{function}
%
% \begin{function}{\unravelsetup}
%   \begin{syntax}
%     \cs{unravelsetup} \marg{options}
%   \end{syntax}
%   Sets \meta{options} that apply to all subsequent \cs{unravel}.  See
%   options in Section~\ref{sec:options}.
% \end{function}
%
% \begin{function}{\unravel:nn}
%   \begin{syntax}
%     \cs{unravel:nn} \Arg{options} \Arg{code}
%   \end{syntax}
%   See \cs{unravel}.
% \end{function}
%
% \begin{function}{\unravel_get:nnN}
%   \begin{syntax}
%     \cs{unravel_get:nnN} \Arg{options} \Arg{code} \meta{tl var}
%   \end{syntax}
%   Performs \cs{unravel:nn} with the \meta{options} and \meta{code}
%   then saves the output into the \meta{tl var}.  The option
%   |mute| is useful in this case.
% \end{function}
%
% \begin{function}{\unravel_setup:n}
%   \begin{syntax}
%     \cs{unravel_setup:n} \Arg{options}
%   \end{syntax}
%   See \cs{unravelsetup}.
% \end{function}
%
% \subsection{Examples}
%
% The \pkg{unravel} package is currently based on the behaviour of
% \pdfTeX{}, but it should work in all engines supported by \pkg{expl3}
% (\pdfTeX{}, \XeTeX{}, \LuaTeX{}, ep\TeX{}, eup\TeX{}) as long as none
% of the primitives specific to those engines is used.  Any
% difference between how \pkg{unravel} and (pdf)\TeX{} process a given
% piece of code, unless described in the section~\ref{sec:differences},
% should be reported on the issue tracker
% (\url{https://github.com/blefloch/latex-unravel/issues}).
%
% As a simple example, one can run \LaTeX{} on the following file.
% \begin{verbatim}
% \documentclass{article}
% \usepackage{unravel}
% \unravel
%   {
%     \title{My title}
%     \author{Me}
%     \date{\today}
%   }
% \begin{document}
% \maketitle
% \end{document}
% \end{verbatim}
%
% A more elaborate example is to understand how \tn{newcommand} works.
% \begin{verbatim}
% \documentclass{article}
% \usepackage{unravel}
% \begin{document}
% \unravel
%   {
%     \newcommand*{\foo}[1]{bar(#1)}
%     \foo{3}
%   }
% \end{document}
% \end{verbatim}
%
% The \pkg{unravel} package understands deeply nested expansions as can
% be seen for instance by unravelling functions from \pkg{l3fp}, such as
% with the following code (given the current default settings, this code
% runs for roughly 2000 steps: you can type |s1980| as a response to the
% prompt, then press ``enter'' a few times to see the last few steps of
% expansion).
% \begin{verbatim}
% \documentclass{article}
% \usepackage{unravel}
% \begin{document}
% \ExplSyntaxOn
% \unravel { \fp_eval:n { 3.45 * 2 pi } }
% \ExplSyntaxOff
% \end{document}
% \end{verbatim}
%
% Given all the work that \pkg{unravel} has to do to emulate \TeX{}, it
% is not fast on very large pieces of code.  For instance, running it on
% |\documentclass{article}| takes about thirty seconds on my machine, and
% finishes after somewhat less than $21000$ steps.
% \begin{verbatim}
% \RequirePackage{unravel}
% \unravel{\documentclass{article}\relax}
% \usepackage{lipsum}
% \begin{document}
% \lipsum
% \end{document}
% \end{verbatim}
% The \tn{relax} command is needed after |\documentclass{article}|
% because this command tries to look for an optional argument:
% \cs{unravel} would not find any token, and would give up, as \TeX{}
% would if your file ended just after |\documentclass{article}|.  After
% running the above through \pdfTeX{}, one can check that the result is
% identical to that without \pkg{unravel}.  Note that
% \cs{unravel}|{\usepackage{lipsum}\relax}|, despite taking roughly as many steps
% to complete, is ten times slower, because \tn{newcommand} uses
% delimited arguments, which prevent some optimizations that
% \pkg{unravel} can otherwise obtain.  For comparison,
% |\unravel{\lipsum[1-30]}| also takes $20000$ step and is ten times
% faster than loading the package.
%
% \subsection{Options}
% \label{sec:options}
%
% \begin{function}{explicit-prompt}
%   Boolean option (default \texttt{false}) determining whether to give
%   an explicit prompt.  If \texttt{true}, the text ``|Your input=|''
%   will appear at the beginning of lines where user input is expected.
% \end{function}
%
% \begin{function}{internal-debug}
%   Boolean option (default \texttt{false}) used to debug \pkg{unravel}
%   itself.
% \end{function}
%
% \begin{function}{machine}
%   Option which takes no value and makes \pkg{unravel} produce an
%   output that is somewhat more suitable for automatic processing.  In
%   particular, it sets |max-action|, |max-output|, |max-input| to very
%   large values, and |number-steps| to \texttt{false}.
% \end{function}
%
% \begin{function}{max-action, max-output, max-input}
%   Integer options (defaults $50$, $300$, $300$) determining the
%   maximum number of characters displayed for the action, the output
%   part, and the input part.
% \end{function}
%
% \begin{function}{mute}
%   Make none of the steps produce any output, by setting
%   |trace-assigns|, |trace-expansion|, |trace-other|, |welcome-message|
%   to \texttt{false}.  This is only useful with \cs{unravel_get:nnN} or
%   when other options change some of these settings.
% \end{function}
%
% \begin{function}{number-steps}
%   Boolean option (default \texttt{true}) determining whether to
%   number steps.
% \end{function}
%
% \begin{function}{online}
%   Integer option determining where to write the output: terminal and
%   log if the option is positive, log only if the option is zero,
%   neither if the option is negative.
% \end{function}
%
% \begin{function}{trace-assigns, trace-expansion, trace-other}
%   Boolean options (default \texttt{true}) controlling what steps
%   produce any output at all.  The keys |trace-assigns|,
%   |trace-expansion|, |trace-other| control tracing of different types
%   of steps.
% \end{function}
%
% \begin{function}{welcome-message}
%   Boolean option (default \texttt{true}) determining whether to
%   display the welcome message.
% \end{function}
%
% \subsection{Differences between \pkg{unravel} and \TeX{}'s processing}
% \label{sec:differences}
%
% Bugs are listed at \url{https://github.com/blefloch/latex-unravel/issues}.
%
% Differences.
% \begin{itemize}
% \item Kerning between letters of a word is omitted, which can lead to
%   incorrect widths.
% \item Some primitives are not implemented yet: alignments
%   (\tn{halign}, \tn{valign}, \tn{noalign}, \tn{omit}, \tn{span},
%   \tn{cr}, \tn{crcr},~|&|), many math mode primitives, and
%   \tn{pdfprimitive}, \tn{discretionary}, as well as
%   all primitives specific to engines other than \pdfTeX{}.
%   This list may sadly be incomplete!
% \item \tn{aftergroup} is only partially implemented.
% \item \tn{everyhbox}, \tn{everyvbox}, \tn{everymath},
%   \tn{everydisplay}, \tn{lastkern}, \tn{lastnodetype},
%   \tn{lastpenalty}, \tn{lastskip}, \tn{currentiflevel} and
%   \tn{currentiftype} may have wrong values.  Perhaps
%   \tn{currentgrouplevel} and \tn{currentgrouptype} too.
% \item Tokens passed to \tn{afterassignment} are not yet kept after
%   \pkg{unravel} is done even if there has been no assignment.
% \item Tokens passed to \tn{aftergroup} are lost when \pkg{unravel} is
%   done.
% \item In \XeTeX{}, characters beyond the basic multilingual plane may
%   break \pkg{unravel} (not tested).
% \item For \pkg{unravel}, category codes are fixed when a file is read
%   using \tn{input},
%   while \TeX{} only fixes category codes when the corresponding
%   characters are converted to tokens.  Similarly, the argument of
%   \tn{scantokens} is converted to the new category code r\'egime in
%   one go, and the result must be balanced.
% \item Explicit begin-group and end-group characters other than the
%   usual left and right braces may make \pkg{unravel} choke, or may be
%   silently replaced by the usual left and right braces.
% \item \tn{endinput} is ignored with a warning, as it is very difficult
%   to implement it in a way similar to \TeX{}'s, and as it is most
%   often used at the very end of files, in a redundant way.
% \item \tn{outer} is not supported.
% \item \cs{unravel} cannot be nested.
% \item Control sequences of the form \cs{notexpanded:\ldots{}} are
%   reserved for use by \pkg{unravel}.
% \end{itemize}
%
% \subsection{Future perhaps}
%
% \begin{itemize}
% \item Allow users to change some settings globally/for one |\unravel|.
% \item Allow to replay steps that have already been run.
% \item Fix the display for \tn{if} and \tn{ifcat} (remove extraneous
%   \cs{exp_not:N}).
% \item Use the |file-error| fatal error message: first implement
%   \cs{@@_file_if_exist:nTF} and use it to determine whether
%   \tn{input} will throw a fatal error in \tn{batchmode} and
%   \tn{nonstopmode}.
% \item Use the |interwoven-preambles| fatal error message once
%   alignments are implemented.
% \item Look at all places where \TeX{}'s procedure |prepare_mag| is
%   called.
% \item Find out why so many input levels are used (see the log of the
%   |unravel003| testfile for instance)
% \end{itemize}
%
% \end{documentation}
%
% \begin{implementation}
%
% \section{\pkg{unravel} implementation}
%
% Some support packages are loaded first, then we declare the package's
% name, date, version, and purpose.
%
%    \begin{macrocode}
%<*package>
%    \end{macrocode}
%
%    \begin{macrocode}
%<@@=unravel>
%    \end{macrocode}
%
% Catcode settings.  In a group, set |\c| to be a synonym of |\catcode|
% for short, set the catcode of space to be 10 (using |\fam| to avoid
% needing a space or an equal sign to separate the two integer
% arguments of |\catcode|) and that of |%| to be 14 (using |\fam| again
% to avoid needing the digit 7 to have catcode other: we need the digit
% 5 anyway in two steps).  Then make |-|, |6|, |7|, |8|, |9| other (we
% must assume that |0| through |5| are already other), and make |:|,
% |_|, |h|, |j|, |k|, |q|, |s|, |w|, |x|, |y|, |z| letters (other
% lowercase letters already need to be letters in the rest of the
% code).  Make sure there is no |\endlinechar|.  We are finally ready
% to safely test whether the package has already been loaded and bail
% out in case it has.  Expanding |\fi| before ending the group ensures
% that the whole line has been read by \TeX{} before restoring earlier
% catcodes.
%    \begin{macrocode}
\begingroup\let\c\catcode\fam32\c\fam10\advance\fam5\c\fam14\c45 12 %
\c54 12\c55 12\c56 12\c57 12\c58 11\c95 11\c104 11\c106 11\c107 11 %
\c113 11\c115 11\c119 11\c120 11\c121 11\c122 11\endlinechar-1 %
\expandafter\ifx\csname unravel\endcsname\relax
\else\endinput\expandafter\endgroup\fi
%    \end{macrocode}
%
% Set |T| and |X| to be letters for an error message.  Set up braces
% and |#| for definitions, |=| for nicer character code assignments,
% |>| for integer comparison, |+| for integer expressions.
%    \begin{macrocode}
\c84 11\c88 11\c35 6\c123 1\c125 2\c62 12\c61 12\c43 12 %
%    \end{macrocode}
%
% If \eTeX{}'s \tn{numexpr} or \tn{protected} are not available, bail
% out with an error.
%    \begin{macrocode}
\expandafter\ifx\csname numexpr\endcsname\relax
\errmessage{unravel requires \numexpr from eTeX}
\endinput\expandafter\endgroup\fi
\expandafter\ifx\csname protected\endcsname\relax
\errmessage{unravel requires \protected from eTeX}
\endinput\expandafter\endgroup\fi
%    \end{macrocode}
%
% If \pkg{unravel} is loaded within a group, bail out because
% \pkg{expl3} would not be loaded properly.
%    \begin{macrocode}
\expandafter\ifx\csname currentgrouplevel\endcsname\relax\else
\ifnum\currentgrouplevel>1 \errmessage{unravel loaded in a group}
\endinput\expandafter\expandafter\expandafter\endgroup\fi\fi
%    \end{macrocode}
%
% Make spaces ignored and make |~| a space, to prettify code.
%    \begin{macrocode}
\catcode 32 = 9 \relax
\catcode 126 = 10 \relax
%    \end{macrocode}
%
% \begin{variable}{\l_@@_setup_restore_tl}
%   This token list variable will contain code to restore category
%   codes to their value when the package was loaded.
%    \begin{macrocode}
\gdef \l_@@_setup_restore_tl { }
%    \end{macrocode}
% \end{variable}
%
% \begin{macro}{\@@_setup_restore:}
%   Use the token list to restore catcodes to their former values, then
%   empty the list since there is no catcode to restore anymore.  This
%   mechanism cannot be nested.
%    \begin{macrocode}
\protected \gdef \@@_setup_restore:
  {
    \l_@@_setup_restore_tl
    \def \l_@@_setup_restore_tl { }
  }
%    \end{macrocode}
% \end{macro}
%
% \begin{macro}{\@@_setup_save:}
% \begin{macro}[EXP]{\@@_setup_save_aux:n}
%   This saves into \cs{l_@@_setup_restore_tl} the current catcodes
%   (from $0$ to $255$ only), \tn{endlinechar}, \tn{escapechar},
%   \tn{newlinechar}.
%    \begin{macrocode}
\protected \gdef \@@_setup_save:
  {
    \edef \l_@@_setup_restore_tl
      {
        \@@_setup_save_aux:w 0 =
        \endlinechar = \the \endlinechar
        \escapechar = \the \escapechar
        \newlinechar = \the \newlinechar
        \relax
      }
  }
\long \gdef \@@_setup_save_aux:w #1 =
  {
    \catcode #1 = \the \catcode #1 ~
    \ifnum 255 > #1 ~
      \expandafter \@@_setup_save_aux:w
      \the \numexpr #1 + 1 \expandafter =
    \fi
  }
%    \end{macrocode}
% \end{macro}
% \end{macro}
%
% \begin{macro}{\@@_setup_catcodes:nnn}
%   This sets all characters from |#1| to |#2| (inclusive) to have
%   catcode |#3|.
%    \begin{macrocode}
\protected \long \gdef \@@_setup_catcodes:nnn #1 #2 #3
  {
    \ifnum #1 > #2 ~ \else
      \catcode #1 = #3 ~
      \expandafter \@@_setup_catcodes:nnn \expandafter
        { \the \numexpr #1 + 1 } {#2} {#3}
    \fi
  }
%    \end{macrocode}
% \end{macro}
%
% \begin{macro}{\@@_setup_latexe:}
%   This saves the catcodes and related parameters, then sets them to
%   the value they normally have in a \LaTeXe{} package (in particular,
%   |@| is a letter).
%    \begin{macrocode}
\protected \gdef \@@_setup_latexe:
  {
    \@@_setup_save:
    \@@_setup_catcodes:nnn {0} {8} {15}
    \catcode 9 = 10 ~
    \catcode 10 = 12 ~
    \catcode 11 = 15 ~
    \catcode 12 = 13 ~
    \catcode 13 = 5 ~
    \@@_setup_catcodes:nnn {14} {31} {15}
    \catcode 32 = 10 ~
    \catcode 33 = 12 ~
    \catcode 34 = 12 ~
    \catcode 35 = 6 ~
    \catcode 36 = 3 ~
    \catcode 37 = 14 ~
    \catcode 38 = 4 ~
    \@@_setup_catcodes:nnn {39} {63} {12}
    \@@_setup_catcodes:nnn {64} {90} {11}
    \catcode 91 = 12 ~
    \catcode 92 = 0 ~
    \catcode 93 = 12 ~
    \catcode 94 = 7 ~
    \catcode 95 = 8 ~
    \catcode 96 = 12 ~
    \@@_setup_catcodes:nnn {97} {122} {11}
    \catcode 123 = 1 ~
    \catcode 124 = 12 ~
    \catcode 125 = 2 ~
    \catcode 126 = 13 ~
    \catcode 127 = 15 ~
    \@@_setup_catcodes:nnn {128} {255} {12}
    \endlinechar = 13 ~
    \escapechar = 92 ~
    \newlinechar = 10 ~
  }
%    \end{macrocode}
% \end{macro}
%
% \begin{macro}{\@@_setup_unravel:}
%   Catcodes for \pkg{unravel} (in particular, |@| is other, |:| and
%   |_| are letters, spaces are ignored, |~| is a space).
%    \begin{macrocode}
\protected \gdef \@@_setup_unravel:
  {
    \@@_setup_save:
    \@@_setup_catcodes:nnn {0} {8} {15}
    \catcode 9 = 9 ~
    \catcode 10 = 12 ~
    \catcode 11 = 15 ~
    \catcode 12 = 13 ~
    \catcode 13 = 5 ~
    \@@_setup_catcodes:nnn {14} {31} {15}
    \catcode 32 = 9 ~
    \catcode 33 = 12 ~
    \catcode 34 = 12 ~
    \catcode 35 = 6 ~
    \catcode 36 = 3 ~
    \catcode 37 = 14 ~
    \catcode 38 = 4 ~
    \@@_setup_catcodes:nnn {39} {57} {12}
    \catcode 58 = 11 ~
    \@@_setup_catcodes:nnn {59} {64} {12}
    \@@_setup_catcodes:nnn {65} {90} {11}
    \catcode 91 = 12 ~
    \catcode 92 = 0 ~
    \catcode 93 = 12 ~
    \catcode 94 = 7 ~
    \catcode 95 = 11 ~
    \catcode 96 = 12 ~
    \@@_setup_catcodes:nnn {97} {122} {11}
    \catcode 123 = 1 ~
    \catcode 124 = 12 ~
    \catcode 125 = 2 ~
    \catcode 126 = 10 ~
    \catcode 127 = 15 ~
    \@@_setup_catcodes:nnn {128} {255} {12}
    \escapechar  = 92 ~
    \endlinechar  = 32 ~
    \newlinechar  = 10 ~
  }
%    \end{macrocode}
% \end{macro}
%
% End the group where all catcodes where changed, but expand
% \cs{@@_setup_latexe:} to sanitize catcodes again outside the
% group.  The catcodes are saved.
%    \begin{macrocode}
\expandafter \endgroup \@@_setup_latexe:
%    \end{macrocode}
%
% Load a few dependencies: \pkg{expl3}, \pkg{xparse}, \pkg{gtl}.
% Load \pkg{l3str} if \pkg{expl3} is too old and does not define
% \cs{str_range:nnn}.  Otherwise loading \pkg{l3str} would give
% an error.
%    \begin{macrocode}
\RequirePackage{expl3,xparse}[2018/02/21]
\RequirePackage{gtl}[2018/04/04]
\csname cs_if_exist:cF\endcsname{str_range:nnn}{\RequirePackage{l3str}}
%    \end{macrocode}
%
% Before loading \pkg{unravel}, restore catcodes, so that the implicit
% \cs{ExplSyntaxOn} in \cs{ProvidesExplPackage} picks up the correct
% catcodes to restore when \cs{ExplSyntaxOff} is run at the end of the
% package.  The place where catcodes are restored are beyond
% \pkg{unravel}'s reach, which is why we cannot bypass \pkg{expl3} and
% simply restore the catcodes once everything is done.  To avoid issues
% with crazy catcodes, make \TeX{} read the arguments of
% \cs{ProvidesExplPackage} before restoring catcodes.  Then immediately
% go to the catcodes we want.
%    \begin{macrocode}
\csname use:n\endcsname
  {%
    \csname @@_setup_restore:\endcsname
    \ProvidesExplPackage
      {unravel} {2018/04/04} {0.2e} {Watching TeX digest tokens}%
    \csname @@_setup_unravel:\endcsname
  }%
%    \end{macrocode}
%
% \subsection{Primitives, variants, and helpers}
%
% \subsubsection{Renamed primitives}
%
% \begin{macro}
%   {
%     \@@_currentgrouptype:, \@@_everyeof:w, \@@_everypar:w,
%     \@@_set_escapechar:n, \@@_nullfont:,
%     \@@_hbox:w, \@@_the:w,
%   }
%   Copy primitives which are used multiple times, to avoid littering
%   the code with |:D| commands.  Primitives are left as |:D| in the
%   code when that is clearer (typically when testing the meaning of
%   a token against that of a primitive).
%    \begin{macrocode}
\cs_new_eq:NN \@@_currentgrouptype:      \etex_currentgrouptype:D
\cs_new_protected:Npn \@@_set_escapechar:n
  { \int_set:Nn \tex_escapechar:D }
\cs_new_eq:NN \@@_everyeof:w             \etex_everyeof:D
\cs_new_eq:NN \@@_everypar:w             \tex_everypar:D
\cs_new_eq:NN \@@_hbox:w                 \tex_hbox:D
\cs_new_eq:NN \@@_mag:                   \tex_mag:D
\cs_new_eq:NN \@@_nullfont:              \tex_nullfont:D
\cs_new_eq:NN \@@_the:w                  \tex_the:D
\cs_new_eq:NN \@@_number:w               \tex_number:D
%    \end{macrocode}
% \end{macro}
%
% \begin{macro}{\@@_special_relax:}
%   A special marker slightly different from \tn{relax} (its
%   \tn{meaning} is \tn{relax} but it differs from \tn{relax} according
%   to \tn{ifx}).  In the right-hand side of our assignment,
%   \cs{@@_special_relax:} could be replaced by any other expandable
%   command.
%    \begin{macrocode}
\exp_after:wN \cs_new_eq:NN
  \exp_after:wN \@@_special_relax:
  \exp_not:N \@@_special_relax:
%    \end{macrocode}
% \end{macro}
%
% \begin{macro}{\c_@@_prompt_ior, \c_@@_noprompt_ior}
%   These are not quite primitives, but are very low-level
%   |ior| streams to prompt the user explicitly or not.
%    \begin{macrocode}
\cs_new_eq:NN \c_@@_prompt_ior \c_sixteen
\int_const:Nn \c_@@_noprompt_ior { -1 }
%    \end{macrocode}
% \end{macro}
%
% \subsubsection{Variants}
%
% Variants that we need.
%    \begin{macrocode}
\cs_generate_variant:Nn \seq_push:Nn { Nf }
\cs_generate_variant:Nn \str_head:n { f }
\cs_generate_variant:Nn \tl_to_str:n { o }
\cs_generate_variant:Nn \tl_if_eq:nnTF { o }
\cs_generate_variant:Nn \tl_if_head_eq_meaning:nNT { V }
\cs_generate_variant:Nn \tl_if_single_token:nT { V }
\cs_generate_variant:Nn \gtl_gput_right:Nn { NV }
\cs_generate_variant:Nn \ior_str_get:NN { Nc }
\cs_generate_variant:Nn \gtl_if_empty:NTF { c }
\cs_generate_variant:Nn \gtl_to_str:N { c }
\cs_generate_variant:Nn \gtl_gpop_left:NN { c }
\cs_generate_variant:Nn \gtl_get_left:NN { c }
\cs_generate_variant:Nn \gtl_gset:Nn { c }
\cs_generate_variant:Nn \gtl_gconcat:NNN { ccc , cNc }
\cs_generate_variant:Nn \gtl_gclear:N { c }
\cs_generate_variant:Nn \gtl_gclear_new:N { c }
%    \end{macrocode}
%
% \begin{macro}{\@@_tl_if_in:ooTF}
%   Analogue of \cs{tl_if_in:ooTF} but with an extra group because that
%   function redefines an auxiliary that may appear in the code being
%   debugged (see Github issue |#27|).
%    \begin{macrocode}
\cs_new_protected:Npn \@@_tl_if_in:ooTF #1#2#3#4
  {
    \group_begin:
    \exp_args:Noo \tl_if_in:nnTF {#1} {#2}
      { \group_end: #3 } { \group_end: #4 }
  }
%    \end{macrocode}
% \end{macro}
%
% \begin{variable}{\l_@@_exp_tl}
% \begin{macro}{\@@_exp_args:Nx, \@@_exp_args:NNx}
%   Low-level because \cs{exp_args:Nx} redefines an internal
%   \pkg{l3expan} variable which may be appearing in code that we debug.
%    \begin{macrocode}
\tl_new:N \l_@@_exp_tl
\cs_new_protected:Npn \@@_exp_args:Nx #1#2
  {
    \cs_set_nopar:Npx \l_@@_exp_tl { \exp_not:N #1 {#2} }
    \l_@@_exp_tl
  }
\cs_new_protected:Npn \@@_exp_args:NNx #1#2#3
  {
    \cs_set_nopar:Npx \l_@@_exp_tl { \exp_not:N #1 \exp_not:N #2 {#3} }
    \l_@@_exp_tl
  }
%    \end{macrocode}
% \end{macro}
% \end{variable}
%
% \subsubsection{Miscellanous helpers}
%
% \begin{macro}{\@@_tmp:w}
%   Temporary function used to define other functions.
%    \begin{macrocode}
\cs_new_protected:Npn \@@_tmp:w { }
%    \end{macrocode}
% \end{macro}
%
% \begin{macro}{\@@_file_get:nN}
% \begin{macro}{\@@_file_get_aux:wN}
%    \begin{macrocode}
\cs_set_protected:Npn \@@_tmp:w #1
  {
    \cs_new_protected:Npn \@@_file_get:nN ##1##2
      {
        \group_begin:
          \@@_everyeof:w { #1 ##2 }
          \exp_after:wN \@@_file_get_aux:wN
          \exp_after:wN \prg_do_nothing:
            \tex_input:D ##1 \scan_stop:
      }
    \cs_new_protected:Npn \@@_file_get_aux:wN ##1 #1 ##2
      {
        \group_end:
        \tl_set:Nx ##2
          { \exp_not:o {##1} \exp_not:V \@@_everyeof:w }
      }
  }
\exp_args:No \@@_tmp:w { \token_to_str:N : : }
%    \end{macrocode}
% \end{macro}
% \end{macro}
%
% \begin{macro}[EXP]{\@@_tl_first_int:N}
% \begin{macro}[EXP]{\@@_tl_first_int_aux:Nn}
%   Function that finds an explicit number in a token list.  This is
%   used for instance when implementing \tn{read}, to find the stream
%   \meta{number} within the whole \tn{read} \meta{number} |to|
%   \meta{cs} construction.  The auxiliary initially has itself as a
%   first argument, and once a first digit is found it has
%   \cs{use_none_delimit_by_q_stop:w}.  That first argument is used
%   whenever what follows is not a digit, hence initially we loop,
%   while after the first digit is found any non-digit stops the
%   recursion.  If no integer is found, $0$ is left in the token list.
%   The surrounding \cs{int_eval:n} lets us dump digits in the input
%   stream while keeping the function fully expandable.
%    \begin{macrocode}
\cs_new:Npn \@@_tl_first_int:N #1
  {
    \int_eval:n
      {
        \exp_after:wN \@@_tl_first_int_aux:Nn
        \exp_after:wN \@@_tl_first_int_aux:Nn
        #1 ? 0 ? \q_stop
      }
  }
\cs_new:Npn \@@_tl_first_int_aux:Nn #1#2
  {
    \tl_if_single:nT {#2}
      {
        \token_if_eq_catcode:NNT + #2
          {
            \if_int_compare:w 1 < 1 #2 \exp_stop_f:
              #2
              \exp_after:wN \use_i_ii:nnn
              \exp_after:wN \@@_tl_first_int_aux:Nn
              \exp_after:wN \use_none_delimit_by_q_stop:w
            \fi:
          }
      }
    #1
  }
%    \end{macrocode}
% \end{macro}
% \end{macro}
%
% \begin{macro}{\@@_prepare_mag:}
%   Used whenever \TeX{} needs the value of \tn{mag}.  ^^A todo: move
%    \begin{macrocode}
\cs_new_protected:Npn \@@_prepare_mag:
  {
    \int_compare:nNnT { \g_@@_mag_set_int } > { 0 }
      {
        \int_compare:nNnF { \@@_mag: } = { \g_@@_mag_set_int }
          {
            \@@_tex_error:nn { incompatible-mag } { }
            \int_gset_eq:NN \@@_mag: \g_@@_mag_set_int
          }
      }
    \int_compare:nF { 1 <= \@@_mag: <= 32768 }
      {
        \@@_tex_error:nV { illegal-mag } \l_@@_head_tl
        \int_gset:Nn \@@_mag: { 1000 }
      }
    \int_gset_eq:NN \g_@@_mag_set_int \@@_mag:
  }
%    \end{macrocode}
% \end{macro}
%
% \subsubsection{String helpers}
%
% \begin{macro}{\@@_strip_escape:w}
% \begin{macro}{\@@_strip_escape_aux:N, \@@_strip_escape_aux:w}
%   This is based on the 2013-07-19 (and earlier) version of
%   \cs{cs_to_str:N}.  There are three cases.  If the escape character
%   is printable, the charcode test is false, and
%   \cs{@@_strip_escape_aux:N} removes one character.  If the escape
%   character is a space, the charcode test is true, and if there is no
%   escape charcter, the test is unfinished after |\token_to_str:N \ |.
%   In both of those cases, \cs{@@_strip_escape_aux:w} inserts
%   |-\@@_number:w \fi: \c_zero|.  If the escape character was a
%   space, the test was true, and \cs{int_value:w} converts
%   \cs{c_zero} to~|0|, hence the leading roman numeral expansion
%   removes a space from what follows (it is important that what follows
%   cannot start with a digit).  Otherwise, the test takes~|-| as its
%   second operand, is false, and the roman numeral expansion only sees
%   \cs{c_zero}, thus does not remove anything from what follows.
%    \begin{macrocode}
\cs_new:Npn \@@_strip_escape:w
  {
    \tex_romannumeral:D
      \if_charcode:w \token_to_str:N \ \@@_strip_escape_aux:w \fi:
      \@@_strip_escape_aux:N
  }
\cs_new:Npn \@@_strip_escape_aux:N #1 { \c_zero }
\cs_new:Npn \@@_strip_escape_aux:w #1#2
  { - \@@_number:w #1 \c_zero }
%    \end{macrocode}
% \end{macro}
% \end{macro}
%
% ^^A todo: change approach: this abuses future l3obj.
% \begin{macro}[EXP]{\@@_to_str:n}
% \begin{macro}[EXP]
%   {\@@_to_str_auxi:w, \@@_to_str_auxii:w, \@@_gtl_to_str:n}
%   Use the type-appropriate conversion to string.
%    \begin{macrocode}
\cs_new:Npn \@@_to_str:n #1
  {
    \tl_if_head_eq_meaning:nNTF {#1} \scan_stop:
      { \@@_to_str_auxi:w #1 ? \q_stop }
      { \tl_to_str:n }
    {#1}
  }
\cs_set:Npn \@@_tmp:w #1
  {
    \cs_new:Npn \@@_to_str_auxi:w ##1##2 \q_stop
      {
        \exp_after:wN \@@_to_str_auxii:w \token_to_str:N ##1 \q_mark
          #1 tl \q_mark \q_stop
      }
    \cs_new:Npn \@@_to_str_auxii:w ##1 #1 ##2 \q_mark ##3 \q_stop
      { \cs_if_exist_use:cF { @@_ ##2 _to_str:n } { \tl_to_str:n } }
  }
\exp_args:No \@@_tmp:w { \tl_to_str:n { s _ _ } }
\cs_new:Npn \@@_gtl_to_str:n { \gtl_to_str:n }
%    \end{macrocode}
% \end{macro}
% \end{macro}
%
% \begin{macro}{\@@_str_truncate_left:nn}
% \begin{macro}{\@@_str_truncate_left_aux:nnn}
%   Truncate the string |#1| to a maximum of |#2| characters.  If it is
%   longer, replace some characters on the left of the string by
%   |(123~more~chars)~| with the appropriate number instead of~|123|.
%   In any reasonable case, $25$ is big enough to fit this extra text.
%    \begin{macrocode}
\cs_new:Npn \@@_str_truncate_left:nn #1#2
  {
    \exp_args:Nf \@@_str_truncate_left_aux:nnn
      { \str_count:n {#1} } {#1} {#2}
  }
\cs_new:Npn \@@_str_truncate_left_aux:nnn #1#2#3
  {
    \int_compare:nNnTF {#1} > {#3}
      {
        ( \int_eval:n { #1 - #3 + 25 } ~ more~chars ) ~
        \str_range:nnn {#2} { #1 - #3 + 26 } {#1}
      }
      { \tl_to_str:n {#2} }
  }
%    \end{macrocode}
% \end{macro}
% \end{macro}
%
% \begin{macro}{\@@_str_truncate_right:nn}
% \begin{macro}{\@@_str_truncate_right_aux:nnn}
%   Truncate the string |#1| to a maximum of |#2| characters.  If it is
%   longer, replace some characters on the right of the string by
%   |~(123~more~chars)| with the appropriate number instead of~|123|.
%   In any reasonable case, $25$ is big enough to fit this extra text.
%    \begin{macrocode}
\cs_new:Npn \@@_str_truncate_right:nn #1#2
  {
    \exp_args:Nf \@@_str_truncate_right_aux:nnn
      { \str_count:n {#1} } {#1} {#2}
  }
\cs_new:Npn \@@_str_truncate_right_aux:nnn #1#2#3
  {
    \int_compare:nNnTF {#1} > {#3}
      {
        \str_range:nnn {#2} { 1 } { #3 - 25 } ~
        ( \int_eval:n { #1 - #3 + 25 } ~ more~chars )
      }
      { \tl_to_str:n {#2} }
  }
%    \end{macrocode}
% \end{macro}
% \end{macro}
%
% \subsubsection{Helpers for control flow}
%
% \begin{macro}[EXP]{\@@_exit:w, \@@_exit_point:}
%   Jump to the very end of this instance of \cs{unravel}.
%    \begin{macrocode}
\cs_new_eq:NN \@@_exit_point: \prg_do_nothing:
\cs_new:Npn \@@_exit:w #1 \@@_exit_point: { }
%    \end{macrocode}
% \end{macro}
%
% \begin{macro}[EXP]{\@@_break:w, \@@_break_point:}
%   Useful to jump out of complicated conditionals.
%    \begin{macrocode}
\cs_new_eq:NN \@@_break_point: \prg_do_nothing:
\cs_new:Npn \@@_break:w #1 \@@_break_point: { }
%    \end{macrocode}
% \end{macro}
%
% \begin{macro}{\@@_cmd_if_internal:TF}
%   Test whether the \cs{l_@@_head_cmd_int} denotes an ``internal''
%   command, between |min_internal| and |max_internal| (see
%   Section~\ref{sec:numeric-codes}).
%    \begin{macrocode}
\prg_new_conditional:Npnn \@@_cmd_if_internal: { TF }
  {
    \int_compare:nNnTF
      \l_@@_head_cmd_int < { \@@_tex_use:n { min_internal } }
      { \prg_return_false: }
      {
        \int_compare:nNnTF
          \l_@@_head_cmd_int
          > { \@@_tex_use:n { max_internal } }
          { \prg_return_false: }
          { \prg_return_true: }
      }
  }
%    \end{macrocode}
% \end{macro}
%
% \subsubsection{Helpers concerning tokens}
%
% \begin{macro}{\@@_token_to_char:N}
% \begin{macro}{\@@_meaning_to_char:n, \@@_meaning_to_char:o}
% \begin{macro}
%   {\@@_meaning_to_char_auxi:w, \@@_meaning_to_char_auxii:w}
%   From the meaning of a character token (with arbitrary character
%   code, except active), extract the character itself (with string
%   category codes).  This is somewhat robust against wrong input.
%    \begin{macrocode}
\cs_new:Npn \@@_meaning_to_char:n #1
  { \@@_meaning_to_char_auxi:w #1 \q_mark ~ {} ~ \q_mark \q_stop }
\cs_new:Npn \@@_meaning_to_char_auxi:w #1 ~ #2 ~ #3 \q_mark #4 \q_stop
  { \@@_meaning_to_char_auxii:w #3 ~ #3 ~ \q_stop }
\cs_new:Npn \@@_meaning_to_char_auxii:w #1 ~ #2 ~ #3 \q_stop
  { \tl_if_empty:nTF {#2} { ~ } {#2} }
\cs_generate_variant:Nn \@@_meaning_to_char:n { o }
\cs_new:Npn \@@_token_to_char:N #1
  { \@@_meaning_to_char:o { \token_to_meaning:N #1 } }
%    \end{macrocode}
% \end{macro}
% \end{macro}
% \end{macro}
%
% \begin{macro}[EXP,pTF]{\@@_token_if_expandable:N}
%   We need to cook up our own version of \cs{token_if_expandable:NTF}
%   because the \pkg{expl3} one does not think that |undefined| is
%   expandable.
%    \begin{macrocode}
\prg_new_conditional:Npnn \@@_token_if_expandable:N #1
  { p , T ,  F , TF }
  {
    \exp_after:wN \if_meaning:w \exp_not:N #1 #1
      \prg_return_false:
    \else:
      \prg_return_true:
    \fi:
  }
%    \end{macrocode}
% \end{macro}
%
% \begin{macro}[EXP,pTF]{\@@_token_if_protected:N}
%   Returns \texttt{true} if the token is either not expandable or is a
%   protected macro.
%    \begin{macrocode}
\prg_new_conditional:Npnn \@@_token_if_protected:N #1
  { p , T ,  F , TF }
  {
    \@@_token_if_expandable:NTF #1
      {
        \token_if_protected_macro:NTF #1
          { \prg_return_true: }
          {
            \token_if_protected_long_macro:NTF #1
              { \prg_return_true: }
              { \prg_return_false: }
          }
      }
      { \prg_return_true: }
  }
%    \end{macrocode}
% \end{macro}
%
% \begin{macro}[TF]{\@@_token_if_active_char:N}
%   Lowercase the token after setting its \tn{lccode} (more precisely
%   the \tn{lccode} of the first character in its string representation)
%   to a known value, then compare the result with that active
%   character.
%    \begin{macrocode}
\group_begin:
  \char_set_catcode_active:n { `Z }
  \prg_new_protected_conditional:Npnn \@@_token_if_active_char:N #1
    { TF }
    {
      \group_begin:
        \@@_exp_args:Nx \char_set_lccode:nn
          { ` \exp_args:No \str_head:n { \token_to_str:N #1 } }
          { ` Z }
        \tex_lowercase:D { \tl_if_eq:nnTF {#1} } { Z }
          { \group_end: \prg_return_true: }
          { \group_end: \prg_return_false: }
    }
\group_end:
%    \end{macrocode}
% \end{macro}
%
% \begin{macro}[TF]{\@@_token_if_definable:N}
%   Within a group, set the escape character to a non-space value
%   (backslash).  Convert the token to a string with
%   \cs{token_to_str:N}.  The result is multiple characters if the token
%   is a control sequence, and a single character otherwise (even for
%   explicit catcode~$6$ character tokens which would be doubled if we
%   used \cs{tl_to_str:n} instead of \cs{token_to_str:N}).  Thus
%   \cs{str_tail:n} gives a non-empty result exactly for control
%   sequences.  Those are definable (technically, not always:
%   |\expandafter\font\csname\endcsname=cmr10|
%   |\expandafter\def\the\csname\endcsname{}|).  For characters just
%   check for active characters.  In both cases remember to end the
%   group.
%    \begin{macrocode}
\group_begin:
  \char_set_catcode_active:n { `Z }
  \prg_new_protected_conditional:Npnn \@@_token_if_definable:N #1
    { TF }
    {
      \group_begin:
        \@@_set_escapechar:n { 92 }
        \tl_set:Nx \l_@@_tmpa_tl
          { \exp_args:No \str_tail:n { \token_to_str:N #1 } }
        \tl_if_empty:NTF \l_@@_tmpa_tl
          {
            \@@_token_if_active_char:NTF #1
              { \group_end: \prg_return_true: }
              { \group_end: \prg_return_false: }
          }
          { \group_end: \prg_return_true: }
    }
\group_end:
%    \end{macrocode}
% \end{macro}
%
% \begin{macro}[TF]{\@@_gtl_if_head_is_definable:N}
%   Tests if a generalized token list is a single control sequence or a
%   single active character.  First test that it is single, then filter
%   out the case of (explicit) begin-group, end-group, and blank space
%   characters: those are neither control sequences nor active.  Then
%   feed the single normal token to a first auxiliary.
%    \begin{macrocode}
\prg_new_protected_conditional:Npnn \@@_gtl_if_head_is_definable:N #1
  { TF , F }
  {
    \gtl_if_single_token:NTF #1
      {
        \gtl_if_head_is_N_type:NTF #1
          {
            \gtl_head_do:NN #1 \@@_token_if_definable:NTF
              { \prg_return_true: }
              { \prg_return_false: }
          }
          { \prg_return_false: }
      }
      { \prg_return_false: }
  }
%    \end{macrocode}
% \end{macro}
%
% \subsubsection{Helpers for previous input}
%
% \begin{macro}[EXP]{\@@_prev_input_count:, \@@_prev_input_count_aux:n}
%    \begin{macrocode}
\cs_new:Npn \@@_prev_input_count:
  {
    \int_eval:n
      {
        0
        \seq_map_function:NN \g_@@_prev_input_seq
          \@@_prev_input_count_aux:n
      }
  }
\cs_new:Npn \@@_prev_input_count_aux:n #1
  { \tl_if_empty:nF {#1} { + 1 } }
%    \end{macrocode}
% \end{macro}
%
% \begin{macro}
%   {
%     \@@_prev_input_get:N,
%     \@@_prev_input_gpush:,
%     \@@_prev_input_gpush:N,
%     \@@_prev_input_gpop:N,
%     \@@_prev_input_gpush_gtl:,
%     \@@_prev_input_gpush_gtl:N,
%     \@@_prev_input_gpop_gtl:N,
%   }
% ^^A todo: doc and add optional sanity checking
%    \begin{macrocode}
\cs_new_protected:Npn \@@_prev_input_get:N
  { \seq_get_right:NN \g_@@_prev_input_seq }
\cs_new_protected:Npn \@@_prev_input_gpush:
  { \seq_gput_right:Nn \g_@@_prev_input_seq { } }
\cs_new_protected:Npn \@@_prev_input_gpush:N
  { \seq_gput_right:NV \g_@@_prev_input_seq }
\cs_new_protected:Npn \@@_prev_input_gpop:N
  { \seq_gpop_right:NN \g_@@_prev_input_seq }
\cs_new_protected:Npn \@@_prev_input_gpush_gtl:
  { \seq_gput_right:NV \g_@@_prev_input_seq \c_empty_gtl }
\cs_new_protected:Npn \@@_prev_input_gpush_gtl:N
  { \seq_gput_right:NV \g_@@_prev_input_seq }
\cs_new_protected:Npn \@@_prev_input_gpop_gtl:N
  { \seq_gpop_right:NN \g_@@_prev_input_seq }
%    \end{macrocode}
% \end{macro}
%
% \begin{macro}
%   {
%     \@@_prev_input_silent:n,
%     \@@_prev_input_silent:V,
%     \@@_prev_input_silent:x
%   }
% \begin{macro}
%   {
%     \@@_prev_input:n,
%     \@@_prev_input:V,
%     \@@_prev_input:x
%   }
%    \begin{macrocode}
\cs_new_protected:Npn \@@_prev_input_silent:n #1
  {
    \@@_prev_input_gpop:N \l_@@_prev_input_tl
    \tl_put_right:Nn \l_@@_prev_input_tl {#1}
    \@@_prev_input_gpush:N \l_@@_prev_input_tl
  }
\cs_generate_variant:Nn \@@_prev_input_silent:n { V }
\cs_new_protected:Npn \@@_prev_input_silent:x
  { \@@_exp_args:Nx \@@_prev_input_silent:n }
\cs_new_protected:Npn \@@_prev_input:n #1
  {
    \@@_prev_input_silent:n {#1}
    \@@_print_action:x { \tl_to_str:n {#1} }
  }
\cs_generate_variant:Nn \@@_prev_input:n { V }
\cs_new_protected:Npn \@@_prev_input:x
  { \@@_exp_args:Nx \@@_prev_input:n }
%    \end{macrocode}
% \end{macro}
% \end{macro}
%
% \begin{macro}{\@@_prev_input_gtl:N}
%    \begin{macrocode}
\cs_new_protected:Npn \@@_prev_input_gtl:N #1
  {
    \@@_prev_input_gpop_gtl:N \l_@@_prev_input_gtl
    \gtl_concat:NNN \l_@@_prev_input_gtl \l_@@_prev_input_gtl #1
    \@@_prev_input_gpush_gtl:N \l_@@_prev_input_gtl
  }
%    \end{macrocode}
% \end{macro}
%
% \begin{macro}{\@@_prev_input_join_get:nN}
% \begin{macro}{\@@_join_get_aux:NNN}
%   Pops the previous-input sequence twice to get some value in
%   \cs{l_@@_head_tl} and some sign or decimal number in
%   \cs{l_@@_tmpa_tl}.  Combines them into a value, using
%   the appropriate evaluation function, determined based
%   on~|#1|.
%    \begin{macrocode}
\cs_new_protected:Npn \@@_prev_input_join_get:nN #1
  {
    \int_case:nnF {#1}
      {
        { 2 } { \@@_join_get_aux:NNN \skip_eval:n \etex_glueexpr:D }
        { 3 } { \@@_join_get_aux:NNN \muskip_eval:n \etex_muexpr:D }
      }
      {
        \@@_error:nnnnn { internal } { join-factor } { } { } { }
        \@@_join_get_aux:NNN \use:n \prg_do_nothing:
      }
  }
\cs_new_protected:Npn \@@_join_get_aux:NNN #1#2#3
  {
    \@@_prev_input_gpop:N \l_@@_head_tl
    \@@_prev_input_gpop:N \l_@@_tmpa_tl
    \tl_set:Nx #3 { #1 { \l_@@_tmpa_tl #2 \l_@@_head_tl } }
  }
%    \end{macrocode}
% \end{macro}
% \end{macro}
%
% \subsection{Variables}
%
% \subsubsection{User interaction}
%
% \begin{variable}{\g_@@_before_print_state_tl, \g_@@_before_prompt_tl}
%   Code to run before printing the state or before the prompt.
%    \begin{macrocode}
\tl_new:N \g_@@_before_print_state_tl
\tl_new:N  \g_@@_before_prompt_tl
%    \end{macrocode}
% \end{variable}
%
% \begin{variable}{\l_@@_prompt_tmpa_int}
%    \begin{macrocode}
\int_new:N \l_@@_prompt_tmpa_int
%    \end{macrocode}
% \end{variable}
%
% \begin{variable}{\g_@@_nonstop_int}
%   The number of prompts to skip.
%    \begin{macrocode}
\int_new:N \g_@@_nonstop_int
%    \end{macrocode}
% \end{variable}
%
% \begin{variable}
%   {
%     \g_@@_default_explicit_prompt_bool , \g_@@_explicit_prompt_bool ,
%     \g_@@_default_internal_debug_bool  , \g_@@_internal_debug_bool  ,
%     \g_@@_default_number_steps_bool    , \g_@@_number_steps_bool    ,
%     \g_@@_default_online_int           , \g_@@_online_int           ,
%     \g_@@_default_trace_assigns_bool   , \g_@@_trace_assigns_bool   ,
%     \g_@@_default_trace_expansion_bool , \g_@@_trace_expansion_bool ,
%     \g_@@_default_trace_other_bool     , \g_@@_trace_other_bool     ,
%     \g_@@_default_welcome_message_bool , \g_@@_welcome_message_bool ,
%   }
%   Variables for the options |explicit-prompt|,
%   |internal-debug|, |number-steps|, and so on.  The
%   \texttt{default_} booleans/integers store the default value of these
%   options, and are affected by \cs{unravelsetup} or
%   \cs{unravel_setup:n}.
%    \begin{macrocode}
\bool_new:N \g_@@_default_explicit_prompt_bool
\bool_new:N \g_@@_default_internal_debug_bool
\bool_new:N \g_@@_default_number_steps_bool
\int_new:N  \g_@@_default_online_int
\bool_new:N \g_@@_default_trace_assigns_bool
\bool_new:N \g_@@_default_trace_expansion_bool
\bool_new:N \g_@@_default_trace_other_bool
\bool_new:N \g_@@_default_welcome_message_bool
\bool_gset_true:N \g_@@_default_number_steps_bool
\int_gset_eq:NN   \g_@@_default_online_int \c_one
\bool_gset_true:N \g_@@_default_trace_assigns_bool
\bool_gset_true:N \g_@@_default_trace_expansion_bool
\bool_gset_true:N \g_@@_default_trace_other_bool
\bool_gset_true:N \g_@@_default_welcome_message_bool
\bool_new:N \g_@@_explicit_prompt_bool
\bool_new:N \g_@@_internal_debug_bool
\bool_new:N \g_@@_number_steps_bool
\int_new:N  \g_@@_online_int
\bool_new:N \g_@@_trace_assigns_bool
\bool_new:N \g_@@_trace_expansion_bool
\bool_new:N \g_@@_trace_other_bool
\bool_new:N \g_@@_welcome_message_bool
%    \end{macrocode}
% \end{variable}
%
% \begin{variable}{\g_@@_step_int}
%   Current expansion step.
%    \begin{macrocode}
\int_new:N \g_@@_step_int
%    \end{macrocode}
% \end{variable}
%
% \begin{variable}{\g_@@_action_text_str}
%   Text describing the action, displayed at each step.  This should
%   only be altered through \cs{@@_set_action_text:x}, which sets the
%   escape character as appropriate before converting the argument to a
%   string.
%    \begin{macrocode}
\str_new:N \g_@@_action_text_str
%    \end{macrocode}
% \end{variable}
%
% \begin{variable}
%   {
%     \g_@@_default_max_action_int,
%     \g_@@_default_max_output_int,
%     \g_@@_default_max_input_int,
%     \g_@@_max_action_int,
%     \g_@@_max_output_int,
%     \g_@@_max_input_int
%   }
%   Maximum length of various pieces of what is shown on the terminal.
%    \begin{macrocode}
\int_new:N \g_@@_default_max_action_int
\int_new:N \g_@@_default_max_output_int
\int_new:N \g_@@_default_max_input_int
\int_gset:Nn \g_@@_default_max_action_int { 50 }
\int_gset:Nn \g_@@_default_max_output_int { 300 }
\int_gset:Nn \g_@@_default_max_input_int { 300 }
\int_new:N \g_@@_max_action_int
\int_new:N \g_@@_max_output_int
\int_new:N \g_@@_max_input_int
%    \end{macrocode}
% \end{variable}
%
% \begin{variable}{\g_@@_speedup_macros_bool}
%   If this boolean is true, speed up macros which have a simple
%   parameter text.  This may not be safe if very weird macros appear.
%    \begin{macrocode}
\bool_new:N \g_@@_speedup_macros_bool
\bool_gset_true:N \g_@@_speedup_macros_bool
%    \end{macrocode}
% \end{variable}
%
% \begin{variable}{\l_@@_print_int}
%   The length of one piece of the terminal output.
%    \begin{macrocode}
\int_new:N \l_@@_print_int
%    \end{macrocode}
% \end{variable}
%
% \subsubsection{Working with tokens}
%
% \begin{variable}{\g_@@_input_int}
%   The user input, at each stage of expansion, is stored in multiple
%   \texttt{gtl} variables, from |\g_@@_input_|\meta{n}|_gtl| to
%   \cs{g_@@_input_1_gtl}.  The split between variables is akin to
%   \TeX{}'s input stack, and allows us to manipulate smaller token
%   lists, speeding up processing.  The total number \meta{n} of lists
%   is \cs{g_@@_input_int}.  The highest numbered \texttt{gtl}
%   represents input that comes to the left of lower numbered ones.
%    \begin{macrocode}
\int_new:N \g_@@_input_int
%    \end{macrocode}
% \end{variable}
%
% \begin{variable}{\g_@@_input_tmpa_int}
% \begin{variable}{\l_@@_input_tmpa_tl}
%    \begin{macrocode}
\int_new:N \g_@@_input_tmpa_int
\tl_new:N \l_@@_input_tmpa_tl
%    \end{macrocode}
% \end{variable}
% \end{variable}
%
% \begin{variable}
%   {\g_@@_prev_input_seq, \l_@@_prev_input_tl, \l_@@_prev_input_gtl}
%   The different levels of expansion are stored in
%   \cs{g_@@_prev_input_seq}, with the innermost at the end of the
%   sequence (otherwise the sequence would have to be reversed for
%   display).  When adding material to the last level of expansion,
%   \cs{l_@@_prev_input_tl} or \cs{l_@@_prev_input_gtl} are used to
%   temporarily store the last level of expansion.
%    \begin{macrocode}
\seq_new:N \g_@@_prev_input_seq
\tl_new:N \l_@@_prev_input_tl
\gtl_new:N \l_@@_prev_input_gtl
%    \end{macrocode}
% \end{variable}
%
% \begin{variable}{\g_@@_output_gtl}
%   Material that is ``typeset'' or otherwise sent further down \TeX{}'s
%   digestion.
%    \begin{macrocode}
\gtl_new:N \g_@@_output_gtl
%    \end{macrocode}
% \end{variable}
%
% \begin{variable}
%   {
%     \l_@@_head_gtl,
%     \l_@@_head_tl,
%     \l_@@_head_token,
%     \l_@@_head_cmd_int,
%     \l_@@_head_char_int
%   }
%   First token in the input, as a generalized token list (general case)
%   or as a token list whenever this is possible.  Also, a token set
%   equal to it, and its command code and character code, following
%   \TeX{}.
%    \begin{macrocode}
\gtl_new:N \l_@@_head_gtl
\tl_new:N  \l_@@_head_tl
\cs_new_eq:NN \l_@@_head_token ?
\int_new:N \l_@@_head_cmd_int
\int_new:N \l_@@_head_char_int
%    \end{macrocode}
% \end{variable}
%
% \begin{variable}{\l_@@_head_meaning_tl}
%    \begin{macrocode}
\tl_new:N \l_@@_head_meaning_tl
%    \end{macrocode}
% \end{variable}
%
% \begin{variable}{\l_@@_argi_tl, \l_@@_argii_tl}
%   Token list variables used to store first/second arguments.
%    \begin{macrocode}
\tl_new:N \l_@@_argi_tl
\tl_new:N \l_@@_argii_tl
%    \end{macrocode}
% \end{variable}
%
% \begin{variable}{\l_@@_tmpa_tl, \l_@@_tmpb_gtl, \g_@@_tmpc_tl, \l_@@_tmpa_seq, \l_@@_unused_gtl, \l_@@_tmpb_token}
%   Temporary storage.  The \cs{l_@@_unused_gtl} is only used once,
%   to ignore some unwanted tokens.
%    \begin{macrocode}
\tl_new:N \l_@@_tmpa_tl
\gtl_new:N \l_@@_unused_gtl
\gtl_new:N \l_@@_tmpb_gtl
\tl_new:N \g_@@_tmpc_tl
\seq_new:N \l_@@_tmpa_seq
\cs_new_eq:NN \l_@@_tmpb_token ?
%    \end{macrocode}
% \end{variable}
%
% \begin{variable}{\l_@@_defined_tl, \l_@@_defining_tl}
%   The token that is defined by the prefixed command (such as
%   \tn{chardef} or \tn{futurelet}), and the code to define it.  We do
%   not use the the previous-input sequence to store that code: rather,
%   this sequence contains a string representation of the code, which is
%   not suitable for the definition.  This is safe, as definitions
%   cannot be nested.  This is needed for expanding assignments, as
%   expansion should be shown to the user, but then later should not be
%   performed again when defining.
%    \begin{macrocode}
\tl_new:N \l_@@_defined_tl
\tl_new:N \l_@@_defining_tl
%    \end{macrocode}
% \end{variable}
%
% \begin{macro}{\@@_inaccessible:w}
%    \begin{macrocode}
\cs_new_eq:NN \@@_inaccessible:w ?
%    \end{macrocode}
% \end{macro}
%
% \begin{variable}
%   {
%     \g_@@_after_assignment_gtl,
%     \g_@@_set_box_allowed_bool,
%     \g_@@_name_in_progress_bool
%   }
%   Global variables keeping track of the state of \TeX{}.  Token to
%   insert after the next assignment.  Is \tn{setbox} currently allowed?
%   Should \tn{input} expand?
%    \begin{macrocode}
\gtl_new:N \g_@@_after_assignment_gtl
\bool_new:N \g_@@_set_box_allowed_bool
\bool_new:N \g_@@_name_in_progress_bool
%    \end{macrocode}
% \end{variable}
%
% \begin{variable}{\l_@@_after_group_gtl}
%   Tokens to insert after the current group ends.  This variable must
%   be emptied at the beginning of every group.
%    \begin{macrocode}
\gtl_new:N \l_@@_after_group_gtl
%    \end{macrocode}
% \end{variable}
%
% \begin{variable}{\c_@@_parameters_tl}
%   Used to determine if a macro has simple parameters or not.
%    \begin{macrocode}
\group_begin:
  \cs_set:Npx \@@_tmp:w #1 { \c_hash_str #1 }
  \tl_const:Nx \c_@@_parameters_tl
    { ^ \tl_map_function:nN { 123456789 } \@@_tmp:w }
\group_end:
%    \end{macrocode}
% \end{variable}
%
% \subsubsection{Numbers and conditionals}
%
% \begin{variable}{\g_@@_val_level_int}
%   See \TeX{}'s |cur_val_level| variable.  This is set by
%   \cs{@@_scan_something_internal:n} to
%   \begin{itemize}
%   \item $0$ for integer values,
%   \item $1$ for dimension values,
%   \item $2$ for glue values,
%   \item $3$ for mu glue values,
%   \item $4$ for font identifiers,
%   \item $5$ for token lists.
%   \end{itemize}
%    \begin{macrocode}
\int_new:N \g_@@_val_level_int
%    \end{macrocode}
% \end{variable}
%
% \begin{variable}{\g_@@_if_limit_tl}
% \begin{variable}{\g_@@_if_limit_int}
% \begin{variable}{\g_@@_if_depth_int}
%   Stack for what \TeX{} calls |if_limit|, and its depth.
%    \begin{macrocode}
\tl_new:N \g_@@_if_limit_tl
\int_new:N \g_@@_if_limit_int
\int_new:N \g_@@_if_depth_int
%    \end{macrocode}
% \end{variable}
% \end{variable}
% \end{variable}
%
% \begin{variable}{\l_@@_if_nesting_int}
%    \begin{macrocode}
\int_new:N \l_@@_if_nesting_int
%    \end{macrocode}
% \end{variable}
%
% \subsubsection{Boxes and groups}
%
% \begin{variable}{\l_@@_leaders_box_seq}
%   A stack of letters: the first token in the token list is |h|~if the
%   innermost explicit box (created with \tn{vtop}, \tn{vbox}, or
%   \tn{hbox}) appears in a horizontal (or math) mode leaders
%   construction; it is |v|~if the innermost explicit box appears in a
%   vertical mode leaders construction; it is |Z|~otherwise.
%    \begin{macrocode}
\seq_new:N \l_@@_leaders_box_seq
%    \end{macrocode}
% \end{variable}
%
% \begin{variable}{\g_@@_ends_int}
%   Number of times \tn{end} will be put back into the input in case
%   there remains to ship some pages.
%    \begin{macrocode}
\int_new:N \g_@@_ends_int
\int_gset:Nn \g_@@_ends_int { 3 }
%    \end{macrocode}
% \end{variable}
%
% \subsubsection{Constants}
%
% \begin{variable}
%   {
%     \c_@@_plus_tl, \c_@@_minus_tl, \c_@@_times_tl, \c_@@_over_tl,
%     \c_@@_lq_tl, \c_@@_rq_tl, \c_@@_dq_tl, \c_@@_lp_tl, \c_@@_rp_tl,
%     \c_@@_eq_tl, \c_@@_comma_tl, \c_@@_point_tl,
%   }
%    \begin{macrocode}
\tl_const:Nn \c_@@_plus_tl { + }
\tl_const:Nn \c_@@_minus_tl { - }
\tl_const:Nn \c_@@_times_tl { * }
\tl_const:Nn \c_@@_over_tl { / }
\tl_const:Nn \c_@@_lq_tl { ` }
\tl_const:Nn \c_@@_rq_tl { ' }
\tl_const:Nn \c_@@_dq_tl { " }
\tl_const:Nn \c_@@_lp_tl { ( }
\tl_const:Nn \c_@@_rp_tl { ) }
\tl_const:Nn \c_@@_eq_tl { = }
\tl_const:Nn \c_@@_comma_tl { , }
\tl_const:Nn \c_@@_point_tl { . }
%    \end{macrocode}
% \end{variable}
%
% \begin{variable}{\c_@@_frozen_relax_gtl}
%   \TeX{}'s |frozen_relax|, inserted by \cs{@@_insert_relax:}.
%    \begin{macrocode}
\gtl_const:Nx \c_@@_frozen_relax_gtl { \if_int_compare:w 0 = 0 \fi: }
%    \end{macrocode}
% \end{variable}
%
% \subsubsection{\TeX{} parameters}
%
% \begin{variable}{\g_@@_mag_set_int}
%   The first time \TeX{} uses the value of \tn{mag}, it stores it in a
%   global parameter |mag_set| (initially $0$ to denote not being set).
%   Any time \TeX{} needs the value of \tn{mag}, it checks that the
%   value matches |mag_set|.  This is done in \pkg{unravel} by
%   \cs{@@_prepare_mag:}, storing |mag_set| in \cs{g_@@_mag_set_int}.
%    \begin{macrocode}
\int_new:N \g_@@_mag_set_int
%    \end{macrocode}
% \end{variable}
%
% \subsection{Numeric codes}
% \label{sec:numeric-codes}
%
% First we define some numeric codes, following Section~15 of the \TeX{}
% web code, then we associate a command code to each \TeX{} primitive,
% and a character code, to decide what action to perform upon seeing
% them.
%
% \begin{macro}{\@@_tex_const:nn}
% \begin{macro}[EXP]{\@@_tex_use:n}
%    \begin{macrocode}
\cs_new_protected:Npn \@@_tex_const:nn #1#2
  { \int_const:cn { c_@@_tex_#1_int } {#2} }
\cs_new:Npn \@@_tex_use:n #1 { \int_use:c { c_@@_tex_#1_int } }
%    \end{macrocode}
% \end{macro}
% \end{macro}
%
% \begin{macro}{\@@_tex_primitive:nnn}
%    \begin{macrocode}
\cs_new_protected:Npn \@@_tex_primitive:nnn #1#2#3
  {
    \tl_const:cx { c_@@_tex_#1_tl }
      { { \@@_tex_use:n {#2} } {#3} }
  }
%    \end{macrocode}
% \end{macro}
%
% \begin{macro}{\@@_new_tex_cmd:nn, \@@_new_eq_tex_cmd:nn}
%    \begin{macrocode}
\cs_new_protected:Npn \@@_new_tex_cmd:nn #1#2
  {
    \cs_new_protected:cpn
      { @@_cmd_ \@@_tex_use:n {#1} : } {#2}
  }
\cs_new_protected:Npn \@@_new_eq_tex_cmd:nn #1#2
  {
    \cs_new_eq:cc
      { @@_cmd_ \@@_tex_use:n {#1} : }
      { @@_cmd_ \@@_tex_use:n {#2} : }
  }
%    \end{macrocode}
% \end{macro}
%
% \begin{macro}{\@@_new_tex_expandable:nn}
%    \begin{macrocode}
\cs_new_protected:Npn \@@_new_tex_expandable:nn #1#2
  {
    \cs_new_protected:cpn
      { @@_expandable_ \@@_tex_use:n {#1} : } {#2}
  }
%    \end{macrocode}
% \end{macro}
%
% Contrarily to \TeX{}, all macros are |call|, no |long_call| and the
% like.
%    \begin{macrocode}
\@@_tex_const:nn { relax                    } { 0 }
\@@_tex_const:nn { begin-group_char         } { 1 }
\@@_tex_const:nn { end-group_char           } { 2 }
\@@_tex_const:nn { math_char                } { 3 }
\@@_tex_const:nn { tab_mark                 } { 4 }
\@@_tex_const:nn { alignment_char           } { 4 }
\@@_tex_const:nn { car_ret                  } { 5 }
\@@_tex_const:nn { macro_char               } { 6 }
\@@_tex_const:nn { superscript_char         } { 7 }
\@@_tex_const:nn { subscript_char           } { 8 }
\@@_tex_const:nn { endv                     } { 9 }
\@@_tex_const:nn { blank_char               } { 10 }
\@@_tex_const:nn { the_char                 } { 11 }
\@@_tex_const:nn { other_char               } { 12 }
\@@_tex_const:nn { par_end                  } { 13 }
\@@_tex_const:nn { stop                     } { 14 }
\@@_tex_const:nn { delim_num                } { 15 }
\@@_tex_const:nn { max_char_code            } { 15 }
\@@_tex_const:nn { char_num                 } { 16 }
\@@_tex_const:nn { math_char_num            } { 17 }
\@@_tex_const:nn { mark                     } { 18 }
\@@_tex_const:nn { xray                     } { 19 }
\@@_tex_const:nn { make_box                 } { 20 }
\@@_tex_const:nn { hmove                    } { 21 }
\@@_tex_const:nn { vmove                    } { 22 }
\@@_tex_const:nn { un_hbox                  } { 23 }
\@@_tex_const:nn { un_vbox                  } { 24 }
\@@_tex_const:nn { remove_item              } { 25 }
\@@_tex_const:nn { hskip                    } { 26 }
\@@_tex_const:nn { vskip                    } { 27 }
\@@_tex_const:nn { mskip                    } { 28 }
\@@_tex_const:nn { kern                     } { 29 }
\@@_tex_const:nn { mkern                    } { 30 }
\@@_tex_const:nn { leader_ship              } { 31 }
\@@_tex_const:nn { halign                   } { 32 }
\@@_tex_const:nn { valign                   } { 33 }
\@@_tex_const:nn { no_align                 } { 34 }
\@@_tex_const:nn { vrule                    } { 35 }
\@@_tex_const:nn { hrule                    } { 36 }
\@@_tex_const:nn { insert                   } { 37 }
\@@_tex_const:nn { vadjust                  } { 38 }
\@@_tex_const:nn { ignore_spaces            } { 39 }
\@@_tex_const:nn { after_assignment         } { 40 }
\@@_tex_const:nn { after_group              } { 41 }
\@@_tex_const:nn { break_penalty            } { 42 }
\@@_tex_const:nn { start_par                } { 43 }
\@@_tex_const:nn { ital_corr                } { 44 }
\@@_tex_const:nn { accent                   } { 45 }
\@@_tex_const:nn { math_accent              } { 46 }
\@@_tex_const:nn { discretionary            } { 47 }
\@@_tex_const:nn { eq_no                    } { 48 }
\@@_tex_const:nn { left_right               } { 49 }
\@@_tex_const:nn { math_comp                } { 50 }
\@@_tex_const:nn { limit_switch             } { 51 }
\@@_tex_const:nn { above                    } { 52 }
\@@_tex_const:nn { math_style               } { 53 }
\@@_tex_const:nn { math_choice              } { 54 }
\@@_tex_const:nn { non_script               } { 55 }
\@@_tex_const:nn { vcenter                  } { 56 }
\@@_tex_const:nn { case_shift               } { 57 }
\@@_tex_const:nn { message                  } { 58 }
\@@_tex_const:nn { extension                } { 59 }
\@@_tex_const:nn { in_stream                } { 60 }
\@@_tex_const:nn { begin_group              } { 61 }
\@@_tex_const:nn { end_group                } { 62 }
\@@_tex_const:nn { omit                     } { 63 }
\@@_tex_const:nn { ex_space                 } { 64 }
\@@_tex_const:nn { no_boundary              } { 65 }
\@@_tex_const:nn { radical                  } { 66 }
\@@_tex_const:nn { end_cs_name              } { 67 }
\@@_tex_const:nn { min_internal             } { 68 }
\@@_tex_const:nn { char_given               } { 68 }
\@@_tex_const:nn { math_given               } { 69 }
\@@_tex_const:nn { last_item                } { 70 }
\@@_tex_const:nn { max_non_prefixed_command } { 70 }
\@@_tex_const:nn { toks_register            } { 71 }
\@@_tex_const:nn { assign_toks              } { 72 }
\@@_tex_const:nn { assign_int               } { 73 }
\@@_tex_const:nn { assign_dimen             } { 74 }
\@@_tex_const:nn { assign_glue              } { 75 }
\@@_tex_const:nn { assign_mu_glue           } { 76 }
\@@_tex_const:nn { assign_font_dimen        } { 77 }
\@@_tex_const:nn { assign_font_int          } { 78 }
\@@_tex_const:nn { set_aux                  } { 79 }
\@@_tex_const:nn { set_prev_graf            } { 80 }
\@@_tex_const:nn { set_page_dimen           } { 81 }
\@@_tex_const:nn { set_page_int             } { 82 }
\@@_tex_const:nn { set_box_dimen            } { 83 }
\@@_tex_const:nn { set_shape                } { 84 }
\@@_tex_const:nn { def_code                 } { 85 }
\@@_tex_const:nn { def_family               } { 86 }
\@@_tex_const:nn { set_font                 } { 87 }
\@@_tex_const:nn { def_font                 } { 88 }
\@@_tex_const:nn { register                 } { 89 }
\@@_tex_const:nn { max_internal             } { 89 }
\@@_tex_const:nn { advance                  } { 90 }
\@@_tex_const:nn { multiply                 } { 91 }
\@@_tex_const:nn { divide                   } { 92 }
\@@_tex_const:nn { prefix                   } { 93 }
\@@_tex_const:nn { let                      } { 94 }
\@@_tex_const:nn { shorthand_def            } { 95 }
\@@_tex_const:nn { read_to_cs               } { 96 }
\@@_tex_const:nn { def                      } { 97 }
\@@_tex_const:nn { set_box                  } { 98 }
\@@_tex_const:nn { hyph_data                } { 99 }
\@@_tex_const:nn { set_interaction          } { 100 }
\@@_tex_const:nn { letterspace_font         } { 101 }
\@@_tex_const:nn { pdf_copy_font            } { 102 }
\@@_tex_const:nn { max_command              } { 102 }
\@@_tex_const:nn { undefined_cs             } { 103 }
\@@_tex_const:nn { expand_after             } { 104 }
\@@_tex_const:nn { no_expand                } { 105 }
\@@_tex_const:nn { input                    } { 106 }
\@@_tex_const:nn { if_test                  } { 107 }
\@@_tex_const:nn { fi_or_else               } { 108 }
\@@_tex_const:nn { cs_name                  } { 109 }
\@@_tex_const:nn { convert                  } { 110 }
\@@_tex_const:nn { the                      } { 111 }
\@@_tex_const:nn { top_bot_mark             } { 112 }
\@@_tex_const:nn { call                     } { 113 }
\@@_tex_const:nn { end_template             } { 117 }
%    \end{macrocode}
% So far we've implemented properly [71,104]; [107,113].
%
% A few minor differences with pdf\TeX{}'s internal numbers are as
% follows.
% \begin{itemize}
% \item |case_shift| is shifted by |3983|.
% \item |assign_toks| is shifted by |local_base=3412|.
% \item |assign_int| is shifted by |int_base=5263|.
% \item |assign_dimen| is shifted by |dimen_base=5830|.
% \item |assign_glue| and |assign_mu_glue| are shifted by
%   |glue_base=2882|.
% \item |set_shape| is shifted (in \eTeX{}) by |local_base|.
% \item |def_code| and |def_family| is shifted by |cat_code_base=3983|.
% \item In \TeX{}, |inputlineno.char=3| and |badness.char=4|.
% \end{itemize}
% A special case for \LuaTeX{} deals with the fact that the
% \cs{@@_special_relax:} has a strange meaning \enquote{[unknown command
% code! (0, 1)]}.  For instance \tn{expandafter} \tn{show} \tn{noexpand}
% \tn{undefined} shows this.
%    \begin{macrocode}
\sys_if_engine_luatex:T
  {
    \@@_tex_primitive:nnn
      { \exp_after:wN \use_none:n \token_to_meaning:N \@@_special_relax: }
      { relax } { 1 }
  }
\@@_tex_primitive:nnn { relax             } { relax    } { 256 }
\@@_tex_primitive:nnn { span              } { tab_mark } { 256 }
\@@_tex_primitive:nnn { cr                } { car_ret  } { 257 }
\@@_tex_primitive:nnn { crcr              } { car_ret  } { 258 }
\@@_tex_primitive:nnn { par               } { par_end  } { 256 }
\@@_tex_primitive:nnn { end               } { stop } { 0 }
\@@_tex_primitive:nnn { dump              } { stop } { 1 }
\@@_tex_primitive:nnn { delimiter         } { delim_num } { 0 }
\@@_tex_primitive:nnn { char              } { char_num } { 0 }
\@@_tex_primitive:nnn { mathchar          } { math_char_num } { 0 }
\@@_tex_primitive:nnn { mark              } { mark } { 0 }
\@@_tex_primitive:nnn { marks             } { mark } { 5 }
\@@_tex_primitive:nnn { show              } { xray } { 0 }
\@@_tex_primitive:nnn { showbox           } { xray } { 1 }
\@@_tex_primitive:nnn { showthe           } { xray } { 2 }
\@@_tex_primitive:nnn { showlists         } { xray } { 3 }
\@@_tex_primitive:nnn { showgroups        } { xray } { 4 }
\@@_tex_primitive:nnn { showtokens        } { xray } { 5 }
\@@_tex_primitive:nnn { showifs           } { xray } { 6 }
\@@_tex_primitive:nnn { box               } { make_box } { 0 }
\@@_tex_primitive:nnn { copy              } { make_box } { 1 }
\@@_tex_primitive:nnn { lastbox           } { make_box } { 2 }
\@@_tex_primitive:nnn { vsplit            } { make_box } { 3 }
\@@_tex_primitive:nnn { vtop              } { make_box } { 4 }
\@@_tex_primitive:nnn { vbox              } { make_box } { 5 }
\@@_tex_primitive:nnn { hbox              } { make_box } { 106 }
\@@_tex_primitive:nnn { moveright         } { hmove } { 0 }
\@@_tex_primitive:nnn { moveleft          } { hmove } { 1 }
\@@_tex_primitive:nnn { lower             } { vmove } { 0 }
\@@_tex_primitive:nnn { raise             } { vmove } { 1 }
\@@_tex_primitive:nnn { unhbox            } { un_hbox } { 0 }
\@@_tex_primitive:nnn { unhcopy           } { un_hbox } { 1 }
\@@_tex_primitive:nnn { unvbox            } { un_vbox } { 0 }
\@@_tex_primitive:nnn { unvcopy           } { un_vbox } { 1 }
\@@_tex_primitive:nnn { pagediscards      } { un_vbox } { 2 }
\@@_tex_primitive:nnn { splitdiscards     } { un_vbox } { 3 }
\@@_tex_primitive:nnn { unpenalty         } { remove_item } { 12 }
\@@_tex_primitive:nnn { unkern            } { remove_item } { 11 }
\@@_tex_primitive:nnn { unskip            } { remove_item } { 10 }
\@@_tex_primitive:nnn { hfil              } { hskip } { 0 }
\@@_tex_primitive:nnn { hfill             } { hskip } { 1 }
\@@_tex_primitive:nnn { hss               } { hskip } { 2 }
\@@_tex_primitive:nnn { hfilneg           } { hskip } { 3 }
\@@_tex_primitive:nnn { hskip             } { hskip } { 4 }
\@@_tex_primitive:nnn { vfil              } { vskip } { 0 }
\@@_tex_primitive:nnn { vfill             } { vskip } { 1 }
\@@_tex_primitive:nnn { vss               } { vskip } { 2 }
\@@_tex_primitive:nnn { vfilneg           } { vskip } { 3 }
\@@_tex_primitive:nnn { vskip             } { vskip } { 4 }
\@@_tex_primitive:nnn { mskip             } { mskip } { 5 }
\@@_tex_primitive:nnn { kern              } { kern } { 1 }
\@@_tex_primitive:nnn { mkern             } { mkern } { 99 }
\@@_tex_primitive:nnn { shipout           } { leader_ship } { 99 }
\@@_tex_primitive:nnn { leaders           } { leader_ship } { 100 }
\@@_tex_primitive:nnn { cleaders          } { leader_ship } { 101 }
\@@_tex_primitive:nnn { xleaders          } { leader_ship } { 102 }
\@@_tex_primitive:nnn { halign            } { halign } { 0 }
\@@_tex_primitive:nnn { valign            } { valign } { 0 }
\@@_tex_primitive:nnn { beginL            } { valign } { 4 }
\@@_tex_primitive:nnn { endL              } { valign } { 5 }
\@@_tex_primitive:nnn { beginR            } { valign } { 8 }
\@@_tex_primitive:nnn { endR              } { valign } { 9 }
\@@_tex_primitive:nnn { noalign           } { no_align } { 0 }
\@@_tex_primitive:nnn { vrule             } { vrule } { 0 }
\@@_tex_primitive:nnn { hrule             } { hrule } { 0 }
\@@_tex_primitive:nnn { insert            } { insert } { 0 }
\@@_tex_primitive:nnn { vadjust           } { vadjust } { 0 }
\@@_tex_primitive:nnn { ignorespaces      } { ignore_spaces } { 0 }
\@@_tex_primitive:nnn { afterassignment   } { after_assignment } { 0 }
\@@_tex_primitive:nnn { aftergroup        } { after_group } { 0 }
\@@_tex_primitive:nnn { penalty           } { break_penalty } { 0 }
\@@_tex_primitive:nnn { indent            } { start_par } { 1 }
\@@_tex_primitive:nnn { noindent          } { start_par } { 0 }
\@@_tex_primitive:nnn { quitvmode         } { start_par } { 2 }
\@@_tex_primitive:nnn { /                 } { ital_corr } { 0 }
\@@_tex_primitive:nnn { accent            } { accent } { 0 }
\@@_tex_primitive:nnn { mathaccent        } { math_accent } { 0 }
\@@_tex_primitive:nnn { -                 } { discretionary } { 1 }
\@@_tex_primitive:nnn { discretionary     } { discretionary } { 0 }
\@@_tex_primitive:nnn { eqno              } { eq_no } { 0 }
\@@_tex_primitive:nnn { leqno             } { eq_no } { 1 }
\@@_tex_primitive:nnn { left              } { left_right } { 30 }
\@@_tex_primitive:nnn { right             } { left_right } { 31 }
\@@_tex_primitive:nnn { middle            } { left_right } { 17 }
\@@_tex_primitive:nnn { mathord           } { math_comp } { 16 }
\@@_tex_primitive:nnn { mathop            } { math_comp } { 17 }
\@@_tex_primitive:nnn { mathbin           } { math_comp } { 18 }
\@@_tex_primitive:nnn { mathrel           } { math_comp } { 19 }
\@@_tex_primitive:nnn { mathopen          } { math_comp } { 20 }
\@@_tex_primitive:nnn { mathclose         } { math_comp } { 21 }
\@@_tex_primitive:nnn { mathpunct         } { math_comp } { 22 }
\@@_tex_primitive:nnn { mathinner         } { math_comp } { 23 }
\@@_tex_primitive:nnn { underline         } { math_comp } { 26 }
\@@_tex_primitive:nnn { overline          } { math_comp } { 27 }
\@@_tex_primitive:nnn { displaylimits     } { limit_switch } { 0 }
\@@_tex_primitive:nnn { limits            } { limit_switch } { 1 }
\@@_tex_primitive:nnn { nolimits          } { limit_switch } { 2 }
\@@_tex_primitive:nnn { above             } { above } { 0 }
\@@_tex_primitive:nnn { over              } { above } { 1 }
\@@_tex_primitive:nnn { atop              } { above } { 2 }
\@@_tex_primitive:nnn { abovewithdelims   } { above } { 3 }
\@@_tex_primitive:nnn { overwithdelims    } { above } { 4 }
\@@_tex_primitive:nnn { atopwithdelims    } { above } { 5 }
\@@_tex_primitive:nnn { displaystyle      } { math_style } { 0 }
\@@_tex_primitive:nnn { textstyle         } { math_style } { 2 }
\@@_tex_primitive:nnn { scriptstyle       } { math_style } { 4 }
\@@_tex_primitive:nnn { scriptscriptstyle } { math_style } { 6 }
\@@_tex_primitive:nnn { mathchoice        } { math_choice } { 0 }
\@@_tex_primitive:nnn { nonscript         } { non_script } { 0 }
\@@_tex_primitive:nnn { vcenter           } { vcenter } { 0 }
\@@_tex_primitive:nnn { lowercase         } { case_shift } { 256 }
\@@_tex_primitive:nnn { uppercase         } { case_shift } { 512 }
\@@_tex_primitive:nnn { message           } { message } { 0 }
\@@_tex_primitive:nnn { errmessage        } { message } { 1 }
\@@_tex_primitive:nnn { openout           } { extension } { 0 }
\@@_tex_primitive:nnn { write             } { extension } { 1 }
\@@_tex_primitive:nnn { closeout          } { extension } { 2 }
\@@_tex_primitive:nnn { special           } { extension } { 3 }
\@@_tex_primitive:nnn { immediate         } { extension } { 4 }
\@@_tex_primitive:nnn { setlanguage       } { extension } { 5 }
\@@_tex_primitive:nnn { pdfliteral        } { extension } { 6 }
\@@_tex_primitive:nnn { pdfobj            } { extension } { 7 }
\@@_tex_primitive:nnn { pdfrefobj         } { extension } { 8 }
\@@_tex_primitive:nnn { pdfxform          } { extension } { 9 }
\@@_tex_primitive:nnn { pdfrefxform       } { extension } { 10 }
\@@_tex_primitive:nnn { pdfximage         } { extension } { 11 }
\@@_tex_primitive:nnn { pdfrefximage      } { extension } { 12 }
\@@_tex_primitive:nnn { pdfannot          } { extension } { 13 }
\@@_tex_primitive:nnn { pdfstartlink      } { extension } { 14 }
\@@_tex_primitive:nnn { pdfendlink        } { extension } { 15 }
\@@_tex_primitive:nnn { pdfoutline        } { extension } { 16 }
\@@_tex_primitive:nnn { pdfdest           } { extension } { 17 }
\@@_tex_primitive:nnn { pdfthread         } { extension } { 18 }
\@@_tex_primitive:nnn { pdfstartthread    } { extension } { 19 }
\@@_tex_primitive:nnn { pdfendthread      } { extension } { 20 }
\@@_tex_primitive:nnn { pdfsavepos        } { extension } { 21 }
\@@_tex_primitive:nnn { pdfinfo           } { extension } { 22 }
\@@_tex_primitive:nnn { pdfcatalog        } { extension } { 23 }
\@@_tex_primitive:nnn { pdfnames          } { extension } { 24 }
\@@_tex_primitive:nnn { pdffontattr       } { extension } { 25 }
\@@_tex_primitive:nnn { pdfincludechars   } { extension } { 26 }
\@@_tex_primitive:nnn { pdfmapfile        } { extension } { 27 }
\@@_tex_primitive:nnn { pdfmapline        } { extension } { 28 }
\@@_tex_primitive:nnn { pdftrailer        } { extension } { 29 }
\@@_tex_primitive:nnn { pdfresettimer     } { extension } { 30 }
\@@_tex_primitive:nnn { pdffontexpand     } { extension } { 31 }
\@@_tex_primitive:nnn { pdfsetrandomseed  } { extension } { 32 }
\@@_tex_primitive:nnn { pdfsnaprefpoint   } { extension } { 33 }
\@@_tex_primitive:nnn { pdfsnapy          } { extension } { 34 }
\@@_tex_primitive:nnn { pdfsnapycomp      } { extension } { 35 }
\@@_tex_primitive:nnn { pdfglyphtounicode } { extension } { 36 }
\@@_tex_primitive:nnn { pdfcolorstack     } { extension } { 37 }
\@@_tex_primitive:nnn { pdfsetmatrix      } { extension } { 38 }
\@@_tex_primitive:nnn { pdfsave           } { extension } { 39 }
\@@_tex_primitive:nnn { pdfrestore        } { extension } { 40 }
\@@_tex_primitive:nnn { pdfnobuiltintounicode } { extension } { 41 }
\@@_tex_primitive:nnn { openin                } { in_stream } { 1 }
\@@_tex_primitive:nnn { closein               } { in_stream } { 0 }
\@@_tex_primitive:nnn { begingroup            } { begin_group } { 0 }
\@@_tex_primitive:nnn { endgroup              } { end_group } { 0 }
\@@_tex_primitive:nnn { omit                  } { omit } { 0 }
\@@_tex_primitive:nnn { ~                     } { ex_space } { 0 }
\@@_tex_primitive:nnn { noboundary            } { no_boundary } { 0 }
\@@_tex_primitive:nnn { radical               } { radical } { 0 }
\@@_tex_primitive:nnn { endcsname             } { end_cs_name } { 0 }
\@@_tex_primitive:nnn { lastpenalty           } { last_item } { 0 }
\@@_tex_primitive:nnn { lastkern              } { last_item } { 1 }
\@@_tex_primitive:nnn { lastskip              } { last_item } { 2 }
\@@_tex_primitive:nnn { lastnodetype          } { last_item } { 3 }
\@@_tex_primitive:nnn { inputlineno           } { last_item } { 4 }
\@@_tex_primitive:nnn { badness               } { last_item } { 5 }
\@@_tex_primitive:nnn { pdftexversion         } { last_item } { 6 }
\@@_tex_primitive:nnn { pdflastobj            } { last_item } { 7 }
\@@_tex_primitive:nnn { pdflastxform          } { last_item } { 8 }
\@@_tex_primitive:nnn { pdflastximage         } { last_item } { 9 }
\@@_tex_primitive:nnn { pdflastximagepages    } { last_item } { 10 }
\@@_tex_primitive:nnn { pdflastannot          } { last_item } { 11 }
\@@_tex_primitive:nnn { pdflastxpos           } { last_item } { 12 }
\@@_tex_primitive:nnn { pdflastypos           } { last_item } { 13 }
\@@_tex_primitive:nnn { pdfretval             } { last_item } { 14 }
\@@_tex_primitive:nnn { pdflastximagecolordepth } { last_item } { 15 }
\@@_tex_primitive:nnn { pdfelapsedtime        } { last_item } { 16 }
\@@_tex_primitive:nnn { pdfshellescape        } { last_item } { 17 }
\@@_tex_primitive:nnn { pdfrandomseed         } { last_item } { 18 }
\@@_tex_primitive:nnn { pdflastlink           } { last_item } { 19 }
\@@_tex_primitive:nnn { eTeXversion           } { last_item } { 20 }
\@@_tex_primitive:nnn { currentgrouplevel     } { last_item } { 21 }
\@@_tex_primitive:nnn { currentgrouptype      } { last_item } { 22 }
\@@_tex_primitive:nnn { currentiflevel        } { last_item } { 23 }
\@@_tex_primitive:nnn { currentiftype         } { last_item } { 24 }
\@@_tex_primitive:nnn { currentifbranch       } { last_item } { 25 }
\@@_tex_primitive:nnn { gluestretchorder      } { last_item } { 26 }
\@@_tex_primitive:nnn { glueshrinkorder       } { last_item } { 27 }
\@@_tex_primitive:nnn { fontcharwd            } { last_item } { 28 }
\@@_tex_primitive:nnn { fontcharht            } { last_item } { 29 }
\@@_tex_primitive:nnn { fontchardp            } { last_item } { 30 }
\@@_tex_primitive:nnn { fontcharic            } { last_item } { 31 }
\@@_tex_primitive:nnn { parshapelength        } { last_item } { 32 }
\@@_tex_primitive:nnn { parshapeindent        } { last_item } { 33 }
\@@_tex_primitive:nnn { parshapedimen         } { last_item } { 34 }
\@@_tex_primitive:nnn { gluestretch           } { last_item } { 35 }
\@@_tex_primitive:nnn { glueshrink            } { last_item } { 36 }
\@@_tex_primitive:nnn { mutoglue              } { last_item } { 37 }
\@@_tex_primitive:nnn { gluetomu              } { last_item } { 38 }
\@@_tex_primitive:nnn { numexpr               } { last_item } { 39 }
\@@_tex_primitive:nnn { dimexpr               } { last_item } { 40 }
\@@_tex_primitive:nnn { glueexpr              } { last_item } { 41 }
\@@_tex_primitive:nnn { muexpr                } { last_item } { 42 }
\@@_tex_primitive:nnn { toks } { toks_register } { 0 }
\@@_tex_primitive:nnn { output                } { assign_toks } { 1 }
\@@_tex_primitive:nnn { everypar              } { assign_toks } { 2 }
\@@_tex_primitive:nnn { everymath             } { assign_toks } { 3 }
\@@_tex_primitive:nnn { everydisplay          } { assign_toks } { 4 }
\@@_tex_primitive:nnn { everyhbox             } { assign_toks } { 5 }
\@@_tex_primitive:nnn { everyvbox             } { assign_toks } { 6 }
\@@_tex_primitive:nnn { everyjob              } { assign_toks } { 7 }
\@@_tex_primitive:nnn { everycr               } { assign_toks } { 8 }
\@@_tex_primitive:nnn { errhelp               } { assign_toks } { 9 }
\@@_tex_primitive:nnn { pdfpagesattr          } { assign_toks } { 10 }
\@@_tex_primitive:nnn { pdfpageattr           } { assign_toks } { 11 }
\@@_tex_primitive:nnn { pdfpageresources      } { assign_toks } { 12 }
\@@_tex_primitive:nnn { pdfpkmode             } { assign_toks } { 13 }
\@@_tex_primitive:nnn { everyeof              } { assign_toks } { 14 }
\@@_tex_primitive:nnn { pretolerance          } { assign_int } { 0 }
\@@_tex_primitive:nnn { tolerance             } { assign_int } { 1 }
\@@_tex_primitive:nnn { linepenalty           } { assign_int } { 2 }
\@@_tex_primitive:nnn { hyphenpenalty         } { assign_int } { 3 }
\@@_tex_primitive:nnn { exhyphenpenalty       } { assign_int } { 4 }
\@@_tex_primitive:nnn { clubpenalty           } { assign_int } { 5 }
\@@_tex_primitive:nnn { widowpenalty          } { assign_int } { 6 }
\@@_tex_primitive:nnn { displaywidowpenalty   } { assign_int } { 7 }
\@@_tex_primitive:nnn { brokenpenalty         } { assign_int } { 8 }
\@@_tex_primitive:nnn { binoppenalty          } { assign_int } { 9 }
\@@_tex_primitive:nnn { relpenalty            } { assign_int } { 10 }
\@@_tex_primitive:nnn { predisplaypenalty     } { assign_int } { 11 }
\@@_tex_primitive:nnn { postdisplaypenalty    } { assign_int } { 12 }
\@@_tex_primitive:nnn { interlinepenalty      } { assign_int } { 13 }
\@@_tex_primitive:nnn { doublehyphendemerits  } { assign_int } { 14 }
\@@_tex_primitive:nnn { finalhyphendemerits   } { assign_int } { 15 }
\@@_tex_primitive:nnn { adjdemerits           } { assign_int } { 16 }
\@@_tex_primitive:nnn { mag                   } { assign_int } { 17 }
\@@_tex_primitive:nnn { delimiterfactor       } { assign_int } { 18 }
\@@_tex_primitive:nnn { looseness             } { assign_int } { 19 }
\@@_tex_primitive:nnn { time                  } { assign_int } { 20 }
\@@_tex_primitive:nnn { day                   } { assign_int } { 21 }
\@@_tex_primitive:nnn { month                 } { assign_int } { 22 }
\@@_tex_primitive:nnn { year                  } { assign_int } { 23 }
\@@_tex_primitive:nnn { showboxbreadth        } { assign_int } { 24 }
\@@_tex_primitive:nnn { showboxdepth          } { assign_int } { 25 }
\@@_tex_primitive:nnn { hbadness              } { assign_int } { 26 }
\@@_tex_primitive:nnn { vbadness              } { assign_int } { 27 }
\@@_tex_primitive:nnn { pausing               } { assign_int } { 28 }
\@@_tex_primitive:nnn { tracingonline         } { assign_int } { 29 }
\@@_tex_primitive:nnn { tracingmacros         } { assign_int } { 30 }
\@@_tex_primitive:nnn { tracingstats          } { assign_int } { 31 }
\@@_tex_primitive:nnn { tracingparagraphs     } { assign_int } { 32 }
\@@_tex_primitive:nnn { tracingpages          } { assign_int } { 33 }
\@@_tex_primitive:nnn { tracingoutput         } { assign_int } { 34 }
\@@_tex_primitive:nnn { tracinglostchars      } { assign_int } { 35 }
\@@_tex_primitive:nnn { tracingcommands       } { assign_int } { 36 }
\@@_tex_primitive:nnn { tracingrestores       } { assign_int } { 37 }
\@@_tex_primitive:nnn { uchyph                } { assign_int } { 38 }
\@@_tex_primitive:nnn { outputpenalty         } { assign_int } { 39 }
\@@_tex_primitive:nnn { maxdeadcycles         } { assign_int } { 40 }
\@@_tex_primitive:nnn { hangafter             } { assign_int } { 41 }
\@@_tex_primitive:nnn { floatingpenalty       } { assign_int } { 42 }
\@@_tex_primitive:nnn { globaldefs            } { assign_int } { 43 }
\@@_tex_primitive:nnn { fam                   } { assign_int } { 44 }
\@@_tex_primitive:nnn { escapechar            } { assign_int } { 45 }
\@@_tex_primitive:nnn { defaulthyphenchar     } { assign_int } { 46 }
\@@_tex_primitive:nnn { defaultskewchar       } { assign_int } { 47 }
\@@_tex_primitive:nnn { endlinechar           } { assign_int } { 48 }
\@@_tex_primitive:nnn { newlinechar           } { assign_int } { 49 }
\@@_tex_primitive:nnn { language              } { assign_int } { 50 }
\@@_tex_primitive:nnn { lefthyphenmin         } { assign_int } { 51 }
\@@_tex_primitive:nnn { righthyphenmin        } { assign_int } { 52 }
\@@_tex_primitive:nnn { holdinginserts        } { assign_int } { 53 }
\@@_tex_primitive:nnn { errorcontextlines     } { assign_int } { 54 }
\@@_tex_primitive:nnn { pdfoutput             } { assign_int } { 55 }
\@@_tex_primitive:nnn { pdfcompresslevel      } { assign_int } { 56 }
\@@_tex_primitive:nnn { pdfdecimaldigits      } { assign_int } { 57 }
\@@_tex_primitive:nnn { pdfmovechars          } { assign_int } { 58 }
\@@_tex_primitive:nnn { pdfimageresolution    } { assign_int } { 59 }
\@@_tex_primitive:nnn { pdfpkresolution       } { assign_int } { 60 }
\@@_tex_primitive:nnn { pdfuniqueresname      } { assign_int } { 61 }
\@@_tex_primitive:nnn
  { pdfoptionalwaysusepdfpagebox    } { assign_int } { 62 }
\@@_tex_primitive:nnn
  { pdfoptionpdfinclusionerrorlevel } { assign_int } { 63 }
\@@_tex_primitive:nnn
  { pdfoptionpdfminorversion        } { assign_int } { 64 }
\@@_tex_primitive:nnn { pdfminorversion       } { assign_int } { 64 }
\@@_tex_primitive:nnn { pdfforcepagebox       } { assign_int } { 65 }
\@@_tex_primitive:nnn { pdfpagebox            } { assign_int } { 66 }
\@@_tex_primitive:nnn
  { pdfinclusionerrorlevel } { assign_int } { 67 }
\@@_tex_primitive:nnn { pdfgamma              } { assign_int } { 68 }
\@@_tex_primitive:nnn { pdfimagegamma         } { assign_int } { 69 }
\@@_tex_primitive:nnn { pdfimagehicolor       } { assign_int } { 70 }
\@@_tex_primitive:nnn { pdfimageapplygamma    } { assign_int } { 71 }
\@@_tex_primitive:nnn { pdfadjustspacing      } { assign_int } { 72 }
\@@_tex_primitive:nnn { pdfprotrudechars      } { assign_int } { 73 }
\@@_tex_primitive:nnn { pdftracingfonts       } { assign_int } { 74 }
\@@_tex_primitive:nnn { pdfobjcompresslevel   } { assign_int } { 75 }
\@@_tex_primitive:nnn
  { pdfadjustinterwordglue } { assign_int } { 76 }
\@@_tex_primitive:nnn { pdfprependkern        } { assign_int } { 77 }
\@@_tex_primitive:nnn { pdfappendkern         } { assign_int } { 78 }
\@@_tex_primitive:nnn { pdfgentounicode       } { assign_int } { 79 }
\@@_tex_primitive:nnn { pdfdraftmode          } { assign_int } { 80 }
\@@_tex_primitive:nnn { pdfinclusioncopyfonts } { assign_int } { 81 }
\@@_tex_primitive:nnn { tracingassigns        } { assign_int } { 82 }
\@@_tex_primitive:nnn { tracinggroups         } { assign_int } { 83 }
\@@_tex_primitive:nnn { tracingifs            } { assign_int } { 84 }
\@@_tex_primitive:nnn { tracingscantokens     } { assign_int } { 85 }
\@@_tex_primitive:nnn { tracingnesting        } { assign_int } { 86 }
\@@_tex_primitive:nnn { predisplaydirection   } { assign_int } { 87 }
\@@_tex_primitive:nnn { lastlinefit           } { assign_int } { 88 }
\@@_tex_primitive:nnn { savingvdiscards       } { assign_int } { 89 }
\@@_tex_primitive:nnn { savinghyphcodes       } { assign_int } { 90 }
\@@_tex_primitive:nnn { TeXXeTstate           } { assign_int } { 91 }
\@@_tex_primitive:nnn { parindent             } { assign_dimen } { 0 }
\@@_tex_primitive:nnn { mathsurround          } { assign_dimen } { 1 }
\@@_tex_primitive:nnn { lineskiplimit         } { assign_dimen } { 2 }
\@@_tex_primitive:nnn { hsize                 } { assign_dimen } { 3 }
\@@_tex_primitive:nnn { vsize                 } { assign_dimen } { 4 }
\@@_tex_primitive:nnn { maxdepth              } { assign_dimen } { 5 }
\@@_tex_primitive:nnn { splitmaxdepth         } { assign_dimen } { 6 }
\@@_tex_primitive:nnn { boxmaxdepth           } { assign_dimen } { 7 }
\@@_tex_primitive:nnn { hfuzz                 } { assign_dimen } { 8 }
\@@_tex_primitive:nnn { vfuzz                 } { assign_dimen } { 9 }
\@@_tex_primitive:nnn { delimitershortfall   } { assign_dimen } { 10 }
\@@_tex_primitive:nnn { nulldelimiterspace   } { assign_dimen } { 11 }
\@@_tex_primitive:nnn { scriptspace          } { assign_dimen } { 12 }
\@@_tex_primitive:nnn { predisplaysize       } { assign_dimen } { 13 }
\@@_tex_primitive:nnn { displaywidth         } { assign_dimen } { 14 }
\@@_tex_primitive:nnn { displayindent        } { assign_dimen } { 15 }
\@@_tex_primitive:nnn { overfullrule         } { assign_dimen } { 16 }
\@@_tex_primitive:nnn { hangindent           } { assign_dimen } { 17 }
\@@_tex_primitive:nnn { hoffset              } { assign_dimen } { 18 }
\@@_tex_primitive:nnn { voffset              } { assign_dimen } { 19 }
\@@_tex_primitive:nnn { emergencystretch     } { assign_dimen } { 20 }
\@@_tex_primitive:nnn { pdfhorigin           } { assign_dimen } { 21 }
\@@_tex_primitive:nnn { pdfvorigin           } { assign_dimen } { 22 }
\@@_tex_primitive:nnn { pdfpagewidth         } { assign_dimen } { 23 }
\@@_tex_primitive:nnn { pdfpageheight        } { assign_dimen } { 24 }
\@@_tex_primitive:nnn { pdflinkmargin        } { assign_dimen } { 25 }
\@@_tex_primitive:nnn { pdfdestmargin        } { assign_dimen } { 26 }
\@@_tex_primitive:nnn { pdfthreadmargin      } { assign_dimen } { 27 }
\@@_tex_primitive:nnn { pdffirstlineheight   } { assign_dimen } { 28 }
\@@_tex_primitive:nnn { pdflastlinedepth     } { assign_dimen } { 29 }
\@@_tex_primitive:nnn { pdfeachlineheight    } { assign_dimen } { 30 }
\@@_tex_primitive:nnn { pdfeachlinedepth     } { assign_dimen } { 31 }
\@@_tex_primitive:nnn { pdfignoreddimen      } { assign_dimen } { 32 }
\@@_tex_primitive:nnn { pdfpxdimen           } { assign_dimen } { 33 }
\@@_tex_primitive:nnn { lineskip              } { assign_glue } { 0 }
\@@_tex_primitive:nnn { baselineskip          } { assign_glue } { 1 }
\@@_tex_primitive:nnn { parskip               } { assign_glue } { 2 }
\@@_tex_primitive:nnn { abovedisplayskip      } { assign_glue } { 3 }
\@@_tex_primitive:nnn { belowdisplayskip      } { assign_glue } { 4 }
\@@_tex_primitive:nnn { abovedisplayshortskip } { assign_glue } { 5 }
\@@_tex_primitive:nnn { belowdisplayshortskip } { assign_glue } { 6 }
\@@_tex_primitive:nnn { leftskip              } { assign_glue } { 7 }
\@@_tex_primitive:nnn { rightskip             } { assign_glue } { 8 }
\@@_tex_primitive:nnn { topskip               } { assign_glue } { 9 }
\@@_tex_primitive:nnn { splittopskip          } { assign_glue } { 10 }
\@@_tex_primitive:nnn { tabskip               } { assign_glue } { 11 }
\@@_tex_primitive:nnn { spaceskip             } { assign_glue } { 12 }
\@@_tex_primitive:nnn { xspaceskip            } { assign_glue } { 13 }
\@@_tex_primitive:nnn { parfillskip           } { assign_glue } { 14 }
\@@_tex_primitive:nnn { thinmuskip       } { assign_mu_glue } { 15 }
\@@_tex_primitive:nnn { medmuskip        } { assign_mu_glue } { 16 }
\@@_tex_primitive:nnn { thickmuskip      } { assign_mu_glue } { 17 }
\@@_tex_primitive:nnn { fontdimen        } { assign_font_dimen } { 0 }
\@@_tex_primitive:nnn { hyphenchar       } { assign_font_int } { 0 }
\@@_tex_primitive:nnn { skewchar         } { assign_font_int } { 1 }
\@@_tex_primitive:nnn { lpcode           } { assign_font_int } { 2 }
\@@_tex_primitive:nnn { rpcode           } { assign_font_int } { 3 }
\@@_tex_primitive:nnn { efcode           } { assign_font_int } { 4 }
\@@_tex_primitive:nnn { tagcode          } { assign_font_int } { 5 }
\@@_tex_primitive:nnn { pdfnoligatures   } { assign_font_int } { 6 }
\@@_tex_primitive:nnn { knbscode         } { assign_font_int } { 7 }
\@@_tex_primitive:nnn { stbscode         } { assign_font_int } { 8 }
\@@_tex_primitive:nnn { shbscode         } { assign_font_int } { 9 }
\@@_tex_primitive:nnn { knbccode         } { assign_font_int } { 10 }
\@@_tex_primitive:nnn { knaccode         } { assign_font_int } { 11 }
\@@_tex_primitive:nnn { spacefactor      } { set_aux } { 102 }
\@@_tex_primitive:nnn { prevdepth        } { set_aux } { 1 }
\@@_tex_primitive:nnn { prevgraf         } { set_prev_graf } { 0 }
\@@_tex_primitive:nnn { pagegoal         } { set_page_dimen } { 0 }
\@@_tex_primitive:nnn { pagetotal        } { set_page_dimen } { 1 }
\@@_tex_primitive:nnn { pagestretch      } { set_page_dimen } { 2 }
\@@_tex_primitive:nnn { pagefilstretch   } { set_page_dimen } { 3 }
\@@_tex_primitive:nnn { pagefillstretch  } { set_page_dimen } { 4 }
\@@_tex_primitive:nnn { pagefilllstretch } { set_page_dimen } { 5 }
\@@_tex_primitive:nnn { pageshrink       } { set_page_dimen } { 6 }
\@@_tex_primitive:nnn { pagedepth        } { set_page_dimen } { 7 }
\@@_tex_primitive:nnn { deadcycles       } { set_page_int } { 0 }
\@@_tex_primitive:nnn { insertpenalties  } { set_page_int } { 1 }
\@@_tex_primitive:nnn { interactionmode  } { set_page_int } { 2 }
\@@_tex_primitive:nnn { wd               } { set_box_dimen } { 1 }
\@@_tex_primitive:nnn { dp               } { set_box_dimen } { 2 }
\@@_tex_primitive:nnn { ht               } { set_box_dimen } { 3 }
\@@_tex_primitive:nnn { parshape              } { set_shape } { 0 }
\@@_tex_primitive:nnn { interlinepenalties    } { set_shape } { 1 }
\@@_tex_primitive:nnn { clubpenalties         } { set_shape } { 2 }
\@@_tex_primitive:nnn { widowpenalties        } { set_shape } { 3 }
\@@_tex_primitive:nnn { displaywidowpenalties } { set_shape } { 4 }
\@@_tex_primitive:nnn { catcode               } { def_code } { 0 }
\@@_tex_primitive:nnn { lccode                } { def_code } { 256 }
\@@_tex_primitive:nnn { uccode                } { def_code } { 512 }
\@@_tex_primitive:nnn { sfcode                } { def_code } { 768 }
\@@_tex_primitive:nnn { mathcode              } { def_code } { 1024 }
\@@_tex_primitive:nnn { delcode               } { def_code } { 1591 }
\@@_tex_primitive:nnn { textfont              } { def_family } { -48 }
\@@_tex_primitive:nnn { scriptfont            } { def_family } { -32 }
\@@_tex_primitive:nnn { scriptscriptfont      } { def_family } { -16 }
\@@_tex_primitive:nnn { nullfont              } { set_font } { 0 }
\@@_tex_primitive:nnn { font                  } { def_font } { 0 }
\@@_tex_primitive:nnn { count           } { register } { 1 000 000 }
\@@_tex_primitive:nnn { dimen           } { register } { 2 000 000 }
\@@_tex_primitive:nnn { skip            } { register } { 3 000 000 }
\@@_tex_primitive:nnn { muskip          } { register } { 4 000 000 }
\@@_tex_primitive:nnn { advance         } { advance } { 0 }
\@@_tex_primitive:nnn { multiply        } { multiply } { 0 }
\@@_tex_primitive:nnn { divide          } { divide } { 0 }
\@@_tex_primitive:nnn { long            } { prefix } { 1 }
\@@_tex_primitive:nnn { outer           } { prefix } { 2 }
\@@_tex_primitive:nnn { global          } { prefix } { 4 }
\@@_tex_primitive:nnn { protected       } { prefix } { 8 }
\@@_tex_primitive:nnn { let             } { let } { 0 }
\@@_tex_primitive:nnn { futurelet       } { let } { 1 }
\@@_tex_primitive:nnn { chardef         } { shorthand_def } { 0 }
\@@_tex_primitive:nnn { mathchardef     } { shorthand_def } { 1 }
\@@_tex_primitive:nnn { countdef        } { shorthand_def } { 2 }
\@@_tex_primitive:nnn { dimendef        } { shorthand_def } { 3 }
\@@_tex_primitive:nnn { skipdef         } { shorthand_def } { 4 }
\@@_tex_primitive:nnn { muskipdef       } { shorthand_def } { 5 }
\@@_tex_primitive:nnn { toksdef         } { shorthand_def } { 6 }
\@@_tex_primitive:nnn { read            } { read_to_cs } { 0 }
\@@_tex_primitive:nnn { readline        } { read_to_cs } { 1 }
\@@_tex_primitive:nnn { def             } { def } { 0 }
\@@_tex_primitive:nnn { gdef            } { def } { 1 }
\@@_tex_primitive:nnn { edef            } { def } { 2 }
\@@_tex_primitive:nnn { xdef            } { def } { 3 }
\@@_tex_primitive:nnn { setbox          } { set_box } { 0 }
\@@_tex_primitive:nnn { hyphenation     } { hyph_data } { 0 }
\@@_tex_primitive:nnn { patterns        } { hyph_data } { 1 }
\@@_tex_primitive:nnn { batchmode       } { set_interaction } { 0 }
\@@_tex_primitive:nnn { nonstopmode     } { set_interaction } { 1 }
\@@_tex_primitive:nnn { scrollmode      } { set_interaction } { 2 }
\@@_tex_primitive:nnn { errorstopmode   } { set_interaction } { 3 }
\@@_tex_primitive:nnn { letterspacefont } { letterspace_font } { 0 }
\@@_tex_primitive:nnn { pdfcopyfont     } { pdf_copy_font } { 0 }
\@@_tex_primitive:nnn { undefined         } { undefined_cs } { 0 }
\@@_tex_primitive:nnn { ndefined          } { undefined_cs } { 0 }
\@@_tex_primitive:nnn { expandafter       } { expand_after } { 0 }
\@@_tex_primitive:nnn { unless            } { expand_after } { 1 }
\@@_tex_primitive:nnn { pdfprimitive      } { no_expand } { 1 }
\@@_tex_primitive:nnn { noexpand          } { no_expand } { 0 }
\@@_tex_primitive:nnn { input             } { input } { 0 }
\@@_tex_primitive:nnn { endinput          } { input } { 1 }
\@@_tex_primitive:nnn { scantokens        } { input } { 2 }
\@@_tex_primitive:nnn { if                } { if_test } { 0 }
\@@_tex_primitive:nnn { ifcat             } { if_test } { 1 }
\@@_tex_primitive:nnn { ifnum             } { if_test } { 2 }
\@@_tex_primitive:nnn { ifdim             } { if_test } { 3 }
\@@_tex_primitive:nnn { ifodd             } { if_test } { 4 }
\@@_tex_primitive:nnn { ifvmode           } { if_test } { 5 }
\@@_tex_primitive:nnn { ifhmode           } { if_test } { 6 }
\@@_tex_primitive:nnn { ifmmode           } { if_test } { 7 }
\@@_tex_primitive:nnn { ifinner           } { if_test } { 8 }
\@@_tex_primitive:nnn { ifvoid            } { if_test } { 9 }
\@@_tex_primitive:nnn { ifhbox            } { if_test } { 10 }
\@@_tex_primitive:nnn { ifvbox            } { if_test } { 11 }
\@@_tex_primitive:nnn { ifx               } { if_test } { 12 }
\@@_tex_primitive:nnn { ifeof             } { if_test } { 13 }
\@@_tex_primitive:nnn { iftrue            } { if_test } { 14 }
\@@_tex_primitive:nnn { iffalse           } { if_test } { 15 }
\@@_tex_primitive:nnn { ifcase            } { if_test } { 16 }
\@@_tex_primitive:nnn { ifdefined         } { if_test } { 17 }
\@@_tex_primitive:nnn { ifcsname          } { if_test } { 18 }
\@@_tex_primitive:nnn { iffontchar        } { if_test } { 19 }
\@@_tex_primitive:nnn { ifincsname        } { if_test } { 20 }
\@@_tex_primitive:nnn { ifpdfprimitive    } { if_test } { 21 }
\@@_tex_primitive:nnn { ifpdfabsnum       } { if_test } { 22 }
\@@_tex_primitive:nnn { ifpdfabsdim       } { if_test } { 23 }
\@@_tex_primitive:nnn { fi                } { fi_or_else } { 2 }
\@@_tex_primitive:nnn { else              } { fi_or_else } { 3 }
\@@_tex_primitive:nnn { or                } { fi_or_else } { 4 }
\@@_tex_primitive:nnn { csname            } { cs_name } { 0 }
\@@_tex_primitive:nnn { number            } { convert } { 0 }
\@@_tex_primitive:nnn { romannumeral      } { convert } { 1 }
\@@_tex_primitive:nnn { string            } { convert } { 2 }
\@@_tex_primitive:nnn { meaning           } { convert } { 3 }
\@@_tex_primitive:nnn { fontname          } { convert } { 4 }
\@@_tex_primitive:nnn { eTeXrevision      } { convert } { 5 }
\@@_tex_primitive:nnn { pdftexrevision    } { convert } { 6 }
\@@_tex_primitive:nnn { pdftexbanner      } { convert } { 7 }
\@@_tex_primitive:nnn { pdffontname       } { convert } { 8 }
\@@_tex_primitive:nnn { pdffontobjnum     } { convert } { 9 }
\@@_tex_primitive:nnn { pdffontsize       } { convert } { 10 }
\@@_tex_primitive:nnn { pdfpageref        } { convert } { 11 }
\@@_tex_primitive:nnn { pdfxformname      } { convert } { 12 }
\@@_tex_primitive:nnn { pdfescapestring   } { convert } { 13 }
\@@_tex_primitive:nnn { pdfescapename     } { convert } { 14 }
\@@_tex_primitive:nnn { leftmarginkern    } { convert } { 15 }
\@@_tex_primitive:nnn { rightmarginkern   } { convert } { 16 }
\@@_tex_primitive:nnn
  { \sys_if_engine_xetex:F { pdf } strcmp } { convert } { 17 }
\@@_tex_primitive:nnn { pdfcolorstackinit } { convert } { 18 }
\@@_tex_primitive:nnn { pdfescapehex      } { convert } { 19 }
\@@_tex_primitive:nnn { pdfunescapehex    } { convert } { 20 }
\@@_tex_primitive:nnn { pdfcreationdate   } { convert } { 21 }
\@@_tex_primitive:nnn { pdffilemoddate    } { convert } { 22 }
\@@_tex_primitive:nnn { pdffilesize       } { convert } { 23 }
\@@_tex_primitive:nnn { pdfmdfivesum      } { convert } { 24 }
\@@_tex_primitive:nnn { pdffiledump       } { convert } { 25 }
\@@_tex_primitive:nnn { pdfmatch          } { convert } { 26 }
\@@_tex_primitive:nnn { pdflastmatch      } { convert } { 27 }
\@@_tex_primitive:nnn { pdfuniformdeviate } { convert } { 28 }
\@@_tex_primitive:nnn { pdfnormaldeviate  } { convert } { 29 }
\@@_tex_primitive:nnn { pdfinsertht       } { convert } { 30 }
\@@_tex_primitive:nnn { pdfximagebbox     } { convert } { 31 }
\@@_tex_primitive:nnn { jobname           } { convert } { 32 }
\sys_if_engine_luatex:T
  {
    \@@_tex_primitive:nnn { directlua       } { convert } { 33 }
    \@@_tex_primitive:nnn { expanded        } { convert } { 34 }
    \@@_tex_primitive:nnn { luaescapestring } { convert } { 35 }
  }
\sys_if_engine_xetex:T
  {
    \@@_tex_primitive:nnn { Ucharcat        } { convert } { 40 }
  }
\@@_tex_primitive:nnn { the               } { the } { 0 }
\@@_tex_primitive:nnn { unexpanded        } { the } { 1 }
\@@_tex_primitive:nnn { detokenize        } { the } { 5 }
\@@_tex_primitive:nnn { topmark           } { top_bot_mark } { 0 }
\@@_tex_primitive:nnn { firstmark         } { top_bot_mark } { 1 }
\@@_tex_primitive:nnn { botmark           } { top_bot_mark } { 2 }
\@@_tex_primitive:nnn { splitfirstmark    } { top_bot_mark } { 3 }
\@@_tex_primitive:nnn { splitbotmark      } { top_bot_mark } { 4 }
\@@_tex_primitive:nnn { topmarks          } { top_bot_mark } { 5 }
\@@_tex_primitive:nnn { firstmarks        } { top_bot_mark } { 6 }
\@@_tex_primitive:nnn { botmarks          } { top_bot_mark } { 7 }
\@@_tex_primitive:nnn { splitfirstmarks   } { top_bot_mark } { 8 }
\@@_tex_primitive:nnn { splitbotmarks     } { top_bot_mark } { 9 }
%    \end{macrocode}
%
% \subsection{Get next token}
%
% We define here two functions which fetch the next token in the token
% list.
% \begin{itemize}
% \item \cs{@@_get_next:} sets \cs{l_@@_head_gtl}, \cs{l_@@_head_token},
%   and if possible \cs{l_@@_head_tl} (otherwise it is cleared).
% \item \cs{@@_get_token:} additionally sets \cs{l_@@_head_cmd_int} and
%   \cs{l_@@_head_char_int}.
% \end{itemize}
% The latter is based on \cs{@@_set_cmd:} which derives the
% \cs{l_@@_head_cmd_int} and \cs{l_@@_head_char_int} from
% \cs{l_@@_head_token}.
%
% \begin{macro}{\@@_get_next:}
% \begin{macro}{\@@_get_next_aux:w}
%   If the input is empty, forcefully exit.  Otherwise,
%   remove the first token in the input, and store it in
%   \cs{l_@@_head_gtl}.  Set \cs{l_@@_head_token} equal in meaning to
%   that first token.  Then set \cs{l_@@_head_tl} to contain the token,
%   unless it is a begin-group or end-group character, in which case
%   this token list is emptied.
%    \begin{macrocode}
\cs_new_protected:Npn \@@_get_next:
  {
    \@@_input_if_empty:TF
      { \@@_exit:w }
      {
        \@@_input_gpop:N \l_@@_head_gtl
        \gtl_head_do:NN \l_@@_head_gtl \@@_get_next_aux:w
        \gtl_if_tl:NTF \l_@@_head_gtl
          {
            \tl_set:Nx \l_@@_head_tl
              { \gtl_head:N \l_@@_head_gtl }
            \token_if_eq_meaning:NNT
              \l_@@_head_token \@@_special_relax:
              \@@_get_next_notexpanded:
          }
          { \tl_clear:N \l_@@_head_tl }
      }
  }
\cs_new_protected:Npn \@@_get_next_aux:w
  { \cs_set_eq:NN \l_@@_head_token }
%    \end{macrocode}
% \end{macro}
% \end{macro}
%
% \begin{macro}{\@@_get_next_notexpanded:}
% \begin{macro}{\@@_notexpanded_test:w, \@@_notexpanded_expand:nN, \@@_notexpanded_expand:NN}
%   At this point we have likely encountered a special \tn{relax} marker
%   that we use to mark cases where \tn{noexpand} acts on a control
%   sequence or an active character.  To make sure of that check the
%   control sequence has the form \cs{notexpanded:\ldots{}}.  Since we
%   don't know the escape character we must use \cs{cs_to_str:N}, but
%   that function is not meant for active characters and has a runaway
%   argument if its argument is a space (active since we know its
%   meaning is the special \tn{relax}).  To avoid the runaway we include
%   an arbitrary delimiter~|Z|.  If the token in \cs{l_@@_head_tl} is
%   not \cs{notexpanded:\ldots{}} we do nothing.  Otherwise
%   \cs{@@_notexpanded_expand:n} reconstructs the token that was hit
%   with \tn{noexpand} (an active character if the argument is a single
%   character) and do the job of \cs{@@_get_next:}, setting
%   \cs{l_@@_head_token} to the special \tn{relax} marker for expandable
%   commands, as \tn{noexpand} would.
%    \begin{macrocode}
\cs_set_protected:Npn \@@_tmp:w #1
  {
    \cs_new_protected:Npn \@@_get_next_notexpanded:
      {
        \tl_if_eq:onTF { \l_@@_head_tl } { \@@_unravel_marker: }
          { \@@_get_next_marker: }
          {
            \@@_exp_args:NNx \use:nn \@@_notexpanded_test:w
              { \scan_stop: \exp_after:wN \cs_to_str:N \l_@@_head_tl Z }
              \q_mark \@@_notexpanded_expand:n
              #1 Z \q_mark \use_none:n
              \q_stop
          }
      }
    \cs_new_protected:Npn \@@_notexpanded_test:w
        ##1 #1 ##2 Z \q_mark ##3##4 \q_stop
      { ##3 {##2} }
  }
\exp_args:Nx \@@_tmp:w { \scan_stop: \tl_to_str:n { notexpanded: } }
\group_begin:
  \char_set_catcode_active:n { 0 }
  \cs_new_protected:Npn \@@_notexpanded_expand:n #1
    {
      \@@_exp_args:Nx \tl_if_empty:nTF { \str_tail:n {#1} }
        {
          \group_begin:
          \char_set_lccode:nn { 0 } { `#1 }
          \tex_lowercase:D
            {
              \group_end:
              \@@_notexpanded_expand:N ^^@
            }
        }
        {
          \group_begin: \exp_args:NNc \group_end:
          \@@_notexpanded_expand:N { \use_none:n #1 }
        }
    }
\group_end:
\cs_new_protected:Npn \@@_notexpanded_expand:N #1
  {
    \gtl_set:Nn \l_@@_head_gtl {#1}
    \tl_set:Nn \l_@@_head_tl {#1}
    \cs_set_eq:NN \l_@@_head_token \@@_special_relax:
  }
%    \end{macrocode}
% \end{macro}
% \end{macro}
%
% \begin{macro}{\@@_get_next_marker:}
%   This is used to deal with nested unravel.
%    \begin{macrocode}
\cs_new_protected:Npn \@@_get_next_marker:
  {
    \@@_get_next:
    \tl_if_eq:onTF \l_@@_head_tl { \@@:nn }
      { \@@_error:nxxxx { nested-unravel } { } { } { } { } }
      { \@@_error:nxxxx { internal } { marker~unknown } { } { } { } }
    \@@_input_gpop_item:NF \l_@@_argi_tl
      { \@@_error:nxxxx { internal } { marker~1 } { } { } { } }
    \@@_input_gpop_item:NF \l_@@_argii_tl
      { \@@_error:nxxxx { internal } { marker~2 } { } { } { } }
    \exp_args:Nno \keys_set:nn { unravel } \l_@@_argi_tl
    \@@_exp_args:Nx \@@_back_input:n
      { \exp_not:N \exp_not:n { \exp_not:o \l_@@_argii_tl } }
    \@@_get_next:
  }
%    \end{macrocode}
% \end{macro}
%
% \begin{macro}{\@@_get_token:}
%   Call \cs{@@_get_next:} to set \cs{l_@@_head_gtl}, \cs{l_@@_head_tl}
%   and \cs{l_@@_head_token}, then call \cs{@@_set_cmd:} to set
%   \cs{l_@@_head_cmd_int} and \cs{l_@@_head_char_int}.
%    \begin{macrocode}
\cs_new_protected:Npn \@@_get_token:
  {
    \@@_get_next:
    \@@_set_cmd:
  }
%    \end{macrocode}
% \end{macro}
%
% \begin{macro}{\@@_set_cmd:}
%   After the call to \cs{@@_get_next:}, we find the command code
%   \cs{l_@@_head_cmd_int} and the character code
%   \cs{l_@@_head_char_int}, based only on \cs{l_@@_head_token}.  First
%   set \cs{l_@@_head_meaning_tl} from the \tn{meaning} of the first
%   token.  If the corresponding primitive exists, use the information
%   to set the two integers.  If the token is expandable, it can either
%   be a macro or be a primitive that we somehow do not know
%   (\emph{e.g.}, an expandable \XeTeX{} or \LuaTeX{} primitive
%   perhaps).  Otherwise, it can be a control sequence or a character.
%    \begin{macrocode}
\cs_new_protected:Npn \@@_set_cmd:
  {
    \@@_set_cmd_aux_meaning:
    \@@_set_cmd_aux_primitive:oTF { \l_@@_head_meaning_tl }
      { }
      {
        \@@_token_if_expandable:NTF \l_@@_head_token
          {
            \token_if_macro:NTF \l_@@_head_token
              { \@@_set_cmd_aux_macro: }
              { \@@_set_cmd_aux_unknown: }
          }
          {
            \token_if_cs:NTF \l_@@_head_token
              { \@@_set_cmd_aux_cs: }
              { \@@_set_cmd_aux_char: }
          }
      }
  }
%    \end{macrocode}
% \end{macro}
%
% \begin{macro}{\@@_set_cmd_aux_meaning:}
% \begin{macro}[EXP]{\@@_set_cmd_aux_meaning:w}
%   Remove the leading escape character (\cs{@@_strip_escape:w} takes
%   care of special cases there) from the \tn{meaning} of the first
%   token, then remove anything after the first~|:|, which is present
%   for macros, for marks, and for that character too.  For any
%   primitive except \tn{nullfont}, this leaves the primitive's name.
%    \begin{macrocode}
\cs_new_protected:Npn \@@_set_cmd_aux_meaning:
  {
    \tl_set:Nx \l_@@_head_meaning_tl
      {
        \exp_after:wN \@@_strip_escape:w
        \token_to_meaning:N \l_@@_head_token
        \tl_to_str:n { : }
      }
    \tl_set:Nx \l_@@_head_meaning_tl
      {
        \exp_after:wN \@@_set_cmd_aux_meaning:w
          \l_@@_head_meaning_tl \q_stop
      }
  }
\use:x
  {
    \cs_new:Npn \exp_not:N \@@_set_cmd_aux_meaning:w
      ##1 \token_to_str:N : ##2 \exp_not:N \q_stop {##1}
  }
%    \end{macrocode}
% \end{macro}
% \end{macro}
%
% \begin{macro}
%   {\@@_set_cmd_aux_primitive:nTF, \@@_set_cmd_aux_primitive:oTF}
% \begin{macro}{\@@_set_cmd_aux_primitive:nn}
%   Test if there is any information about the given (cleaned-up)
%   \tn{meaning}.  If there is, use that as the command and character
%   integers.
%    \begin{macrocode}
\cs_new_protected:Npn \@@_set_cmd_aux_primitive:nTF #1#2
  {
    \cs_if_exist:cTF { c_@@_tex_#1_tl }
      {
        \exp_last_unbraced:Nv \@@_set_cmd_aux_primitive:nn
          { c_@@_tex_#1_tl }
        #2
      }
  }
\cs_generate_variant:Nn \@@_set_cmd_aux_primitive:nTF { o }
\cs_new_protected:Npn \@@_set_cmd_aux_primitive:nn #1#2
  {
    \int_set:Nn \l_@@_head_cmd_int {#1}
    \int_set:Nn \l_@@_head_char_int {#2}
  }
%    \end{macrocode}
% \end{macro}
% \end{macro}
%
% \begin{macro}{\@@_set_cmd_aux_macro:}
%   The token is a macro.  There is no need to determine whether the
%   macro is long/outer.
%    \begin{macrocode}
\cs_new_protected:Npn \@@_set_cmd_aux_macro:
  {
    \int_set:Nn \l_@@_head_cmd_int { \@@_tex_use:n { call } }
    \int_zero:N \l_@@_head_char_int
  }
%    \end{macrocode}
% \end{macro}
%
% \begin{macro}{\@@_set_cmd_aux_unknown:}
%   Complain about an unknown primitive, and consider it as if it were
%   \tn{relax}.
%    \begin{macrocode}
\cs_new_protected:Npn \@@_set_cmd_aux_unknown:
  {
    \exp_last_unbraced:NV \@@_set_cmd_aux_primitive:nn
      \c_@@_tex_relax_tl
    \@@_error:nxxxx { unknown-primitive }
      { \l_@@_head_meaning_tl } { } { } { }
  }
%    \end{macrocode}
% \end{macro}
%
% \begin{macro}{\@@_set_cmd_aux_cs:}
%   If the \tn{meaning} contains \verb*|elect font|, the control
%   sequence is \tn{nullfont} or similar (note that we do not search for
%   \verb*|select font|, as the code to trim the escape character from
%   the meaning may have removed the leading~|s|).  Otherwise, we expect
%   the \tn{meaning} to be \tn{char} or \tn{mathchar} followed by
%   |"|~and an uppercase hexadecimal number, or one of \tn{count},
%   \tn{dimen}, \tn{skip}, \tn{muskip} or \tn{toks} followed by a
%   decimal number.
%    \begin{macrocode}
\cs_new_protected:Npn \@@_set_cmd_aux_cs:
  {
    \@@_tl_if_in:ooTF \l_@@_head_meaning_tl
      { \tl_to_str:n { elect~font } }
      {
        \exp_last_unbraced:NV \@@_set_cmd_aux_primitive:nn
          \c_@@_tex_nullfont_tl
      }
      { \@@_set_cmd_aux_numeric: }
  }
%    \end{macrocode}
% \end{macro}
%
% \begin{macro}
%   {
%     \@@_set_cmd_aux_numeric:,
%     \@@_set_cmd_aux_numeric:w,
%     \@@_set_cmd_aux_given:n
%   }
% \begin{macro}[rEXP]{\@@_set_cmd_aux_numeric:N}
%   Insert \cs{q_mark} before the first non-letter (in fact, anything
%   less than~|A|) in the \tn{meaning} by looping one character at a
%   time (skipping spaces, but there should be none).  We expect the
%   first part to be |char| or~|mathchar| (or |kchar| in up\TeX{}),
%   or one of |count|, |dimen|,
%   |skip|, |muskip|, or |toks|.  In the first two (three) cases, the command is
%   |char_given| or |math_given|.  It is otherwise identical to the
%   corresponding primitive (\tn{count} \emph{etc.}).  We then keep
%   track of the associated number (part after \cs{q_mark}) in
%   \cs{l_@@_head_char_int}.  For unknown non-expandable primitives,
%   assuming that their meaning consists solely of letters, the
%   \cs{q_mark} is inserted at their end, and is followed by~|+0|, so
%   nothing breaks.
%    \begin{macrocode}
\cs_new_protected:Npn \@@_set_cmd_aux_numeric:
  {
    \tl_set:Nx \l_@@_tmpa_tl
      {
        \exp_after:wN \@@_set_cmd_aux_numeric:N
          \l_@@_head_meaning_tl + 0
      }
    \exp_after:wN \@@_set_cmd_aux_numeric:w
      \l_@@_tmpa_tl \q_stop
  }
\cs_new:Npn \@@_set_cmd_aux_numeric:N #1
  {
    \if_int_compare:w `#1 < `A \exp_stop_f:
      \exp_not:N \q_mark
      \exp_after:wN \use_i:nn
    \fi:
    #1 \@@_set_cmd_aux_numeric:N
  }
\cs_new_protected:Npn \@@_set_cmd_aux_numeric:w #1 \q_mark #2 \q_stop
  {
    \str_case:nnF {#1}
      {
        { char }     { \@@_set_cmd_aux_given:n { char_given } }
        { kchar }    { \@@_set_cmd_aux_given:n { char_given } }
        { mathchar } { \@@_set_cmd_aux_given:n { math_given } }
      }
      {
        \@@_set_cmd_aux_primitive:nTF {#1}
          { }
          { \@@_set_cmd_aux_unknown: }
        \int_add:Nn \l_@@_head_char_int { 100 000 }
      }
    \int_add:Nn \l_@@_head_char_int {#2}
  }
\cs_new_protected:Npn \@@_set_cmd_aux_given:n #1
  {
    \int_set:Nn \l_@@_head_cmd_int { \@@_tex_use:n {#1} }
    \int_zero:N \l_@@_head_char_int
  }
%    \end{macrocode}
% \end{macro}
% \end{macro}
%
% \begin{macro}{\@@_set_cmd_aux_char:}
% \begin{macro}{\@@_set_cmd_aux_char:w}
%   At this point, the \tn{meaning} token list has been shortened by the
%   code meant to remove the escape character.  We thus set it again to
%   the \tn{meaning} of the leading token.  The command is then the
%   first word (delimited by a space) of the \tn{meaning}, followed by
%   |_char|, except for category other, where we use |other_char|.  For
%   the character code, there is a need to expand
%   \cs{@@_token_to_char:N} before placing~|`|.
%    \begin{macrocode}
\cs_new_protected:Npn \@@_set_cmd_aux_char:
  {
    \tl_set:Nx \l_@@_head_meaning_tl
      { \token_to_meaning:N \l_@@_head_token }
    \token_if_eq_catcode:NNT \l_@@_head_token \c_catcode_other_token
      { \tl_set:Nn \l_@@_head_meaning_tl { other~ } }
    \exp_after:wN \@@_set_cmd_aux_char:w
      \l_@@_head_meaning_tl \q_stop
    \@@_exp_args:NNx \int_set:Nn \l_@@_head_char_int
      { ` \@@_token_to_char:N \l_@@_head_token }
  }
\cs_new_protected:Npn \@@_set_cmd_aux_char:w #1 ~ #2 \q_stop
  {
    \int_set:Nn \l_@@_head_cmd_int
      { \@@_tex_use:n { #1_char } }
  }
%    \end{macrocode}
% \end{macro}
% \end{macro}
%
% \subsection{Manipulating the input}
%
% \subsubsection{Elementary operations}
%
% \begin{macro}[rEXP]{\@@_input_to_str:}
%   Map \cs{gtl_to_str:c} through the input stack.
%    \begin{macrocode}
\cs_new:Npn \@@_input_to_str:
  {
    \int_step_function:nnnN \g_@@_input_int { -1 } { 1 }
      \@@_input_to_str_aux:n
  }
\cs_new:Npn \@@_input_to_str_aux:n #1
  { \gtl_to_str:c { g_@@_input_#1_gtl } }
%    \end{macrocode}
% \end{macro}
%
% \begin{macro}{\@@_input_if_empty:TF}
%   If the input stack is empty, the input contains no token.
%   Otherwise, check the top of the stack for tokens: if there are, then
%   the input is non-empty, and if there are none, then we get rid of
%   the top of stack and loop.
%    \begin{macrocode}
\cs_new_protected:Npn \@@_input_if_empty:TF
  {
    \int_compare:nNnTF \g_@@_input_int = \c_zero
      { \use_i:nn }
      {
        \gtl_if_empty:cTF
          { g_@@_input_ \int_use:N \g_@@_input_int _gtl }
          {
            \int_gdecr:N \g_@@_input_int
            \@@_input_if_empty:TF
          }
          {
            \@@_input_split:
            \use_ii:nn
          }
      }
  }
%    \end{macrocode}
% \end{macro}
%
% ^^A todo: profiling to know if \@@_input_split: is beneficial.
%
% \begin{macro}{\@@_input_split:}
%   If the input is completely flat, and is a token list starting with
%   an |N|-type token, try to unflatten it by splitting at each
%   occurence of that first character
%    \begin{macrocode}
\cs_new_protected:Npn \@@_input_split:
  {
    \int_compare:nNnT \g_@@_input_int = \c_one
      {
        \exp_args:Nc \@@_input_split_aux:N
          { g_@@_input_1_gtl }
      }
  }
\cs_new_protected:Npn \@@_input_split_aux:N #1
  {
    \gtl_if_tl:NT #1
      {
        \gtl_if_head_is_N_type:NT #1
          {
            \tl_set:Nx \l_@@_input_tmpa_tl { \gtl_left_tl:N #1 }
            \@@_exp_args:NNx \use:nn
                \@@_input_split_auxii:N
                { \tl_head:N \l_@@_input_tmpa_tl }
          }
      }
  }
\cs_new_protected:Npn \@@_input_split_auxii:N #1
  {
    \token_if_parameter:NF #1
      {
        \tl_replace_all:Nnn \l_@@_input_tmpa_tl {#1}
          { \@@_input_split_end: \@@_input_split_auxiii:w #1 }
        \group_begin:
          \cs_set:Npn \@@_input_split_auxiii:w
            ##1 \@@_input_split_end: { + 1 }
          \int_gset:Nn \g_@@_input_int
            { 0 \l_@@_input_tmpa_tl \@@_input_split_end: }
        \group_end:
        \int_gset_eq:NN \g_@@_input_tmpa_int \g_@@_input_int
        \l_@@_input_tmpa_tl \@@_input_split_end:
      }
  }
\cs_new:Npn \@@_input_split_end: { }
\cs_new_protected:Npn \@@_input_split_auxiii:w
    #1 \@@_input_split_end:
  {
    \gtl_gclear_new:c
      { g_@@_input_ \int_use:N \g_@@_input_tmpa_int _gtl }
    \gtl_gset:cn
      { g_@@_input_ \int_use:N \g_@@_input_tmpa_int _gtl } {#1}
    \int_gdecr:N \g_@@_input_tmpa_int
  }
%    \end{macrocode}
% \end{macro}
%
% \begin{macro}{\@@_input_gset:n}
%   At first, all of the input is in the same~\texttt{gtl}.
%    \begin{macrocode}
\cs_new_protected:Npn \@@_input_gset:n
  {
    \int_gzero:N \g_@@_input_int
    \@@_back_input:n
  }
%    \end{macrocode}
% \end{macro}
%
% \begin{macro}{\@@_input_get:N}
%    \begin{macrocode}
\cs_new_protected:Npn \@@_input_get:N #1
  {
    \@@_input_if_empty:TF
      { \gtl_set:Nn #1 { \q_no_value } }
      {
        \gtl_get_left:cN
          { g_@@_input_ \int_use:N \g_@@_input_int _gtl } #1
      }
  }
%    \end{macrocode}
% \end{macro}
%
% \begin{macro}{\@@_input_gpop:N}
%   Call \cs{@@_input_if_empty:TF} to remove empty levels from the input
%   stack, then extract the first token from the left-most non-empty
%   level.
%    \begin{macrocode}
\cs_new_protected:Npn \@@_input_gpop:N #1
  {
    \@@_input_if_empty:TF
      { \gtl_set:Nn #1 { \q_no_value } }
      {
        \gtl_gpop_left:cN
          { g_@@_input_ \int_use:N \g_@@_input_int _gtl } #1
      }
  }
%    \end{macrocode}
% \end{macro}
%
% \begin{macro}{\@@_input_merge:}
%   Merge the top two levels of input.  This requires, but does not
%   check, that \cs{g_@@_input_int} is at least~$2$.
%    \begin{macrocode}
\cs_new_protected:Npn \@@_input_merge:
  {
    \int_gdecr:N \g_@@_input_int
    \gtl_gconcat:ccc
      { g_@@_input_ \int_use:N \g_@@_input_int _gtl }
      { g_@@_input_ \int_eval:n { \g_@@_input_int + 1 } _gtl }
      { g_@@_input_ \int_use:N \g_@@_input_int _gtl }
    \gtl_gclear:c
      { g_@@_input_ \int_eval:n { \g_@@_input_int + 1 } _gtl }
  }
%    \end{macrocode}
% \end{macro}
%
% \begin{macro}[TF]{\@@_input_gpop_item:N}
% \begin{macro}{\@@_input_gpop_item_aux:NN}
%   If there is no input, we cannot pop an item.  Othewise, try to pop
%   from the top of the input stack.  If this succeeds, or if this
%   failed and the top of stack has extra end-group characters, or if
%   the input stack contains only the top-most item, then the answer
%   given by \cs{gtl_gpop_left_item:NNTF} is the correct one, which we
%   return.  Otherwise, merge the top two levels and repeat.
%    \begin{macrocode}
\prg_new_protected_conditional:Npnn \@@_input_gpop_item:N #1 { F }
  {
    \int_compare:nNnTF \g_@@_input_int = \c_zero
      { \prg_return_false: }
      {
        \exp_args:Nc \@@_input_gpop_item_aux:NN
          { g_@@_input_ \int_use:N \g_@@_input_int _gtl } #1
      }
  }
\cs_new_protected:Npn \@@_input_gpop_item_aux:NN #1#2
  {
    \gtl_gpop_left_item:NNTF #1#2
      { \prg_return_true: }
      {
        \int_compare:nNnTF { \gtl_extra_end:N #1 } > \c_zero
          { \prg_return_false: }
          {
            \int_compare:nNnTF \g_@@_input_int = \c_one
              { \prg_return_false: }
              {
                \@@_input_merge:
                \exp_args:Nc \@@_input_gpop_item_aux:NN
                  {
                    g_@@_input_
                    \int_use:N \g_@@_input_int _gtl
                  }
                  #2
              }
          }
      }
  }
%    \end{macrocode}
% \end{macro}
% \end{macro}
%
% \begin{macro}{\@@_input_gpop_tl:N}
%    \begin{macrocode}
\cs_new_protected:Npn \@@_input_gpop_tl:N #1
  { \tl_clear:N #1 \@@_input_gpop_tl_aux:N #1 }
\cs_new_protected:Npn \@@_input_gpop_tl_aux:N #1
  {
    \int_compare:nNnF \g_@@_input_int = \c_zero
      {
        \exp_args:Nc \@@_input_gpop_tl_aux:NN
          { g_@@_input_ \int_use:N \g_@@_input_int _gtl } #1
      }
  }
\cs_new_protected:Npn \@@_input_gpop_tl_aux:NN #1#2
  {
    \gtl_if_tl:NTF #1
      {
        \tl_put_right:Nx #2 { \gtl_left_tl:N #1 }
        \gtl_gclear:N #1
        \int_gdecr:N \g_@@_input_int
        \@@_input_gpop_tl_aux:N #2
      }
      {
        \int_compare:nNnTF \g_@@_input_int > \c_one
          { \int_compare:nNnTF { \gtl_extra_end:N #1 } > \c_zero }
          { \use_i:nn }
          {
            \tl_put_right:Nx #2 { \gtl_left_tl:N #1 }
            \gtl_gpop_left_tl:N #1
          }
          {
            \@@_input_merge:
            \@@_input_gpop_tl_aux:N #2
          }
      }
  }
%    \end{macrocode}
% \end{macro}
%
% \begin{macro}{\@@_back_input:n, \@@_back_input:x}
%   Insert a token list back into the input.  Use \cs{gtl_gclear_new:c}
%   to define the gtl variable if necessary: this happens whenever a
%   new largest value of \cs{g_@@_input_int} is reached.
%    \begin{macrocode}
\cs_new_protected:Npn \@@_back_input:n
  {
    \int_gincr:N \g_@@_input_int
    \gtl_gclear_new:c { g_@@_input_ \int_use:N \g_@@_input_int _gtl }
    \gtl_gset:cn { g_@@_input_ \int_use:N \g_@@_input_int _gtl }
  }
\cs_generate_variant:Nn \@@_back_input:n { V , o }
\cs_new_protected:Npn \@@_back_input:x
  { \@@_exp_args:Nx \@@_back_input:n }
%    \end{macrocode}
% \end{macro}
%
% \begin{macro}{\@@_back_input_gtl:N}
%   Insert a generalized token list back into the input.
%    \begin{macrocode}
\cs_new_protected:Npn \@@_back_input_gtl:N #1
  {
    \gtl_if_tl:NTF #1
      { \@@_back_input:x { \gtl_left_tl:N #1 } }
      {
        \gtl_gconcat:cNc
          { g_@@_input_ \int_use:N \g_@@_input_int _gtl }
          #1
          { g_@@_input_ \int_use:N \g_@@_input_int _gtl }
      }
  }
%    \end{macrocode}
% \end{macro}
%
% \begin{macro}{\@@_back_input:}
%   Insert the last token read back into the input stream.
%    \begin{macrocode}
\cs_new_protected:Npn \@@_back_input:
  { \@@_back_input_gtl:N \l_@@_head_gtl }
%    \end{macrocode}
% \end{macro}
%
% \begin{macro}{\@@_back_input_tl_o:}
%   Insert the \cs{l_@@_head_tl} (may or may not be the last token read)
%   back into the input stream, after expanding it once.  Then print
%   some diagnostic information.
%    \begin{macrocode}
\cs_new_protected:Npn \@@_back_input_tl_o:
  {
    \tl_set:Nx \l_@@_tmpa_tl
      { \exp_args:NV \exp_not:o \l_@@_head_tl }
    \@@_back_input:V \l_@@_tmpa_tl
    \@@_print_expansion:x
      { \tl_to_str:N \l_@@_head_tl = \tl_to_str:N \l_@@_tmpa_tl }
  }
%    \end{macrocode}
% \end{macro}
%
% \subsubsection{Insert token for error recovery}
%
% \begin{macro}{\@@_insert_relax:}
%   This function inserts \TeX{}'s |frozen_relax|.  It is called when a
%   conditional is not done finding its condition, but hits the
%   corresponding \tn{fi} or \tn{or} or \tn{else}, or when \tn{input}
%   appears while \cs{g_@@_name_in_progress_bool} is \texttt{true}.
%    \begin{macrocode}
\cs_new_protected:Npn \@@_insert_relax:
  {
    \@@_back_input:
    \gtl_set_eq:NN \l_@@_head_gtl \c_@@_frozen_relax_gtl
    \@@_back_input:
    \@@_print_action:
  }
%    \end{macrocode}
% \end{macro}
%
% \begin{macro}{\@@_insert_group_begin_error:}
%    \begin{macrocode}
\cs_new_protected:Npn \@@_insert_group_begin_error:
  {
    \tl_set_eq:NN \l_@@_tmpa_tl \l_@@_head_tl
    \@@_back_input:
    \gtl_set_eq:NN \l_@@_head_gtl \c_group_begin_gtl
    \@@_back_input:
    \@@_tex_error:nV { missing-lbrace } \l_@@_tmpa_tl
    \@@_print_action:
  }
%    \end{macrocode}
% \end{macro}
%
% \begin{macro}{\@@_insert_dollar_error:}
%    \begin{macrocode}
\cs_new_protected:Npn \@@_insert_dollar_error:
  {
    \@@_back_input:
    \@@_back_input:n { $ } % $
    \@@_error:nnnnn { missing-dollar } { } { } { } { }
    \@@_print_action:
  }
%    \end{macrocode}
% \end{macro}
%
% \subsubsection{Macro calls}
%
% \begin{macro}[EXP]
%   {
%     \@@_macro_prefix:N,
%     \@@_macro_parameter:N,
%     \@@_macro_replacement:N
%   }
%    \begin{macrocode}
\use:x
  {
    \exp_not:n { \cs_new:Npn \@@_macro_split_do:NN #1 }
      {
        \exp_not:n { \exp_after:wN \@@_macro_split_do:wN }
        \exp_not:n { \token_to_meaning:N #1 \q_mark { } }
        \tl_to_str:n { : } \exp_not:n { -> \q_mark \use_none:nnnn }
        \exp_not:N \q_stop
      }
    \exp_not:n { \cs_new:Npn \@@_macro_split_do:wN }
        \exp_not:n {#1} \tl_to_str:n { : } \exp_not:n { #2 -> }
        \exp_not:n { #3 \q_mark #4 #5 \q_stop #6 }
      { \exp_not:n { #4 #6 {#1} {#2} {#3} } }
  }
\cs_new:Npn \@@_macro_prefix:N #1
  { \@@_macro_split_do:NN #1 \use_i:nnn }
\cs_new:Npn \@@_macro_parameter:N #1
  { \@@_macro_split_do:NN #1 \use_ii:nnn }
\cs_new:Npn \@@_macro_replacement:N #1
  { \@@_macro_split_do:NN #1 \use_iii:nnn }
%    \end{macrocode}
% \end{macro}
%
% ^^A todo: redoc
% \begin{macro}{\@@_macro_call:}
% \begin{macro}
%   {
%     \@@_macro_call_safe:,
%     \@@_macro_call_quick:,
%     \@@_macro_call_quick_loop:NNN,
%     \@@_macro_call_quick_runaway:Nw,
%   }
%   Macros are simply expanded once.  We cannot determine precisely
%   which tokens a macro will need for its parameters, but we know that
%   it must form a balanced token list.  Thus we can be safe by
%   extracting the longest balanced prefix in the input and working with
%   that.
%    \begin{macrocode}
\cs_new_protected:Npn \@@_macro_call:
  {
    \bool_if:NTF \g_@@_speedup_macros_bool
      {
        \tl_set:Nx \l_@@_tmpa_tl
          {^ \exp_after:wN \@@_macro_parameter:N \l_@@_head_tl }
        \@@_tl_if_in:ooTF \c_@@_parameters_tl \l_@@_tmpa_tl
          { \@@_macro_call_quick: } { \@@_macro_call_safe: }
      }
      { \@@_macro_call_safe: }
    \exp_args:NV \@@_back_input:o \l_@@_head_tl
    \@@_print_expansion:
  }
\cs_new_protected:Npn \@@_macro_call_safe:
  {
    \@@_input_gpop_tl:N \l_@@_tmpa_tl
    \tl_put_right:NV \l_@@_head_tl \l_@@_tmpa_tl
  }
\cs_new_protected:Npn \@@_macro_call_quick:
  {
    \exp_after:wN \@@_macro_call_quick_loop:NNN \l_@@_tmpa_tl
      { ? \use_none_delimit_by_q_stop:w } \q_stop
  }
\cs_new_protected:Npn \@@_macro_call_quick_loop:NNN #1#2#3
  {
    \use_none:n #2
    \@@_input_gpop_item:NF \l_@@_tmpa_tl
      { \@@_macro_call_quick_runaway:Nw #3 }
    \tl_put_right:Nx \l_@@_head_tl
      { { \exp_not:V \l_@@_tmpa_tl } }
    \@@_macro_call_quick_loop:NNN
    #3
  }
\cs_new_protected:Npn \@@_macro_call_quick_runaway:Nw #1#2 \q_stop
  {
    \@@_error:nxxxx { runaway-macro-parameter }
      { \tl_to_str:N \l_@@_head_tl } { \tl_to_str:n {#1} } { } { }
  }
%    \end{macrocode}
% \end{macro}
% \end{macro}
%
% \subsection{Expand next token}
%
% \begin{macro}{\@@_expand_do:N}
%   The argument is a command that will almost always be run to continue
%   a loop whose aim is to find the next non-expandable token, for
%   various purposes.  The only case where we will end up grabbing the
%   argument is to suppress the loop by \cs{@@_noexpand:N}.
%   \begin{itemize}
%   \item \cs{@@_get_x_next:} when \TeX{} is looking for the first
%     non-expandable token in the main loop or when looking for numbers,
%     optional spaces etc.
%   \item \cs{@@_get_x_or_protected:} at the start of an alignment cell.
%   \item \cs{@@_get_token_xdef:} in the replacement text of \tn{edef}
%     and \tn{xdef}.
%   \item \cs{@@_get_token_x:} in the argument of \tn{message} and the
%     like.
%   \item \cs{prg_do_nothing:} in \cs{@@_expandafter:} namely after
%     \tn{expandafter}.
%   \end{itemize}
%   We mimick \TeX{}'s structure, distinguishing macros from other
%   commands because we find macro arguments very differently from
%   primitives.
%    \begin{macrocode}
\cs_new_protected:Npn \@@_expand_do:N
  {
    \@@_set_action_text:
    \bool_if:NT \g_@@_internal_debug_bool
      {
        \@@_set_cmd:
        \@@_exp_args:Nx \iow_term:n { Exp:~\int_to_arabic:n { \l_@@_head_cmd_int } }
      }
    \token_if_macro:NTF \l_@@_head_token
      { \@@_macro_call: }
      { \@@_expand_nonmacro: }
  }
%    \end{macrocode}
% \end{macro}
%
% \begin{macro}{\@@_expand_nonmacro:}
%   The token is a primitive.  We find its (cleaned-up) \tn{meaning},
%   and call the function implementing that expansion.  If we do not
%   recognize the meaning then it is probably an unknown primitive.
%   Then do something similar to what we do for macros: get all tokens
%   that are not too unlikely to appear in the arguments of the
%   primitive and expand the resulting token list once before putting it
%   back into the input stream.
%    \begin{macrocode}
\cs_new_protected:Npn \@@_expand_nonmacro:
  {
    \@@_set_cmd_aux_meaning:
    \@@_set_cmd_aux_primitive:oTF { \l_@@_head_meaning_tl }
      {
        \cs_if_exist_use:cF
          { @@_expandable_ \int_use:N \l_@@_head_cmd_int : }
          { \@@_error:nxxxx { internal } { expandable } { } { } { } }
      }
      {
        \@@_error:nxxxx { unknown-primitive }
          { \l_@@_head_meaning_tl } { } { } { }
        \@@_input_gpop_tl:N \l_@@_tmpa_tl
        \tl_put_right:NV \l_@@_head_tl \l_@@_tmpa_tl
        \exp_args:NV \@@_back_input:o \l_@@_head_tl
        \@@_print_expansion:
      }
  }
%    \end{macrocode}
% \end{macro}
%
% \begin{macro}{\@@_get_x_next:}
%   Get a token.  If it is expandable, then expand it, and repeat.  This
%   function does not set the |cmd| and |char| integers.  It is the
%   basis of all routines that look for keywords, numbers, equal signs,
%   filenames, optional spaces etc (in the language of \LaTeX3 these are
%   situations where \TeX{} \enquote{\texttt{f}-expands}).  It is also
%   the basis of the \cs{@@_main_loop:}.
%    \begin{macrocode}
\cs_new_protected:Npn \@@_get_x_next:
  {
    \@@_get_next:
    \@@_token_if_expandable:NT \l_@@_head_token
      { \@@_expand_do:N \@@_get_x_next: }
  }
%    \end{macrocode}
% \end{macro}
%
% \begin{macro}{\@@_get_x_or_protected:}
%   Get a token.  If it is expandable, but not protected, then expand
%   it, and repeat.  This function does not set the |cmd| and |char|
%   integers.  This function is not used at present: it will be used at
%   the start of alignment cells.
%    \begin{macrocode}
\cs_new_protected:Npn \@@_get_x_or_protected:
  {
    \@@_get_next:
    \@@_token_if_protected:NF \l_@@_head_token
      { \@@_expand_do:N \@@_get_x_or_protected: }
  }
%    \end{macrocode}
% \end{macro}
%
% \begin{macro}{\@@_get_token_xdef:, \@@_get_token_x:}
%   These are similar to \cs{@@_get_x_next:}, for use when reading the
%   replacement text of \tn{edef}/\tn{xdef} or the argument of a
%   primitive like \tn{message} that should be expanded as we read
%   tokens.  Loop until finding a non-expandable token (or protected
%   macro).
%    \begin{macrocode}
\cs_new_protected:Npn \@@_get_token_xdef:
  {
    \@@_get_next:
    \@@_token_if_protected:NF \l_@@_head_token
      { \@@_expand_do:N \@@_get_token_xdef: }
  }
\cs_new_protected:Npn \@@_get_token_x:
  {
    \@@_get_next:
    \@@_token_if_protected:NF \l_@@_head_token
      { \@@_expand_do:N \@@_get_token_x: }
  }
%    \end{macrocode}
% \end{macro}
%
% \subsection{Basic scanning subroutines}
%
% ^^A todo: implement |test_no_ligatures|?
% ^^A todo: implement |get_tag_code|?
%
% \begin{macro}{\@@_get_x_non_blank:}
%   This function does not set the |cmd| and |char| integers.
%    \begin{macrocode}
\cs_new_protected:Npn \@@_get_x_non_blank:
  {
    \@@_get_x_next:
    \token_if_eq_catcode:NNT \l_@@_head_token \c_space_token
      { \@@_get_x_non_blank: }
  }
%    \end{macrocode}
% \end{macro}
%
% \begin{macro}{\@@_get_x_non_relax:}
%   This function does not set the |cmd| and |char| integers.
%    \begin{macrocode}
\cs_new_protected:Npn \@@_get_x_non_relax:
  {
    \@@_get_x_next:
    \token_if_eq_meaning:NNTF \l_@@_head_token \scan_stop:
      { \@@_get_x_non_relax: }
      {
        \token_if_eq_meaning:NNTF \l_@@_head_token \@@_special_relax:
          { \@@_get_x_non_relax: }
          {
            \token_if_eq_catcode:NNT \l_@@_head_token \c_space_token
              { \@@_get_x_non_relax: }
          }
      }
  }
%    \end{macrocode}
% \end{macro}
%
% \begin{macro}{\@@_skip_optional_space:}
%    \begin{macrocode}
\cs_new_protected:Npn \@@_skip_optional_space:
  {
    \@@_get_x_next:
    \token_if_eq_catcode:NNF \l_@@_head_token \c_space_token
      { \@@_back_input: }
  }
%    \end{macrocode}
% \end{macro}
%
% \begin{macro}{\@@_scan_optional_equals:}
%   See \TeX{}'s |scan_optional_equals|.  In all cases we forcefully
%   insert an equal sign in the output, because this sign is required,
%   as \cs{@@_scan_something_internal:n} leaves raw numbers in
%   the previous-input sequence.
%    \begin{macrocode}
\cs_new_protected:Npn \@@_scan_optional_equals:
  {
    \@@_get_x_non_blank:
    \tl_if_eq:NNTF \l_@@_head_tl \c_@@_eq_tl
      { \@@_prev_input:n { = } }
      {
        \@@_prev_input_silent:n { = }
        \@@_back_input:
      }
  }
%    \end{macrocode}
% \end{macro}
%
% \begin{macro}{\@@_scan_left_brace:}
%   The presence of \tn{relax} is allowed before a begin-group token.
%   If there is no begin-group token, insert one, produce an error, and
%   scan that begin-group using \cs{@@_get_next:}.
%    \begin{macrocode}
\cs_new_protected:Npn \@@_scan_left_brace:
  {
    \@@_get_x_non_relax:
    \token_if_eq_catcode:NNF \l_@@_head_token \c_group_begin_token
      {
        \@@_insert_group_begin_error:
        \@@_get_next:
      }
  }
%    \end{macrocode}
% \end{macro}
%
% \begin{macro}{\@@_scan_keyword:n}
% \begin{macro}[TF]{\@@_scan_keyword:n}
% \begin{macro}
%   {
%     \@@_scan_keyword_loop:NNN,
%     \@@_scan_keyword_test:NNTF,
%     \@@_scan_keyword_true:,
%     \@@_scan_keyword_false:w
%   }
%
%   The details of how \TeX{} looks for keywords are quite tricky to get
%   right, in particular with respect to expansion, case-insensitivity,
%   and spaces.  We get rid of the case issue by requiring the keyword
%   to be given in both cases, intertwined: for instance,
%   \cs{@@_scan_keyword:n} |{| |pPtT| |}|.  Then loop through pairs of
%   letters (which should be matching lowercase and uppercase letters).
%   The looping auxiliary takes three arguments, the first of which is a
%   boolean, \texttt{true} if spaces are allowed (no letter of the
%   keyword has been found yet).  At each iteration, get a token, with
%   expansion, and test whether it is a non-active character equal (in
%   character code) to either letter of the pair: this happens if the
%   token is not ``definable'' (neither a control sequence nor an active
%   character) and it has the right string representation\ldots{} well,
%   it could also be doubled (macro parameter character), hence we look
%   at the first character only; spaces become an empty string, but this
%   works out because no keyword contains a space.  So, at each
%   iteration, if the token is the correct non-active character, add it
%   to the previous-input sequence (as a generalized token list since
%   keywords may match begin-group or end-group characters), and
%   otherwise break with \cs{@@_scan_keyword_false:w}, unless we are
%   still at the beginning of the keyword and the token is a space.
%   When the loop reaches the end of the keyword letter pairs, complain
%   if there were an odd number of letters, and otherwise conclude the
%   loop with \cs{@@_scan_keyword_true:}, which stores the keyword,
%   converted to a string.  Note that \TeX{}'s skipping of leading
%   spaces here must be intertwined with the search for keyword, as is
%   shown by the (plain \TeX{}) example
%   \begin{verbatim}
%     \lccode32=`f \lowercase{\def\fspace{ }}
%     \skip0=1pt plus 1 \fspace il\relax
%     \message{\the\skip0} % => 1pt plus 1fil
%   \end{verbatim}
%    \begin{macrocode}
\cs_new_protected:Npn \@@_scan_keyword:n #1
  { \@@_scan_keyword:nTF {#1} { } { } }
\prg_new_protected_conditional:Npnn \@@_scan_keyword:n #1
  { T , F , TF }
  {
    \@@_prev_input_gpush_gtl:
    \@@_scan_keyword_loop:NNN \c_true_bool
      #1 \q_recursion_tail \q_recursion_tail \q_recursion_stop
  }
\cs_new_protected:Npn \@@_scan_keyword_loop:NNN #1#2#3
  {
    \quark_if_recursion_tail_stop_do:nn {#2}
      { \@@_scan_keyword_true: }
    \quark_if_recursion_tail_stop_do:nn {#3}
      { \@@_error:nxxxx { internal } { odd-keyword-length } { } { } { } }
    \@@_get_x_next:
    \@@_scan_keyword_test:NNTF #2#3
      {
        \@@_prev_input_gtl:N \l_@@_head_gtl
        \@@_scan_keyword_loop:NNN \c_false_bool
      }
      {
        \token_if_eq_catcode:NNF \l_@@_head_token \c_space_token
          { \@@_scan_keyword_false:w }
        \bool_if:NF #1
          { \@@_scan_keyword_false:w }
        \@@_scan_keyword_loop:NNN #1#2#3
      }
  }
\prg_new_protected_conditional:Npnn \@@_scan_keyword_test:NN #1#2
  { TF }
  {
    \@@_gtl_if_head_is_definable:NTF \l_@@_head_gtl
      { \prg_return_false: }
      {
        \str_if_eq_x:nnTF
          { \str_head:f { \gtl_to_str:N \l_@@_head_gtl } } {#1}
          { \prg_return_true: }
          {
            \str_if_eq_x:nnTF
              { \str_head:f { \gtl_to_str:N \l_@@_head_gtl } } {#2}
              { \prg_return_true: }
              { \prg_return_false: }
          }
      }
  }
\cs_new_protected:Npn \@@_scan_keyword_true:
  {
    \@@_prev_input_gpop_gtl:N \l_@@_tmpb_gtl
    \@@_prev_input:x { \gtl_to_str:N \l_@@_tmpb_gtl }
    \prg_return_true:
  }
\cs_new_protected:Npn \@@_scan_keyword_false:w
    #1 \q_recursion_stop
  {
    \@@_back_input:
    \@@_prev_input_gpop_gtl:N \l_@@_tmpb_gtl
    \@@_back_input_gtl:N \l_@@_tmpb_gtl
    \prg_return_false:
  }
%    \end{macrocode}
% \end{macro}
% \end{macro}
% \end{macro}
%
% \begin{macro}{\@@_scan_to:}
%   Used when |to| is mandatory: after \tn{read} or \tn{readline} and
%   after \tn{vsplit}.
%    \begin{macrocode}
\cs_new_protected:Npn \@@_scan_to:
  {
    \@@_scan_keyword:nF { tToO }
      {
        \@@_error:nnnnn { missing-to } { } { } { } { }
        \@@_prev_input:n { to }
      }
  }
%    \end{macrocode}
% \end{macro}
%
% \begin{macro}{\@@_scan_font_ident:}
%   Find a font identifier.
%    \begin{macrocode}
\cs_new_protected:Npn \@@_scan_font_ident:
  {
    \@@_get_x_non_blank:
    \@@_set_cmd:
    \int_case:nnF \l_@@_head_cmd_int
      {
        { \@@_tex_use:n { def_font } }
          { \@@_prev_input:V \l_@@_head_tl }
        { \@@_tex_use:n { letterspace_font } }
          { \@@_prev_input:V \l_@@_head_tl }
        { \@@_tex_use:n { pdf_copy_font } }
          { \@@_prev_input:V \l_@@_head_tl }
        { \@@_tex_use:n { set_font } }
          { \@@_prev_input:V \l_@@_head_tl }
        { \@@_tex_use:n { def_family } }
          {
            \@@_prev_input:V \l_@@_head_tl
            \@@_scan_int:
          }
      }
      {
        \@@_error:nnnnn { missing-font-id } { } { } { } { }
        \@@_back_input:
        \@@_prev_input:n { \@@_nullfont: }
      }
  }
%    \end{macrocode}
% \end{macro}
%
% \begin{macro}{\@@_scan_font_int:}
%   Find operands for one of \tn{hyphenchar}'s friends (command code
%   |assign_font_int=78|).
%    \begin{macrocode}
\cs_new_protected:Npn \@@_scan_font_int:
  {
    \int_case:nnF \l_@@_head_char_int
      {
        { 0 } { \@@_scan_font_ident: }
        { 1 } { \@@_scan_font_ident: }
        { 6 } { \@@_scan_font_ident: }
      }
      { \@@_scan_font_ident: \@@_scan_int: }
  }
%    \end{macrocode}
% \end{macro}
%
% \begin{macro}{\@@_scan_font_dimen:}
%   Find operands for \tn{fontdimen}.
%    \begin{macrocode}
\cs_new_protected:Npn \@@_scan_font_dimen:
  {
    \@@_scan_int:
    \@@_scan_font_ident:
  }
%    \end{macrocode}
% \end{macro}
%
% ^^A todo: rename this function because all other scan functions do get_x_next themselves.
% \begin{macro}{\@@_scan_something_internal:n}
% \begin{macro}{\@@_scan_something_aux:nwn}
%   Receives an (explicit) \enquote{level} argument:
%   \begin{itemize}
%   \item |int_val=0| for integer values;
%   \item |dimen_val=1| for dimension values;
%   \item |glue_val=2| for glue specifications;
%   \item |mu_val=3| for math glue specifications;
%   \item |ident_val=4| for font identifiers (this never happens);
%   \item |tok_val=5| for token lists (after |\the| or |\showthe|).
%   \end{itemize}
%   Scans something internal, and places its value, converted to the
%   given level, to the right of the last item of
%   the previous-input sequence, then sets \cs{g_@@_val_level_int} to the
%   found level (level before conversion, so this may be higher than
%   requested).
%
%   From \cs{@@_thing_case:}, get the information about what level is
%   produced by the given token once it has received all its operands
%   (head of \cs{l_@@_tmpa_tl}), and about what to do to find those
%   operands (tail of \cs{l_@@_tmpa_tl}).  If the first token may not
%   appear after \tn{the} at all, \cs{@@_thing_case:} gives level~$8$.
%
%   If the argument (|#3|~in the auxiliary) is $<4$ but the level that
%   will be produced (|#1|~in the auxiliary) is $\geq 4$ (that is,
%   $4$, $5$, or $8$) complain about a missing number and insert a
%   zero dimension, to get exactly \TeX{}'s error recovery.  If the
%   level produced is~$8$, complain that |\the| cannot do this.
%
%   Otherwise, scan the arguments (in a new input level).  If both the
%   argument and the level produced are $<4$, then get the value with
%   \cs{@@_thing_use_get:nnNN} which downgrades from glue to dimension
%   to integer and produces the |incompatible-units| error if needed.
%   The only remaining case is that the argument is $5$ (since $4$ is
%   never used) and the level produced is that or less: then the value
%   found is used with \cs{@@_the:w}.
%
%   Finally, tell the user the tokens that have been found (if there
%   was a single token, its meaning as well) and their value.  Use |=>|
%   rather than |=| because the value displayed is the value used, not
%   the actual value (this matters in constructions such as
%   |\parindent=\parskip| where a skip or a dimen is downgraded to a
%   dimen or an int, or when there was an error).
%    \begin{macrocode}
\cs_new_protected:Npn \@@_scan_something_internal:n #1
  {
    \@@_set_cmd:
    \@@_set_action_text:
    \tl_set:Nf \l_@@_tmpa_tl { \@@_thing_case: }
    \exp_after:wN \@@_scan_something_aux:nwn
      \l_@@_tmpa_tl \q_stop {#1}
  }
\cs_new_protected:Npn \@@_scan_something_aux:nwn #1#2 \q_stop #3
  {
    \int_compare:nT { #3 < 4 <= #1 }
      {
        \@@_back_input:
        \@@_tex_error:nV { missing-number } \l_@@_head_tl
        \@@_thing_use_get:nnNN { 1 } {#3} \c_zero_dim \l_@@_tmpa_tl
        \@@_scan_something_internal_auxii:Vn \l_@@_tmpa_tl { 1 }
        \@@_break:w
      }
    \int_compare:nNnT {#1} = { 8 }
      {
        \@@_tex_error:nV { the-cannot } \l_@@_head_tl
        \@@_scan_something_internal_auxii:Vn \c_zero { 0 }
        \@@_break:w
      }
    \tl_if_empty:nF {#2}
      {
        \@@_prev_input_gpush:N \l_@@_head_tl
        \@@_print_action:
        #2
        \@@_prev_input_gpop:N \l_@@_head_tl
      }
    \int_compare:nNnTF {#3} < { 4 }
      { \@@_thing_use_get:nnNN {#1} {#3} \l_@@_head_tl \l_@@_tmpa_tl }
      { \tl_set:Nx \l_@@_tmpa_tl { \@@_the:w \l_@@_head_tl } }
    \@@_scan_something_internal_auxii:Vn \l_@@_tmpa_tl {#1}
    \@@_break_point:
    \int_compare:nNnT {#3} < { 4 } { \@@_print_action: }
  }
\cs_new_protected:Npn \@@_scan_something_internal_auxii:nn #1#2
  {
    \@@_prev_input_silent:n {#1}
    \@@_set_action_text:
    \@@_set_action_text:x
      { \g_@@_action_text_str \use:n { ~ => ~ } \tl_to_str:n {#1} }
    \int_gset:Nn \g_@@_val_level_int {#2}
  }
\cs_generate_variant:Nn \@@_scan_something_internal_auxii:nn { V }
%    \end{macrocode}
% \end{macro}
% \end{macro}
%
% ^^A todo: \the can be followed by ignore_spaces
% \begin{macro}[EXP]
%   {\@@_thing_case:, \@@_thing_last_item:, \@@_thing_register:}
%   This expands to a digit (the level generated by whatever token is
%   the current |head|), followed by some code to fetch necessary
%   operands.  In most cases, this can be done by simply looking at the
%   |cmd| integer, but for |last_item|, |set_aux| and |register|, the level of the
%   token depends on the |char| integer.  When the token is not allowed
%   after \tn{the} (or at any other position where
%   \cs{@@_scan_something_internal:n} is called), the resulting level
%   is~$8$, large enough so that the main function knows it is
%   forbidden.
%    \begin{macrocode}
\cs_new:Npn \@@_thing_case:
  {
    \int_case:nnF \l_@@_head_cmd_int
      {
        { 68 } { 0                         } % char_given
        { 69 } { 0                         } % math_given
        { 70 } { \@@_thing_last_item:      } % last_item
        { 71 } { 5 \@@_scan_toks_register: } % toks_register
        { 72 } { 5                         } % assign_toks
        { 73 } { 0                         } % assign_int
        { 74 } { 1                         } % assign_dimen
        { 75 } { 2                         } % assign_glue
        { 76 } { 3                         } % assign_mu_glue
        { 77 } { 1 \@@_scan_font_dimen:    } % assign_font_dimen
        { 78 } { 0 \@@_scan_font_int:      } % assign_font_int
        { 79 } { \@@_thing_set_aux:        } % set_aux
        { 80 } { 0                         } % set_prev_graf
        { 81 } { 1                         } % set_page_dimen
        { 82 } { 0                         } % set_page_int
        { 83 } { 1 \@@_scan_int:           } % set_box_dimen
        { 84 } { 0 \@@_scan_int:           } % set_shape
        { 85 } { 0 \@@_scan_int:           } % def_code
        { 86 } { 4 \@@_scan_int:           } % def_family
        { 87 } { 4                         } % set_font
        { 88 } { 4                         } % def_font
        { 89 } { \@@_thing_register:       } % register
        {101 } { 4                         } % letterspace_font
        {102 } { 4                         } % pdf_copy_font
      }
      { 8 }
  }
\cs_new:Npn \@@_thing_set_aux:
  { \int_compare:nNnTF \l_@@_head_char_int = { 1 } { 1 } { 0 } }
\cs_new:Npn \@@_thing_last_item:
  {
    \int_compare:nNnTF \l_@@_head_char_int < { 26 }
      {
        \int_case:nnF \l_@@_head_char_int
          {
            { 1 } { 1 } % lastkern
            { 2 } { 2 } % lastskip
          }
          { 0 } % other integer parameters
      }
      {
        \int_case:nnF \l_@@_head_char_int
          {
            { 26 } { 0 \@@_scan_normal_glue: } % gluestretchorder
            { 27 } { 0 \@@_scan_normal_glue: } % glueshrinkorder
            { 28 } % fontcharwd
              { 1 \@@_scan_font_ident: \@@_scan_int: }
            { 29 } % fontcharht
              { 1 \@@_scan_font_ident: \@@_scan_int: }
            { 30 } % fontchardp
              { 1 \@@_scan_font_ident: \@@_scan_int: }
            { 31 } % fontcharic
              { 1 \@@_scan_font_ident: \@@_scan_int: }
            { 32 } { 1 \@@_scan_int: } % parshapelength
            { 33 } { 1 \@@_scan_int: } % parshapeindent
            { 34 } { 1 \@@_scan_int: } % parshapedimen
            { 35 } { 1 \@@_scan_normal_glue: } % gluestretch
            { 36 } { 1 \@@_scan_normal_glue: } % glueshrink
            { 37 } { 2 \@@_scan_mu_glue: } % mutoglue
            { 38 } { 3 \@@_scan_normal_glue: } % gluetomu
            { 39 } % numepr
              { 0 \@@_scan_expr:N \@@_scan_int: }
            { 40 } % dimexpr
              { 1 \@@_scan_expr:N \@@_scan_normal_dimen: }
            { 41 } % glueexpr
              { 2 \@@_scan_expr:N \@@_scan_normal_glue: }
            { 42 } % muexpr
              { 3 \@@_scan_expr:N \@@_scan_mu_glue: }
          }
          { }
      }
  }
\cs_new:Npn \@@_thing_register:
  {
    \int_eval:n { \l_@@_head_char_int / 1 000 000 - 1 }
    \int_compare:nNnT { \tl_tail:V \l_@@_head_char_int } = \c_zero
      { \@@_scan_int: }
  }
%    \end{macrocode}
% \end{macro}
%
% \begin{macro}{\@@_scan_toks_register:}
%   A case where getting operands is not completely trivial.
%    \begin{macrocode}
\cs_new_protected:Npn \@@_scan_toks_register:
  {
    \int_compare:nNnT \l_@@_head_char_int = \c_zero
      { \@@_scan_int: }
  }
%    \end{macrocode}
% \end{macro}
%
% \begin{macro}{\@@_thing_use_get:nnNN}
%   Given a level found~|#1| and a target level~|#2| (both in $[0,3]$),
%   turn the token list |#3| into the desired level or less, and store
%   the result in |#4|.
%    \begin{macrocode}
\cs_new_protected:Npn \@@_thing_use_get:nnNN #1#2#3#4
  {
    \int_compare:nNnTF {#2} < { 3 }
      {
        \int_compare:nNnT {#1} = { 3 }
          { \@@_tex_error:nV { incompatible-units } #3 }
        \tl_set:Nx #4
          {
            \int_case:nn { \int_min:nn {#1} {#2} }
              {
                { 0 } \int_eval:n
                { 1 } \dim_eval:n
                { 2 } \skip_eval:n
              }
            { \int_compare:nNnT {#1} = { 3 } \etex_mutoglue:D #3 }
          }
      }
      {
        \int_case:nnF {#1}
          {
            { 0 } { \tl_set:Nx #4 { \int_eval:n {#3} } }
            { 3 } { \tl_set:Nx #4 { \muskip_eval:n {#3} } }
          }
          {
            \@@_tex_error:nV { incompatible-units } #3
            \tl_set:Nx #4 { \muskip_eval:n { \etex_gluetomu:D #3 } }
          }
      }
  }
%    \end{macrocode}
% \end{macro}
%
% ^^A todo: comment
% \begin{macro}{\@@_scan_expr:N}
% \begin{macro}{\@@_scan_expr_aux:NN, \@@_scan_factor:N}
%    \begin{macrocode}
\cs_new_protected:Npn \@@_scan_expr:N #1
  { \@@_scan_expr_aux:NN #1 \c_false_bool }
\cs_new_protected:Npn \@@_scan_expr_aux:NN #1#2
  {
    \@@_get_x_non_blank:
    \@@_scan_factor:N #1
    \@@_scan_expr_op:NN #1#2
  }
\cs_new_protected:Npn \@@_scan_expr_op:NN #1#2
  {
    \@@_get_x_non_blank:
    \tl_case:NnF \l_@@_head_tl
      {
        \c_@@_plus_tl
          {
            \@@_prev_input:V \l_@@_head_tl
            \@@_scan_expr_aux:NN #1#2
          }
        \c_@@_minus_tl
          {
            \@@_prev_input:V \l_@@_head_tl
            \@@_scan_expr_aux:NN #1#2
          }
        \c_@@_times_tl
          {
            \@@_prev_input:V \l_@@_head_tl
            \@@_get_x_non_blank:
            \@@_scan_factor:N \@@_scan_int:
            \@@_scan_expr_op:NN #1#2
          }
        \c_@@_over_tl
          {
            \@@_prev_input:V \l_@@_head_tl
            \@@_get_x_non_blank:
            \@@_scan_factor:N \@@_scan_int:
            \@@_scan_expr_op:NN #1#2
          }
        \c_@@_rp_tl
          {
            \bool_if:NTF #2
              { \@@_prev_input:V \l_@@_head_tl }
              { \@@_back_input: }
          }
      }
      {
        \bool_if:NTF #2
          {
            \@@_error:nnnnn { missing-rparen } { } { } { } { }
            \@@_back_input:
            \@@_prev_input:V \c_@@_rp_tl
          }
          {
            \token_if_eq_meaning:NNF \l_@@_head_token \scan_stop:
              { \@@_back_input: }
          }
      }
  }
\cs_new_protected:Npn \@@_scan_factor:N #1
  {
    \tl_if_eq:NNTF \l_@@_head_tl \c_@@_lp_tl
      {
        \@@_prev_input:V \l_@@_head_tl
        \@@_scan_expr_aux:NN #1 \c_true_bool
      }
      {
        \@@_back_input:
        #1
      }
  }
%    \end{macrocode}
% \end{macro}
% \end{macro}
%
% \begin{macro}{\@@_scan_signs:}
%   Skips blanks, scans signs, and places them to the right of the last
%   item of \cs{@@_prev_input:n}.
%    \begin{macrocode}
\cs_new_protected:Npn \@@_scan_signs:
  {
    \@@_get_x_non_blank:
    \tl_if_eq:NNTF \l_@@_head_tl \c_@@_plus_tl
      {
        \@@_prev_input:V \l_@@_head_tl
        \@@_scan_signs:
      }
      {
        \tl_if_eq:NNT \l_@@_head_tl \c_@@_minus_tl
          {
            \@@_prev_input:V \l_@@_head_tl
            \@@_scan_signs:
          }
      }
  }
%    \end{macrocode}
% \end{macro}
%
% \begin{macro}{\@@_scan_int:}
% \begin{macro}
%   {\@@_scan_int_char:, \@@_scan_int_lq:, \@@_scan_int_explicit:n}
%    \begin{macrocode}
\cs_new_protected:Npn \@@_scan_int:
  {
    \@@_scan_signs:
    \@@_set_cmd:
    \@@_cmd_if_internal:TF
      { \@@_scan_something_internal:n { 0 } }
      { \@@_scan_int_char: }
  }
\cs_new_protected:Npn \@@_scan_int_char:
  {
    \tl_case:NnF \l_@@_head_tl
      {
        \c_@@_lq_tl { \@@_scan_int_lq: }
        \c_@@_rq_tl
          {
            \@@_prev_input:V \l_@@_head_tl
            \@@_get_x_next:
            \@@_scan_int_explicit:Nn \c_false_bool { ' }
          }
        \c_@@_dq_tl
          {
            \@@_prev_input:V \l_@@_head_tl
            \@@_get_x_next:
            \@@_scan_int_explicit:Nn \c_false_bool { " }
          }
      }
      { \@@_scan_int_explicit:Nn \c_false_bool { } }
  }
\cs_new_protected:Npn \@@_scan_int_lq:
  {
    \@@_get_next:
    \@@_gtl_if_head_is_definable:NF \l_@@_head_gtl
      {
        \tl_set:Nx \l_@@_head_tl
          { \@@_token_to_char:N \l_@@_head_token }
      }
    \tl_set:Nx \l_@@_tmpa_tl
      { \int_eval:n { \exp_after:wN ` \l_@@_head_tl } }
    \@@_prev_input_silent:V \l_@@_tmpa_tl
    \@@_print_action:x
      { ` \gtl_to_str:N \l_@@_head_gtl = \l_@@_tmpa_tl }
    \@@_skip_optional_space:
  }
\cs_new_protected:Npn \@@_scan_int_explicit:Nn #1#2
  {
    \if_int_compare:w \c_one
        < #2 1 \exp_after:wN \exp_not:N \l_@@_head_tl \exp_stop_f:
      \exp_after:wN \use_i:nn
    \else:
      \exp_after:wN \use_ii:nn
    \fi:
    {
      \@@_prev_input:V \l_@@_head_tl
      \@@_get_x_next:
      \@@_scan_int_explicit:Nn \c_true_bool {#2}
    }
    {
      \token_if_eq_catcode:NNF \l_@@_head_token \c_space_token
        { \@@_back_input: }
      \bool_if:NF #1
        {
          \@@_tex_error:nV { missing-number } \l_@@_head_tl
          \@@_prev_input:n { 0 }
        }
    }
  }
%    \end{macrocode}
% \end{macro}
% \end{macro}
%
% \begin{macro}{\@@_scan_normal_dimen:}
%    \begin{macrocode}
\cs_new_protected:Npn \@@_scan_normal_dimen:
  { \@@_scan_dimen:nN { 2 } \c_false_bool }
%    \end{macrocode}
% \end{macro}
%
% \begin{macro}{\@@_scan_dimen:nN}
%   The first argument is $2$ if the unit may not be |mu| and $3$ if
%   the unit must be |mu| (or |fil|).  The second argument is
%   \cs{c_true_bool} if |fil|, |fill|, |filll| are permitted, and is
%   otherwise |false|.  These arguments are similar to those of
%   \TeX{}'s own |scan_dimen| procedure, in which |mu| is |bool(#1=3)|
%   and |inf| is |#2|.  The third argument of this procedure is omitted
%   here, as the corresponding shortcut is provided as a separate
%   function, \cs{@@_scan_dim_unit:nN}.
%    \begin{macrocode}
\cs_new_protected:Npn \@@_scan_dimen:nN #1#2
  {
    \@@_scan_signs:
    \@@_prev_input_gpush:
    \@@_set_cmd:
    \@@_cmd_if_internal:TF
      {
        \int_compare:nNnTF {#1} = { 3 }
          { \@@_scan_something_internal:n { 3 } }
          { \@@_scan_something_internal:n { 1 } }
        \int_compare:nNnT \g_@@_val_level_int = { 0 }
          { \@@_scan_dim_unit:nN {#1} #2 }
      }
      { \@@_scan_dimen_char:nN {#1} #2 }
    \@@_prev_input_gpop:N \l_@@_head_tl
    \@@_prev_input_silent:V \l_@@_head_tl
  }
\cs_new_protected:Npn \@@_scan_dimen_char:nN #1#2
  {
    \tl_if_eq:NNT \l_@@_head_tl \c_@@_comma_tl
      { \tl_set_eq:NN \l_@@_head_tl \c_@@_point_tl }
    \tl_if_eq:NNTF \l_@@_head_tl \c_@@_point_tl
      {
        \@@_prev_input:n { . }
        \@@_scan_decimal_loop:
      }
      {
        \@@_tl_if_in:ooTF { 0123456789 } \l_@@_head_tl
          {
            \@@_back_input:
            \@@_scan_int:
            \tl_if_eq:NNT \l_@@_head_tl \c_@@_comma_tl
              { \tl_set_eq:NN \l_@@_head_tl \c_@@_point_tl }
            \tl_if_eq:NNT \l_@@_head_tl \c_@@_point_tl
              {
                \@@_input_gpop:N \l_@@_tmpb_gtl
                \@@_prev_input:n { . }
                \@@_scan_decimal_loop:
              }
          }
          {
            \@@_back_input:
            \@@_scan_int:
          }
      }
    \@@_scan_dim_unit:nN {#1} #2
  }
\cs_new_protected:Npn \@@_scan_dim_unit:nN #1#2
  {
    \bool_if:NT #2
      {
        \@@_scan_keyword:nT { fFiIlL }
          {
            \@@_scan_inf_unit_loop:
            \@@_break:w
          }
      }
    \@@_get_x_non_blank:
    \@@_set_cmd:
    \@@_cmd_if_internal:TF
      {
        \@@_prev_input_gpush:
        \@@_scan_something_internal:n {#1}
        \@@_prev_input_join_get:nN {#1} \l_@@_tmpa_tl
        \@@_prev_input_gpush:N \l_@@_tmpa_tl
        \@@_break:w
      }
      { \@@_back_input: }
    \int_compare:nNnT {#1} = { 3 }
      {
        \@@_scan_keyword:nT { mMuU } { \@@_break:w }
        \@@_tex_error:nV { missing-mu } \l_@@_head_tl
        \@@_prev_input:n { mu }
        \@@_break:w
      }
    \@@_scan_keyword:nT { eEmM } { \@@_break:w }
    \@@_scan_keyword:nT { eExX } { \@@_break:w }
    \@@_scan_keyword:nT { pPxX } { \@@_break:w }
    \@@_scan_keyword:nT { tTrRuUeE }
      { \@@_prepare_mag: }
    \@@_scan_keyword:nT { pPtT } { \@@_break:w }
    \@@_scan_keyword:nT { iInN } { \@@_break:w }
    \@@_scan_keyword:nT { pPcC } { \@@_break:w }
    \@@_scan_keyword:nT { cCmM } { \@@_break:w }
    \@@_scan_keyword:nT { mMmM } { \@@_break:w }
    \@@_scan_keyword:nT { bBpP } { \@@_break:w }
    \@@_scan_keyword:nT { dDdD } { \@@_break:w }
    \@@_scan_keyword:nT { cCcC } { \@@_break:w }
    \@@_scan_keyword:nT { nNdD } { \@@_break:w }
    \@@_scan_keyword:nT { nNcC } { \@@_break:w }
    \@@_scan_keyword:nT { sSpP } { \@@_break:w }
    \@@_tex_error:nV { missing-pt } \l_@@_head_tl
    \@@_prev_input:n { pt }
    \@@_break_point:
  }
\cs_new_protected:Npn \@@_scan_inf_unit_loop:
  { \@@_scan_keyword:nT { lL } { \@@_scan_inf_unit_loop: } }
\cs_new_protected:Npn \@@_scan_decimal_loop:
  {
    \@@_get_x_next:
    \tl_if_empty:NTF \l_@@_head_tl
      { \use_ii:nn }
      { \@@_tl_if_in:ooTF { 0123456789 } \l_@@_head_tl }
      {
        \@@_prev_input:V \l_@@_head_tl
        \@@_scan_decimal_loop:
      }
      {
        \token_if_eq_catcode:NNF \l_@@_head_token \c_space_token
          { \@@_back_input: }
        \@@_prev_input_silent:n { ~ }
      }
  }
%    \end{macrocode}
% \end{macro}
%
% \begin{macro}{\@@_scan_normal_glue:, \@@_scan_mu_glue:}
%    \begin{macrocode}
\cs_new_protected:Npn \@@_scan_normal_glue:
  { \@@_scan_glue:n { 2 } }
\cs_new_protected:Npn \@@_scan_mu_glue:
  { \@@_scan_glue:n { 3 } }
%    \end{macrocode}
% \end{macro}
%
% \begin{macro}{\@@_scan_glue:n}
%    \begin{macrocode}
\cs_new_protected:Npn \@@_scan_glue:n #1
  {
    \@@_prev_input_gpush:
    \@@_scan_signs:
    \@@_prev_input_gpush:
    \@@_set_cmd:
    \@@_cmd_if_internal:TF
      {
        \@@_scan_something_internal:n {#1}
        \int_case:nnF \g_@@_val_level_int
          {
            { 0 } { \@@_scan_dim_unit:nN {#1} \c_false_bool }
            { 1 } { }
          }
          { \@@_break:w }
      }
      { \@@_back_input: \@@_scan_dimen:nN {#1} \c_false_bool }
    \@@_prev_input_join_get:nN {#1} \l_@@_tmpa_tl
    \@@_prev_input_gpush:
    \@@_prev_input_gpush:N \l_@@_tmpa_tl
    \@@_scan_keyword:nT { pPlLuUsS }
      { \@@_scan_dimen:nN {#1} \c_true_bool }
    \@@_scan_keyword:nT { mMiInNuUsS }
      { \@@_scan_dimen:nN {#1} \c_true_bool }
    \@@_break_point:
    \@@_prev_input_join_get:nN {#1} \l_@@_tmpa_tl
    \@@_prev_input_silent:V \l_@@_tmpa_tl
  }
%    \end{macrocode}
% \end{macro}
%
% ^^A todo: allow \input".. .."
% \begin{macro}{\@@_scan_file_name:}
% ^^A todo: area_delimiter and ext_delimiter ??
%    \begin{macrocode}
\cs_new_protected:Npn \@@_scan_file_name:
  {
    \bool_gset_true:N \g_@@_name_in_progress_bool
    \@@_get_x_non_blank:
    \@@_scan_file_name_loop:
    \bool_gset_false:N \g_@@_name_in_progress_bool
    \@@_prev_input_silent:n { ~ }
  }
\cs_new_protected:Npn \@@_scan_file_name_loop:
  {
    \@@_gtl_if_head_is_definable:NTF \l_@@_head_gtl
      { \@@_back_input: }
      {
        \tl_set:Nx \l_@@_tmpa_tl
          { \@@_token_to_char:N \l_@@_head_token }
        \tl_if_eq:NNF \l_@@_tmpa_tl \c_space_tl
          {
            \@@_prev_input_silent:V \l_@@_tmpa_tl
            \@@_get_x_next:
            \@@_scan_file_name_loop:
          }
      }
  }
%    \end{macrocode}
% \end{macro}
%
% \begin{macro}{\@@_scan_r_token:}
%   This is analogous to \TeX{}'s |get_r_token|.  We store in
%   \cs{l_@@_defined_tl} the token which we found, as this is what will
%   be defined by the next assignment.
%    \begin{macrocode}
\cs_new_protected:Npn \@@_scan_r_token:
  {
    \bool_do_while:nn
      { \tl_if_eq_p:NN \l_@@_head_tl \c_space_tl }
      { \@@_get_next: }
    \@@_gtl_if_head_is_definable:NF \l_@@_head_gtl
      {
        \@@_error:nnnnn { missing-cs } { } { } { } { }
        \@@_back_input:
        \tl_set:Nn \l_@@_head_tl { \@@_inaccessible:w }
      }
    \@@_prev_input_silent:V \l_@@_head_tl
    \tl_set_eq:NN \l_@@_defined_tl \l_@@_head_tl
  }
%    \end{macrocode}
% \end{macro}
%
% ^^A todo: hash doubling is wrong
% \begin{macro}{\@@_scan_toks_to_str:}
%    \begin{macrocode}
\cs_new_protected:Npn \@@_scan_toks_to_str:
  {
    \@@_prev_input_gpush:
    \@@_scan_toks:NN \c_false_bool \c_true_bool
    \@@_prev_input_gpop:N \l_@@_tmpa_tl
    \@@_prev_input_silent:x
      { { \exp_after:wN \tl_to_str:n \l_@@_tmpa_tl } }
  }
%    \end{macrocode}
% \end{macro}
%
% \begin{macro}{\@@_scan_pdf_ext_toks:}
%    \begin{macrocode}
\cs_new_protected:Npn \@@_scan_pdf_ext_toks:
  {
    \@@_prev_input_gpush:
    \@@_scan_toks:NN \c_false_bool \c_true_bool
    \@@_prev_input_gpop:N \l_@@_tmpa_tl
    \@@_prev_input_silent:x
      { { \exp_not:N \exp_not:n \exp_not:V \l_@@_tmpa_tl } }
  }
%    \end{macrocode}
% \end{macro}
%
% ^^A todo: provide separate functions for defs (avoid hash doubling).
% \begin{macro}{\@@_scan_toks:NN}
%   The boolean |#1| is true if we are making a definition (then we
%   start by scanning the parameter text), false if we are simply
%   scanning a general text.  The boolean |#2| is true if we need to
%   expand, false otherwise (for instance for \tn{lowercase}).
%    \begin{macrocode}
\cs_new_protected:Npn \@@_scan_toks:NN #1#2
  {
    \bool_if:NT #1 { \@@_scan_param: }
    \@@_scan_left_brace:
    \bool_if:NTF #2
      { \@@_scan_group_x:N #1 }
      { \@@_scan_group_n:N #1 }
  }
%    \end{macrocode}
% \end{macro}
%
% \begin{macro}{\@@_scan_param:}
% \begin{macro}{\@@_scan_param_aux:}
%   Collect the parameter text into \cs{l_@@_tmpa_tl}, and when seeing
%   either a begin-group or an end-group character, put it back into the
%   input, stop looping, and put what we collected into
%   \cs{l_@@_defining_tl} and into the |prev_input|.
%    \begin{macrocode}
\cs_new_protected:Npn \@@_scan_param:
  {
    \tl_clear:N \l_@@_tmpa_tl
    \@@_scan_param_aux:
    \tl_put_right:NV \l_@@_defining_tl \l_@@_tmpa_tl
    \@@_prev_input_silent:V \l_@@_tmpa_tl
  }
\cs_new_protected:Npn \@@_scan_param_aux:
  {
    \@@_get_next:
    \tl_concat:NNN \l_@@_tmpa_tl
      \l_@@_tmpa_tl \l_@@_head_tl
    \tl_if_empty:NTF \l_@@_head_tl
      { \@@_back_input: } { \@@_scan_param_aux: }
  }
%    \end{macrocode}
% \end{macro}
% \end{macro}
%
% \begin{macro}{\@@_scan_group_n:N}
%   The boolean |#1| is true if we are making a definition, false
%   otherwise.  In both cases put the open brace back and grab the first
%   item.  The only difference is that when making a definition we store
%   the data into \cs{l_@@_defining_tl} as well.
%    \begin{macrocode}
\cs_new_protected:Npn \@@_scan_group_n:N #1
  {
    \gtl_set_eq:NN \l_@@_head_gtl \c_group_begin_gtl
    \@@_back_input:
    \@@_input_gpop_item:NF \l_@@_head_tl
      {
        \@@_error:nnnnn { runaway-text } { } { } { } { }
        \@@_exit:w
      }
    \tl_set:Nx \l_@@_head_tl { { \exp_not:V \l_@@_head_tl } }
    \bool_if:NT #1
      { \tl_put_right:NV \l_@@_defining_tl \l_@@_head_tl }
    \@@_prev_input_silent:V \l_@@_head_tl
  }
%    \end{macrocode}
% \end{macro}
%
% \begin{macro}{\@@_scan_group_x:N}
%   The boolean |#1| is true if we are making a definition, false
%   otherwise.
%    \begin{macrocode}
\cs_new_protected:Npn \@@_scan_group_x:N #1
  {
    \@@_input_gpop_tl:N \l_@@_head_tl
    \@@_back_input:V \l_@@_head_tl
    \bool_if:NTF #1
      {
        \@@_prev_input_silent:V \c_left_brace_str
        \tl_put_right:Nn \l_@@_defining_tl { { \if_false: } \fi: }
        \@@_scan_group_xdef:n { 1 }
      }
      {
        \@@_prev_input_gpush_gtl:
        \@@_prev_input_gtl:N \l_@@_head_gtl
        \@@_scan_group_x:n { 1 }
        \@@_prev_input_gpop_gtl:N \l_@@_tmpb_gtl
        \@@_prev_input_silent:x
          { \gtl_left_tl:N \l_@@_tmpb_gtl }
      }
  }
%    \end{macrocode}
% \end{macro}
%
% \begin{macro}{\@@_scan_group_xdef:n}
%   This is to scan the replacement text of an \tn{edef} or \tn{xdef}.
%   The integer |#1| counts the brace balance.
%    \begin{macrocode}
\cs_new_protected:Npn \@@_scan_group_xdef:n #1
  {
    \@@_get_token_xdef:
    \tl_if_empty:NTF \l_@@_head_tl
      {
        \gtl_if_head_is_group_begin:NTF \l_@@_head_gtl
          {
            \@@_prev_input_silent:V \c_left_brace_str
            \tl_put_right:Nn \l_@@_defining_tl { { \if_false: } \fi: }
            \@@_scan_group_xdef:f { \int_eval:n { #1 + 1 } }
          }
          {
            \@@_prev_input_silent:V \c_right_brace_str
            \tl_put_right:Nn \l_@@_defining_tl { \if_false: { \fi: } }
            \int_compare:nNnF {#1} = \c_one
              { \@@_scan_group_xdef:f { \int_eval:n { #1 - 1 } } }
          }
      }
      {
        \@@_prev_input_silent:V \l_@@_head_tl
        \tl_put_right:Nx \l_@@_defining_tl
          { \exp_not:N \exp_not:N \exp_not:V \l_@@_head_tl }
        \@@_scan_group_xdef:n {#1}
      }
  }
\cs_generate_variant:Nn \@@_scan_group_xdef:n { f }
%    \end{macrocode}
% \end{macro}
%
% \begin{macro}{\@@_scan_group_x:n}
%    \begin{macrocode}
\cs_new_protected:Npn \@@_scan_group_x:n #1
  {
    \@@_get_token_x:
    \@@_prev_input_gtl:N \l_@@_head_gtl
    \tl_if_empty:NTF \l_@@_head_tl
      {
        \gtl_if_head_is_group_begin:NTF \l_@@_head_gtl
          { \@@_scan_group_x:f { \int_eval:n { #1 + 1 } } }
          {
            \int_compare:nNnF {#1} = \c_one
              { \@@_scan_group_x:f { \int_eval:n { #1 - 1 } } }
          }
      }
      { \@@_scan_group_x:n {#1} }
  }
\cs_generate_variant:Nn \@@_scan_group_x:n { f }
%    \end{macrocode}
% \end{macro}
%
% ^^A todo: catch runaway expand/get_x_next. (get_next too, probably)
%
% \begin{macro}{\@@_scan_alt_rule:}
%    \begin{macrocode}
\cs_new_protected:Npn \@@_scan_alt_rule:
  {
    \@@_scan_keyword:nTF { wWiIdDtThH }
      {
        \@@_scan_normal_dimen:
        \@@_scan_alt_rule:
      }
      {
        \@@_scan_keyword:nTF { hHeEiIgGhHtT }
          {
            \@@_scan_normal_dimen:
            \@@_scan_alt_rule:
          }
          {
            \@@_scan_keyword:nT { dDeEpPtThH }
              {
                \@@_scan_normal_dimen:
                \@@_scan_alt_rule:
              }
          }
      }
  }
%    \end{macrocode}
% \end{macro}
%
% \begin{macro}{\@@_scan_spec:}
%   Some \TeX{} primitives accept the keywords |to| and |spread|,
%   followed by a dimension.
%    \begin{macrocode}
\cs_new_protected:Npn \@@_scan_spec:
  {
    \@@_scan_keyword:nTF { tToO } { \@@_scan_normal_dimen: }
      {
        \@@_scan_keyword:nT { sSpPrReEaAdD }
          { \@@_scan_normal_dimen: }
      }
    \@@_scan_left_brace:
  }
%    \end{macrocode}
% \end{macro}
%
% \subsection{Working with boxes}
%
% \begin{macro}{\@@_do_box:N}
%   When this procedure is called, the last item in
%   the previous-input sequence is
%   \begin{itemize}
%   \item empty if the box is meant to be put in the input stream,
%   \item \tn{setbox}\meta{int} if it is meant to be stored somewhere,
%   \item \tn{moveright}\meta{dim}, \tn{moveleft}\meta{dim},
%     \tn{lower}\meta{dim}, \tn{raise}\meta{dim} if it is meant to be
%     shifted,
%   \item \tn{leaders} or \tn{cleaders} or \tn{xleaders}, in which case
%     the argument is \cs{c_true_bool} (otherwise \cs{c_false_bool}).
%   \end{itemize}
%   If a |make_box| command follows, we fetch the operands.  If leaders
%   are followed by a rule, then this is also ok.  In all other cases,
%   call \cs{@@_do_box_error:} to clean up.
%    \begin{macrocode}
\cs_new_protected:Npn \@@_do_box:N #1
  {
    \@@_get_x_non_relax:
    \@@_set_cmd:
    \int_compare:nNnTF
      \l_@@_head_cmd_int = { \@@_tex_use:n { make_box } }
      { \@@_do_begin_box:N #1 }
      {
        \bool_if:NTF #1
          {
            \int_case:nnTF \l_@@_head_cmd_int
              {
                { \@@_tex_use:n { hrule } } { }
                { \@@_tex_use:n { vrule } } { }
              }
              { \@@_do_leaders_rule: }
              { \@@_do_box_error: }
          }
          { \@@_do_box_error: }
      }
  }
%    \end{macrocode}
% \end{macro}
%
% \begin{macro}{\@@_do_box_error:}
%   Put the (non-|make_box|) command back into the input and complain.
%   Then recover by throwing away the action (last item of
%   the previous-input sequence).  For some reason (this appears to be what
%   \TeX{} does), there is no need to remove the after assignment token
%   here.
%    \begin{macrocode}
\cs_new_protected:Npn \@@_do_box_error:
  {
    \@@_back_input:
    \@@_error:nnnnn { missing-box } { } { } { } { }
    \@@_prev_input_gpop:N \l_@@_head_tl
    \@@_print_action:x { \tl_to_str:N \l_@@_head_tl }
  }
%    \end{macrocode}
% \end{macro}
%
% \begin{macro}{\@@_do_begin_box:N}
%   We have just found a |make_box| command and placed it into the last
%   item of the previous-input sequence.  If it is ``simple''
%   (\tn{box}\meta{int}, \tn{copy}\meta{int}, \tn{lastbox},
%   \tn{vsplit}\meta{int} \texttt{to} \meta{dim}) then we grab its
%   operands, then call \cs{@@_do_simple_box:N} to finish up.  If it is
%   \tn{vtop} or \tn{vbox} or \tn{hbox}, we need to work harder.
%    \begin{macrocode}
\cs_new_protected:Npn \@@_do_begin_box:N #1
  {
    \@@_prev_input:V \l_@@_head_tl
    \int_case:nnTF \l_@@_head_char_int
      {
        { 0 } { \@@_scan_int: } % box
        { 1 } { \@@_scan_int: } % copy
        { 2 } { } % lastbox
        { 3 } % vsplit
          {
            \@@_scan_int:
            \@@_scan_to:
            \@@_scan_normal_dimen:
          }
      }
      { \@@_do_simple_box:N #1 }
      { \@@_do_box_explicit:N #1 }
  }
%    \end{macrocode}
% \end{macro}
%
% \begin{macro}{\@@_do_simple_box:N}
%   For leaders, we need to fetch a glue.  In all cases, retrieve the
%   box construction (such as |\raise3pt\vsplit7to5em|).  Finally, let
%   \TeX{} run the code and print what we have done.
%   In the case of \tn{shipout}, check that \tn{mag} has a value
%   between $1$ and $32768$.
%    \begin{macrocode}
\cs_new_protected:Npn \@@_do_simple_box:N #1
  {
    \bool_if:NTF #1 { \@@_do_leaders_fetch_skip: }
      {
        \@@_prev_input_gpop:N \l_@@_head_tl
        \tl_if_head_eq_meaning:VNT \l_@@_head_tl \tex_shipout:D
          { \@@_prepare_mag: }
        \tl_use:N \l_@@_head_tl \scan_stop:
        \gtl_gput_right:NV \g_@@_output_gtl \l_@@_head_tl
        \@@_print_action:x { \tl_to_str:N \l_@@_head_tl }
      }
  }
%    \end{macrocode}
% \end{macro}
%
% \begin{macro}{\@@_do_leaders_fetch_skip:}
%    \begin{macrocode}
\cs_new_protected:Npn \@@_do_leaders_fetch_skip:
  {
    \@@_get_x_non_relax:
    \@@_set_cmd:
    \int_compare:nNnTF \l_@@_head_cmd_int
      = { \@@_tex_use:n { \mode_if_vertical:TF { vskip } { hskip } } }
      {
        \@@_prev_input_gpop:N \l_@@_tmpa_tl
        \tl_put_left:NV \l_@@_head_tl \l_@@_tmpa_tl
        \@@_do_append_glue:
      }
      {
        \@@_back_input:
        \@@_error:nnnnn { improper-leaders } { } { } { } { }
        \@@_prev_input_gpop:N \l_@@_head_tl
        \@@_print_action:x { \tl_to_str:N \l_@@_head_tl }
      }
  }
%    \end{macrocode}
% \end{macro}
%
% ^^A todo: less intrusive hooking into \everyhbox/\everyvbox?
% \begin{macro}{\@@_do_box_explicit:N}
%   At this point, the last item in the previous-input sequence is
%   typically |\setbox0\hbox| or |\raise 3pt\hbox|.  Scan for keywords
%   |to| and |spread| and a left brace.  Install a hook in
%   \tn{everyhbox} or \tn{everyvbox} (whichever \TeX{} is going to
%   insert in the box).  We then retrieve all the material that led to
%   the current box into \cs{l_@@_head_tl} in order to print it, then
%   let \TeX{} perform the box operation (here we need to provide the
%   begin-group token, as it was scanned but not placed in
%   the previous-input sequence).  \TeX{} inserts \tn{everyhbox} or
%   \tn{everyvbox} just after the begin-group token, and the hook we did
%   is such that all that material is collected and put into the input
%   that we will study.
%   We must remember to find a glue for leaders, and for this
%   we use a stack of letters |v|, |h| for vertical/horizontal
%   leaders, and |Z| for normal boxes.
%    \begin{macrocode}
\cs_new_protected:Npn \@@_do_box_explicit:N #1
  {
    \token_if_eq_meaning:NNTF \l_@@_head_token \@@_hbox:w
      { \@@_box_hook:N \tex_everyhbox:D }
      { \@@_box_hook:N \tex_everyvbox:D }
      % ^^A todo: TeX calls |normal_paragraph| here.
    \@@_scan_spec:
    \@@_prev_input_gpop:N \l_@@_head_tl
    \@@_set_action_text:x
      { \tl_to_str:N \l_@@_head_tl \iow_char:N \{ }
    \seq_push:Nf \l_@@_leaders_box_seq
      { \bool_if:NTF #1 { \mode_if_vertical:TF { v } { h } } { Z } }
    \gtl_gput_right:NV \g_@@_output_gtl \l_@@_head_tl
    \gtl_gconcat:NNN \g_@@_output_gtl
      \g_@@_output_gtl \c_group_begin_gtl
    \tl_use:N \l_@@_head_tl
      \c_group_begin_token \@@_box_hook_end:
  }
%    \end{macrocode}
% \end{macro}
%
% \begin{macro}{\@@_box_hook:N}
% \begin{macro}{\@@_box_hook:w, \@@_box_hook_end:}
%   Used to capture the contents of an \tn{everyhbox} or similar,
%   without altering \tn{everyhbox} too much (just add one token at the
%   start).  The various \texttt{o}-expansions remove
%   \cs{prg_do_nothing:}, used to avoid losing braces.
%    \begin{macrocode}
\cs_new_protected:Npn \@@_box_hook:N #1
  {
    \tl_set:NV \l_@@_tmpa_tl #1
    \str_if_eq_x:nnF
      { \tl_head:N \l_@@_tmpa_tl } { \exp_not:N \@@_box_hook:w }
      {
        \@@_exp_args:Nx #1
          {
            \exp_not:n { \@@_box_hook:w \prg_do_nothing: }
            \exp_not:V #1
          }
      }
    \cs_gset_protected:Npn \@@_box_hook:w ##1 \@@_box_hook_end:
      {
        \exp_args:No #1 {##1}
        \cs_gset_eq:NN \@@_box_hook:w \prg_do_nothing:
        \gtl_clear:N \l_@@_after_group_gtl
        \@@_print_action:
        \@@_back_input:o {##1}
        \@@_set_action_text:x
          { \token_to_meaning:N #1 = \tl_to_str:o {##1} }
        \tl_if_empty:oF {##1} { \@@_print_action: }
      }
  }
\cs_new_eq:NN \@@_box_hook:w \prg_do_nothing:
\cs_new_eq:NN \@@_box_hook_end: \prg_do_nothing:
%    \end{macrocode}
% \end{macro}
% \end{macro}
%
% \begin{macro}{\@@_do_leaders_rule:}
%   After finding a |vrule| or |hrule| command and looking for |depth|,
%   |heigh| and |width| keywords, we are in the same situation as after
%   finding a box.  Fetch the required skip accordingly.
%    \begin{macrocode}
\cs_new_protected:Npn \@@_do_leaders_rule:
  {
    \@@_prev_input:V \l_@@_head_tl
    \@@_scan_alt_rule:
    \@@_do_leaders_fetch_skip:
  }
%    \end{macrocode}
% \end{macro}
%
% \subsection{Paragraphs}
%
% \begin{macro}[TF]{\@@_charcode_if_safe:n}
%    \begin{macrocode}
\prg_new_protected_conditional:Npnn \@@_charcode_if_safe:n #1 { TF }
  {
    \bool_if:nTF
      {
        \int_compare_p:n { #1 = `! }
        || \int_compare_p:n { `' <= #1 <= `[ }
        || \int_compare_p:n { #1 = `] }
        || \int_compare_p:n { ` ` <= #1 <= `z }
      }
      { \prg_return_true: }
      { \prg_return_false: }
  }
%    \end{macrocode}
% \end{macro}
%
% \begin{macro}{\@@_char:n, \@@_char:V, \@@_char:x}
%    \begin{macrocode}
\cs_new_protected:Npn \@@_char:n #1
  {
    \tex_char:D #1 \scan_stop:
    \@@_charcode_if_safe:nTF {#1}
      { \tl_set:Nx \l_@@_tmpa_tl { \char_generate:nn {#1} { 12 } } }
      {
        \tl_set:Nx \l_@@_tmpa_tl
          { \exp_not:N \char \int_eval:n {#1} ~ }
      }
    \gtl_gput_right:NV \g_@@_output_gtl \l_@@_tmpa_tl
    \@@_print_action:x { \tl_to_str:N \l_@@_tmpa_tl }
  }
\cs_generate_variant:Nn \@@_char:n { V }
\cs_new_protected:Npn \@@_char:x
  { \@@_exp_args:Nx \@@_char:n }
%    \end{macrocode}
% \end{macro}
%
% \begin{macro}
%   {\@@_char_in_mmode:n, \@@_char_in_mmode:V, \@@_char_in_mmode:x}
%    \begin{macrocode}
\cs_new_protected:Npn \@@_char_in_mmode:n #1
  {
    \int_compare:nNnTF { \tex_mathcode:D #1 } = { "8000 }
      { % math active
        \@@_exp_args:NNx \gtl_set:Nn \l_@@_head_gtl
          { \char_generate:nn {#1} { 12 } }
        \@@_back_input:
      }
      { \@@_char:n {#1} }
  }
\cs_generate_variant:Nn \@@_char_in_mmode:n { V }
\cs_new_protected:Npn \@@_char_in_mmode:x
  { \@@_exp_args:Nx \@@_char_in_mmode:n }
%    \end{macrocode}
% \end{macro}
%
% \begin{macro}{\@@_mathchar:n, \@@_mathchar:x}
%    \begin{macrocode}
\cs_new_protected:Npn \@@_mathchar:n #1
  {
    \tex_mathchar:D #1 \scan_stop:
    \tl_set:Nx \l_@@_tmpa_tl
      { \exp_not:N \mathchar \int_eval:n {#1} ~ }
    \gtl_gput_right:NV \g_@@_output_gtl \l_@@_tmpa_tl
    \@@_print_action:x { \tl_to_str:N \l_@@_tmpa_tl }
  }
\cs_new_protected:Npn \@@_mathchar:x
  { \@@_exp_args:Nx \@@_mathchar:n }
%    \end{macrocode}
% \end{macro}
%
% \begin{macro}{\@@_new_graf:N}
%   The argument is a boolean, indicating whether the paragraph should
%   be indented.  We have much less work to do here than \TeX{} itself.
%   Our only task is to correctly position the \tn{everypar} tokens in
%   the input that we will read, rather than letting \TeX{} run the code
%   right away.
%    \begin{macrocode}
\cs_new_protected:Npn \@@_new_graf:N #1
  {
    \tl_set:NV \l_@@_tmpa_tl \@@_everypar:w
    \@@_everypar:w { }
    \bool_if:NTF #1 { \tex_indent:D } { \tex_noindent:D }
    \exp_args:NV \@@_everypar:w \l_@@_tmpa_tl
    \@@_back_input:V \l_@@_tmpa_tl
    \@@_print_action:x
      {
        \g_@@_action_text_str \c_space_tl : ~
        \token_to_str:N \everypar = { \tl_to_str:N \l_@@_tmpa_tl }
      }
  }
%    \end{macrocode}
% \end{macro}
%
% \begin{macro}{\@@_end_graf:}
%    \begin{macrocode}
\cs_new_protected:Npn \@@_end_graf:
  { \mode_if_horizontal:T { \@@_normal_paragraph: } }
%    \end{macrocode}
% \end{macro}
%
% ^^A todo: Do we need to do more?
% \begin{macro}{\@@_normal_paragraph:}
%    \begin{macrocode}
\cs_new_protected:Npn \@@_normal_paragraph:
  {
    \tex_par:D
    \gtl_gput_right:Nn \g_@@_output_gtl { \par }
    \@@_print_action:x { Paragraph~end. }
  }
%    \end{macrocode}
% \end{macro}
%
% ^^A todo: do something to the \output routine?
% \begin{macro}{\@@_build_page:}
%    \begin{macrocode}
\cs_new_protected:Npn \@@_build_page:
  {
  }
%    \end{macrocode}
% \end{macro}
%
% \subsection{Groups}
%
% ^^A todo: implement \@@_end_align_group:
% ^^A todo: implement \@@_end_no_align_group: (start = \@@_end_graf:)
% ^^A todo: implement \@@_end_output_group:
% ^^A todo: implement \@@_end_disc_group:
% ^^A todo: implement \@@_end_math_choice_group:
%
% \begin{macro}{\@@_handle_right_brace:}
%   When an end-group character is sensed, the result depends on the
%   current group type.
%    \begin{macrocode}
\cs_new_protected:Npn \@@_handle_right_brace:
  {
    \int_compare:nTF { 1 <= \@@_currentgrouptype: <= 13 }
      {
        \gtl_gconcat:NNN \g_@@_output_gtl
          \g_@@_output_gtl \c_group_end_gtl
        \@@_back_input_gtl:N \l_@@_after_group_gtl
        \int_case:nn \@@_currentgrouptype:
          {
            { 1 } { \@@_end_simple_group: } % simple
            { 2 } { \@@_end_box_group: } % hbox
            { 3 } { \@@_end_box_group: } % adjusted_hbox
            { 4 } { \@@_end_graf: \@@_end_box_group: } % vbox
            { 5 } { \@@_end_graf: \@@_end_box_group: } % vtop
            { 6 } { \@@_end_align_group: } % align
            { 7 } { \@@_end_no_align_group: } % no_align
            { 8 } { \@@_end_output_group: } % output
            { 9 } { \@@_end_simple_group: } % math
            { 10 } { \@@_end_disc_group: } % disc
            { 11 } { \@@_end_graf: \@@_end_simple_group: } % insert
            { 12 } { \@@_end_graf: \@@_end_simple_group: } % vcenter
            { 13 } { \@@_end_math_choice_group: } % math_choice
          }
      }
      { % bottom_level, semi_simple, math_shift, math_left
        \l_@@_head_token
        \@@_print_action:
      }
  }
%    \end{macrocode}
% \end{macro}
%
% \begin{macro}{\@@_end_simple_group:}
%   This command is used to simply end a group, when there are no
%   specific operations to perform.
%    \begin{macrocode}
\cs_new_protected:Npn \@@_end_simple_group:
  {
    \l_@@_head_token
    \@@_print_action:
  }
%    \end{macrocode}
% \end{macro}
%
% \begin{macro}{\@@_end_box_group:}
%   The end of an explicit box (generated by \tn{vtop}, \tn{vbox}, or
%   \tn{hbox}) can either be simple, or can mean that we need to find a
%   skip for a \tn{leaders}/\tn{cleaders}/\tn{xleaders} construction.
%    \begin{macrocode}
\cs_new_protected:Npn \@@_end_box_group:
  {
    \seq_pop:NN \l_@@_leaders_box_seq \l_@@_tmpa_tl
    \exp_args:No \@@_end_box_group_aux:n { \l_@@_tmpa_tl }
  }
\cs_new_protected:Npn \@@_end_box_group_aux:n #1
  {
    \str_if_eq_x:nnTF {#1} { Z }
      { \@@_end_simple_group: }
      {
        \@@_get_x_non_relax:
        \@@_set_cmd:
        \int_compare:nNnTF \l_@@_head_cmd_int
          = { \@@_tex_use:n { #1 skip } }
          {
            \tl_put_left:Nn \l_@@_head_tl { \c_group_end_token }
            \@@_do_append_glue:
          }
          {
            \@@_back_input:
            \c_group_end_token \group_begin: \group_end:
            \@@_print_action:
          }
      }
  }
%    \end{macrocode}
% \end{macro}
%
% \begin{macro}{\@@_off_save:}
%    \begin{macrocode}
\cs_new_protected:Npn \@@_off_save:
  {
    \int_compare:nNnTF \@@_currentgrouptype: = { 0 }
      { % bottom-level
        \@@_error:nxxxx { extra-close }
          { \token_to_meaning:N \l_@@_head_token } { } { } { }
      }
      {
        \@@_back_input:
        \int_case:nnF \@@_currentgrouptype:
          {
            { 14 } % semi_simple_group
              { \gtl_set:Nn \l_@@_head_gtl { \group_end: } }
            { 15 } % math_shift_group
              { \gtl_set:Nn \l_@@_head_gtl { $ } } % $
            { 16 } % math_left_group
              { \gtl_set:Nn \l_@@_head_gtl { \tex_right:D . } }
          }
          { \gtl_set_eq:NN \l_@@_head_gtl \c_group_end_gtl }
        \@@_back_input:
        \@@_error:nxxxx { off-save }
          { \gtl_to_str:N \l_@@_head_gtl } { } { } { }
      }
  }
%    \end{macrocode}
% \end{macro}
%
% \subsection{Modes}
% ^^A todo: implement \aftergroup and things happening at end-group
% ^^A todo: implement \@@_scan_math:
%
% \begin{macro}
%   {\@@_mode_math:n, \@@_mode_non_math:n, \@@_mode_vertical:n}
%    \begin{macrocode}
\cs_new_protected:Npn \@@_mode_math:n #1
  { \mode_if_math:TF {#1} { \@@_insert_dollar_error: } }
\cs_new_protected:Npn \@@_mode_non_math:n #1
  { \mode_if_math:TF { \@@_insert_dollar_error: } {#1} }
\cs_new_protected:Npn \@@_mode_vertical:n #1
  {
    \mode_if_math:TF
      { \@@_insert_dollar_error: }
      { \mode_if_horizontal:TF { \@@_head_for_vmode: } {#1} }
  }
\cs_new_protected:Npn \@@_mode_non_vertical:n #1
  {
    \mode_if_vertical:TF
      { \@@_back_input: \@@_new_graf:N \c_true_bool }
      {#1}
  }
%    \end{macrocode}
% \end{macro}
%
% \begin{macro}{\@@_head_for_vmode:}
%   See \TeX{}'s |head_for_vmode|.
%    \begin{macrocode}
\cs_new_protected:Npn \@@_head_for_vmode:
  {
    \mode_if_inner:TF
      {
        \token_if_eq_meaning:NNTF \l_@@_head_token \tex_hrule:D
          {
            \@@_error:nnnnn { hrule-bad-mode } { } { } { } { }
            \@@_print_action:
          }
          { \@@_off_save: }
      }
      {
        \@@_back_input:
        \gtl_set:Nn \l_@@_head_gtl { \par }
        \@@_back_input:
      }
  }
%    \end{macrocode}
% \end{macro}
%
% ^^A todo: should push_math be provided?
% \begin{macro}{\@@_goto_inner_math:}
%    \begin{macrocode}
\cs_new_protected:Npn \@@_goto_inner_math:
  {
    \@@_box_hook:N \tex_everymath:D
    $ % $
    \@@_box_hook_end:
  }
%    \end{macrocode}
% \end{macro}
%
% \begin{macro}{\@@_goto_display_math:}
%    \begin{macrocode}
\cs_new_protected:Npn \@@_goto_display_math:
  {
    \@@_box_hook:N \tex_everydisplay:D
    $ $
    \@@_box_hook_end:
  }
%    \end{macrocode}
% \end{macro}
%
% \begin{macro}{\@@_after_math:}
%    \begin{macrocode}
\cs_new_protected:Npn \@@_after_math:
  {
    \mode_if_inner:TF
      {
        \gtl_gput_right:NV \g_@@_output_gtl \l_@@_head_tl
        \@@_back_input_gtl:N \l_@@_after_group_gtl
        $ % $
      }
      {
        \gtl_gput_right:NV \g_@@_output_gtl \l_@@_head_tl
        \@@_get_x_next:
        \token_if_eq_catcode:NNF
          \l_@@_head_token \c_math_toggle_token
          {
            \@@_back_input:
            \tl_set:Nn \l_@@_head_tl { $ } % $
            \@@_error:nnnnn { missing-dollar } { } { } { } { }
          }
        \gtl_gput_right:NV \g_@@_output_gtl \l_@@_head_tl
        \@@_back_input_gtl:N \l_@@_after_group_gtl
        $ $
      }
    \@@_print_action:
  }
%    \end{macrocode}
% \end{macro}
%
% \subsection{One step}
%
% \begin{macro}{\@@_do_step:}
%   Perform the action if the corresponding command exists.  If that
%   command does not exist, complain, and leave the token in the output.
%    \begin{macrocode}
\cs_new_protected:Npn \@@_do_step:
  {
    \@@_set_action_text:
    \bool_if:NT \g_@@_internal_debug_bool
      { \@@_exp_args:Nx \iow_term:n { Cmd:~\int_to_arabic:n { \l_@@_head_cmd_int } } }
    \cs_if_exist_use:cF
      { @@_cmd_ \int_use:N \l_@@_head_cmd_int : }
      { \@@_error:nxxxx { internal } { unknown-command } { } { } { } }
  }
%    \end{macrocode}
% \end{macro}
%
% \subsection{Commands}
%
% We will implement commands in order of their command codes (some of
% the more elaborate commands call auxiliaries defined in other
% sections).
%
% \subsubsection{Characters: from 0 to 15}
%
% This section is about command codes in the range $[0, 15]$.
% \begin{itemize}
% \item |relax=0| for \tn{relax}.
% \item |begin-group_char=1| for begin-group characters (catcode~$1$).
% \item |end-group_char=2| for end-group characters (catcode~$2$).
% \item |math_char=3| for math shift (math toggle in \pkg{expl3})
%   characters (catcode~$3$).
% \item |tab_mark=4| for \tn{span}
% \item |alignment_char=4| for alignment tab characters (catcode~$4$).
% \item |car_ret=5| for \tn{cr} and \tn{crcr}.
% \item |macro_char=6| for macro parameter characters (catcode~$6$).
% \item |superscript_char=7| for superscript characters (catcode~$7$).
% \item |subscript_char=8| for subscript characters (catcode~$8$).
% \item |endv=9| for |?|.
% \item |blank_char=10| for blank spaces (catcode~$10$).
% \item |the_char=11| for letters (catcode~$11$).
% \item |other_char=12| for other characters (catcode~$12$).
% \item |par_end=13| for \tn{par}.
% \item |stop=14| for \tn{end} and \tn{dump}.
% \item |delim_num=15| for \tn{delimiter}.
% \end{itemize}
% Not implemented at all: |endv|.
%
% \tn{relax} does nothing.
%    \begin{macrocode}
\@@_new_tex_cmd:nn { relax }                                % 0
  {
    \token_if_eq_meaning:NNT \l_@@_head_token \@@_special_relax:
      {
        \exp_after:wN \@@_token_if_expandable:NTF \l_@@_head_tl
          {
            \@@_set_action_text:x
              { \iow_char:N \\notexpanded: \g_@@_action_text_str }
          }
          { }
      }
    \@@_print_action:
  }
%    \end{macrocode}
%
% Begin-group characters are sent to the output, as their grouping
% behaviour may affect the scope of font changes, for instance.  They are
% also performed.
%    \begin{macrocode}
\@@_new_tex_cmd:nn { begin-group_char }                     % 1
  {
    \gtl_gconcat:NNN \g_@@_output_gtl
      \g_@@_output_gtl \c_group_begin_gtl
    \@@_print_action:
    \l_@@_head_token
    \gtl_clear:N \l_@@_after_group_gtl
  }
%    \end{macrocode}
%
%    \begin{macrocode}
\@@_new_tex_cmd:nn { end-group_char }                       % 2
  { \@@_handle_right_brace: }
%    \end{macrocode}
%
% Math shift characters quit vertical mode, and start math mode.
%    \begin{macrocode}
\@@_new_tex_cmd:nn { math_char }                            % 3
  {
    \@@_mode_non_vertical:n
      {
        \mode_if_math:TF
          {
            \int_compare:nNnTF
              \@@_currentgrouptype: = { 15 } % math_shift_group
              { \@@_after_math: }
              { \@@_off_save: }
          }
          {
            \gtl_gput_right:NV \g_@@_output_gtl \l_@@_head_tl
            \@@_get_next:
            \token_if_eq_catcode:NNTF
              \l_@@_head_token \c_math_toggle_token
              {
                \mode_if_inner:TF
                  { \@@_back_input: \@@_goto_inner_math: }
                  {
                    \gtl_gput_right:NV
                      \g_@@_output_gtl \l_@@_head_tl
                    \@@_goto_display_math:
                  }
              }
              { \@@_back_input: \@@_goto_inner_math: }
          }
      }
  }
%    \end{macrocode}
%
% Some commands are errors when they reach \TeX{}'s stomach.  Among
% others, |tab_mark=alignment_char|, |car_ret| and |macro_char|.  We let
% \TeX{} insert the proper error.
%    \begin{macrocode}
\@@_new_tex_cmd:nn { alignment_char }                       % 4
  { \l_@@_head_token \@@_print_action: }
\@@_new_tex_cmd:nn { car_ret }                              % 5
  { \l_@@_head_token \@@_print_action: }
\@@_new_tex_cmd:nn { macro_char }                           % 6
  { \l_@@_head_token \@@_print_action: }
%    \end{macrocode}
%
% ^^A todo: wrong: needs to collect subsup before output (otherwise b{}^2{}_3)
% ^^A todo: wrong: mathcode "8000 is not recognized.
%    \begin{macrocode}
\@@_new_tex_cmd:nn { superscript_char }                     % 7
  { \@@_mode_math:n { \@@_sub_sup: } }
\@@_new_tex_cmd:nn { subscript_char }                       % 8
  { \@@_mode_math:n { \@@_sub_sup: } }
\cs_new_protected:Npn \@@_sub_sup:
  {
    \@@_prev_input_gpush:N \l_@@_head_tl
    \@@_print_action:
    \@@_get_x_non_relax:
    \@@_set_cmd:
    \int_case:nnTF \l_@@_head_cmd_int
      {
        { \@@_tex_use:n { the_char } }
          { \@@_prev_input:V \l_@@_head_tl }
        { \@@_tex_use:n { other_char } }
          { \@@_prev_input:V \l_@@_head_tl }
        { \@@_tex_use:n { char_given } }
          { \@@_prev_input:V \l_@@_head_tl }
        { \@@_tex_use:n { char_num } }
          {
            \@@_prev_input:V \l_@@_head_tl
            \@@_scan_int:
          }
        { \@@_tex_use:n { math_char_num } }
          {
            \@@_prev_input:V \l_@@_head_tl
            \@@_scan_int:
          }
        { \@@_tex_use:n { math_given } }
          { \@@_prev_input:V \l_@@_head_tl }
        { \@@_tex_use:n { delim_num } }
          { \@@_prev_input:V \l_@@_head_tl \@@_scan_int: }
      }
      {
        \@@_prev_input_gpop:N \l_@@_head_tl
        \gtl_gput_right:NV \g_@@_output_gtl \l_@@_head_tl
        \tl_use:N \l_@@_head_tl \scan_stop:
      }
      {
        \@@_back_input:
        \@@_scan_left_brace:
        \@@_prev_input_gpop:N \l_@@_head_tl
        \gtl_gput_right:NV \g_@@_output_gtl \l_@@_head_tl
        \gtl_gconcat:NNN \g_@@_output_gtl
          \g_@@_output_gtl \c_group_begin_gtl
        \tl_use:N \l_@@_head_tl \c_group_begin_token
      }
    \@@_print_action:x { \tl_to_str:N \l_@@_head_tl }
  }
%    \end{macrocode}
%
%    \begin{macrocode}
\@@_new_tex_cmd:nn { endv }                                 % 9
  { \@@_not_implemented:n { alignments } }
%    \end{macrocode}
%
% Blank spaces are ignored in vertical and math modes in the same way as
% \tn{relax} is in all modes.  In horizontal mode, add them to the
% output.
%    \begin{macrocode}
\@@_new_tex_cmd:nn { blank_char }                           % 10
  {
    \mode_if_horizontal:T
      {
        \gtl_gput_right:Nn \g_@@_output_gtl { ~ }
        \l_@@_head_token
      }
    \@@_print_action:
  }
%    \end{macrocode}
%
% Letters and other characters leave vertical mode.
%    \begin{macrocode}
\@@_new_tex_cmd:nn { the_char }                             % 11
  {
    \@@_mode_non_vertical:n
      {
        \tl_set:Nx \l_@@_tmpa_tl
          { ` \@@_token_to_char:N \l_@@_head_token }
        \mode_if_math:TF
          { \@@_char_in_mmode:V \l_@@_tmpa_tl }
          { \@@_char:V \l_@@_tmpa_tl }
      }
  }
\@@_new_eq_tex_cmd:nn { other_char } { the_char }           % 12
%    \end{macrocode}
%
%    \begin{macrocode}
\@@_new_tex_cmd:nn { par_end }                              % 13
  {
    \@@_mode_non_math:n
      {
        \mode_if_vertical:TF
          { \@@_normal_paragraph: }
          {
            % if align_state<0 then off_save;
            \@@_end_graf:
            \mode_if_vertical:T
              { \mode_if_inner:F { \@@_build_page: } }
          }
      }
  }
%    \end{macrocode}
%
% ^^A todo: \dump only at bottom-level (\currentgrouptype=0)
%    \begin{macrocode}
\@@_new_tex_cmd:nn { stop }                                 % 14
  {
    \@@_mode_vertical:n
      {
        \mode_if_inner:TF
          { \@@_forbidden_case: }
          {
            % ^^A todo: unless its_all_over
            \int_gdecr:N \g_@@_ends_int
            \int_compare:nNnTF \g_@@_ends_int > \c_zero
              {
                \@@_back_input:
                \@@_back_input:n
                  {
                    \@@_hbox:w to \tex_hsize:D { }
                    \tex_vfill:D
                    \tex_penalty:D - '10000000000 ~
                  }
                \@@_build_page:
                \@@_print_action:x { End~everything! }
              }
              {
                \@@_print_outcome:
                \l_@@_head_token
              }
          }
      }
  }
%    \end{macrocode}
%
%    \begin{macrocode}
\@@_new_tex_cmd:nn { delim_num }                            % 15
  {
    \@@_mode_math:n
      {
        \@@_prev_input_gpush:N \l_@@_head_tl
        \@@_print_action:
        \@@_scan_int:
        \@@_prev_input_gpop:N \l_@@_head_tl
        \tl_use:N \l_@@_head_tl \scan_stop:
        \@@_print_action:x { \tl_to_str:N \l_@@_head_tl }
      }
  }
%    \end{macrocode}
%
% \subsubsection{Boxes: from 16 to 31}
%
% \begin{itemize}
% \item |char_num=16| for \tn{char}
% \item |math_char_num=17| for \tn{mathchar}
% \item |mark=18| for \tn{mark} and \tn{marks}
% \item |xray=19| for \tn{show}, \tn{showbox}, \tn{showthe},
%   \tn{showlists}, \tn{showgroups}, \tn{showtokens}, \tn{showifs}.
% \item |make_box=20| for \tn{box}, \tn{copy}, \tn{lastbox},
%   \tn{vsplit}, \tn{vtop}, \tn{vbox}, and \tn{hbox}~(106).
% \item |hmove=21| for \tn{moveright} and \tn{moveleft}.
% \item |vmove=22| for \tn{lower} and \tn{raise}.
% \item |un_hbox=23| for \tn{unhbox} and \tn{unhcopy}.
% \item |unvbox=24| for \tn{unvbox}, \tn{unvcopy}, \tn{pagediscards},
%   and \tn{splitdiscards}.
% \item |remove_item=25| for \tn{unpenalty}~(12), \tn{unkern}~(11),
%   \tn{unskip}~(10).
% \item |hskip=26| for \tn{hfil}, \tn{hfill}, \tn{hss}, \tn{hfilneg},
%   \tn{hskip}.
% \item |vskip=27| for \tn{vfil}, \tn{vfill}, \tn{vss}, \tn{vfilneg},
%   \tn{vskip}.
% \item |mskip=28| for \tn{mskip}~(5).
% \item |kern=29| for \tn{kern}~(1).
% \item |mkern=30| for \tn{mkern}~(99).
% \item |leader_ship=31| for \tn{shipout}~(99), \tn{leaders}~(100),
%   \tn{cleaders}~(101), \tn{xleaders}~(102).
% \end{itemize}
%
% \tn{char} leaves vertical mode, then scans an integer operand, then
% calls \cs{@@_char_in_mmode:n} or \cs{@@_char:n} depending on the mode.
% See implementation of |the_char| and |other_char|.
%    \begin{macrocode}
\@@_new_tex_cmd:nn { char_num }                             % 16
  {
    \@@_mode_non_vertical:n
      {
        \@@_prev_input_gpush:N \l_@@_head_tl
        \@@_print_action:
        \@@_scan_int:
        \@@_prev_input_gpop:N \l_@@_head_tl
        \mode_if_math:TF
          { \@@_char_in_mmode:x { \tl_tail:N \l_@@_head_tl } }
          { \@@_char:x { \tl_tail:N \l_@@_head_tl } }
      }
  }
%    \end{macrocode}
%
% Only allowed in math mode, \tn{mathchar} reads an integer operand, and
% calls \cs{@@_mathchar:n}, which places the corresponding math
% character in the \cs{g_@@_output_gtl}, and in the actual output.
%    \begin{macrocode}
\@@_new_tex_cmd:nn { math_char_num }                        % 17
  {
    \@@_mode_math:n
      {
        \@@_prev_input_gpush:N \l_@@_head_tl
        \@@_print_action:
        \@@_scan_int:
        \@@_prev_input_gpop:N \l_@@_head_tl
        \@@_mathchar:x { \tl_tail:N \l_@@_head_tl }
      }
  }
%    \end{macrocode}
%
%    \begin{macrocode}
\@@_new_tex_cmd:nn { mark }                                 % 18
  {
    \@@_prev_input_gpush:N \l_@@_head_tl
    \@@_print_action:
    \int_compare:nNnF \l_@@_head_char_int = \c_zero
      { \@@_scan_int: }
    \@@_prev_input_gpush:
    \@@_scan_toks:NN \c_false_bool \c_true_bool
    \@@_prev_input_gpop:N \l_@@_tmpa_tl
    \@@_prev_input_gpop:N \l_@@_head_tl
    \@@_print_action:x
      { \tl_to_str:N \l_@@_head_tl \tl_to_str:N \l_@@_tmpa_tl }
    \tl_put_right:Nx \l_@@_head_tl
      { { \exp_not:N \exp_not:n \exp_not:V \l_@@_tmpa_tl } }
    \tl_use:N \l_@@_head_tl
  }
%    \end{macrocode}
%
% We now implement the primitives \tn{show}, \tn{showbox}, \tn{showthe},
% \tn{showlists}, \tn{showgroups}, \tn{showtokens} and \tn{showifs}.
% Those with no operand are sent to \TeX{} after printing the action.
% Those with operands print first, then scan their operands, then are
% sent to \TeX{}.  The case of \tn{show} is a bit special, as its
% operand is a single token, which cannot easily be put into the
% the previous-input sequence in general.  Since no expansion can occur,
% simply grab the token and show it.
%    \begin{macrocode}
\@@_new_tex_cmd:nn { xray }                                 % 19
  {
    \@@_prev_input_gpush:N \l_@@_head_tl
    \@@_print_action:
    \int_case:nnF \l_@@_head_char_int
      {
        { 0 }
          { % show
            \@@_get_next:
            \@@_prev_input_gpop:N \l_@@_tmpa_tl
            \token_if_eq_meaning:NNTF
              \l_@@_head_token \@@_special_relax:
              {
                \exp_after:wN \exp_after:wN \exp_after:wN \l_@@_tmpa_tl
                \exp_after:wN \exp_not:N \l_@@_head_tl
              }
              { \gtl_head_do:NN \l_@@_head_gtl \l_@@_tmpa_tl }
          }
        { 2 }
          { % showthe
            \@@_get_x_next:
            \@@_scan_something_internal:n { 5 }
            \@@_prev_input_gpop:N \l_@@_head_tl
            \@@_exp_args:Nx \use:n
              { \etex_showtokens:D { \tl_tail:N \l_@@_head_tl } }
          }
      }
      { % no operand for showlists, showgroups, showifs
        \int_compare:nNnT \l_@@_head_char_int = \c_one % showbox
          { \@@_scan_int: }
        \int_compare:nNnT \l_@@_head_char_int = \c_five % showtokens
          { \@@_scan_toks:NN \c_false_bool \c_false_bool }
        \@@_prev_input_gpop:N \l_@@_head_tl
        \tl_use:N \l_@@_head_tl \scan_stop:
      }
  }
%    \end{macrocode}
%
% |make_box=20| for \tn{box}, \tn{copy}, \tn{lastbox},
%   \tn{vsplit}, \tn{vtop}, \tn{vbox}, and \tn{hbox}~(106).
%    \begin{macrocode}
\@@_new_tex_cmd:nn { make_box }                             % 20
  {
    \@@_prev_input_gpush:
    \@@_back_input:
    \@@_do_box:N \c_false_bool
  }
%    \end{macrocode}
%
% ^^A todo: implement \@@_forbidden_case:
%
% \begin{macro}{\@@_do_move:}
%   Scan a dimension and a box, and perform the shift, printing the
%   appropriate action.
%    \begin{macrocode}
\cs_new_protected:Npn \@@_do_move:
  {
    \@@_prev_input_gpush:N \l_@@_head_tl
    \@@_print_action:
    \@@_scan_normal_dimen:
    \@@_do_box:N \c_false_bool
  }
%    \end{macrocode}
% \end{macro}
%
% |hmove=21| for \tn{moveright} and \tn{moveleft}.
%    \begin{macrocode}
\@@_new_tex_cmd:nn { hmove }                                % 21
  {
    \mode_if_vertical:TF
      { \@@_do_move: } { \@@_forbidden_case: }
  }
%    \end{macrocode}
%
% |vmove=22| for \tn{lower} and \tn{raise}.
%    \begin{macrocode}
\@@_new_tex_cmd:nn { vmove }                                % 22
  {
    \mode_if_vertical:TF
      { \@@_forbidden_case: } { \@@_do_move: }
  }
%    \end{macrocode}
%
% \begin{macro}{\@@_do_unpackage:}
%    \begin{macrocode}
\cs_new_protected:Npn \@@_do_unpackage:
  {
    \@@_prev_input_gpush:N \l_@@_head_tl
    \@@_print_action:
    \@@_scan_int:
    \@@_prev_input_gpop:N \l_@@_head_tl
    \tl_use:N \l_@@_head_tl \scan_stop:
    \@@_print_action:x { \tl_to_str:N \l_@@_head_tl }
  }
%    \end{macrocode}
% \end{macro}
%
% |un_hbox=23| for \tn{unhbox} and \tn{unhcopy}.
%    \begin{macrocode}
\@@_new_tex_cmd:nn { un_hbox }                              % 23
  { \@@_mode_non_vertical:n { \@@_do_unpackage: } }
%    \end{macrocode}
%
% |unvbox=24| for \tn{unvbox}, \tn{unvcopy}, \tn{pagediscards}, and
% \tn{splitdiscards}.  The latter two take no operands, so we just let
% \TeX{} do its thing, then we show the action.
%    \begin{macrocode}
\@@_new_tex_cmd:nn { un_vbox }                              % 24
  {
    \@@_mode_vertical:n
      {
        \int_compare:nNnTF \l_@@_head_char_int > { 1 }
          { \l_@@_head_token \@@_print_action: }
          { \@@_do_unpackage: }
      }
  }
%    \end{macrocode}
%
% |remove_item=25| for \tn{unpenalty}~(12), \tn{unkern}~(11),
% \tn{unskip}~(10).  Those commands only act on \TeX{}'s box/glue data
% structures, which \pkg{unravel} does not (and cannot) care about.
%    \begin{macrocode}
\@@_new_tex_cmd:nn { remove_item }                          % 25
  { \l_@@_head_token \@@_print_action: }
%    \end{macrocode}
%
% \begin{macro}{\@@_do_append_glue:}
%   For \tn{hfil}, \tn{hfill}, \tn{hss}, \tn{hfilneg} and their vertical
%   analogs, simply call the primitive then print the action.  For
%   \tn{hskip}, \tn{vskip} and \tn{mskip}, read a normal glue or a mu
%   glue (\cs{l_@@_head_char_int} is $4$ or~$5$), then call the
%   primitive with that operand, and print the whole thing as an action.
%    \begin{macrocode}
\cs_new_protected:Npn \@@_do_append_glue:
  {
    \int_compare:nNnTF \l_@@_head_char_int < { 4 }
      { \tl_use:N \l_@@_head_tl \@@_print_action: }
      {
        \@@_prev_input_gpush:N \l_@@_head_tl
        \@@_print_action:
        \exp_args:Nf \@@_scan_glue:n
          { \int_eval:n { \l_@@_head_char_int - 2 } }
        \@@_prev_input_gpop:N \l_@@_head_tl
        \tl_use:N \l_@@_head_tl \scan_stop:
        \@@_print_action:x { \tl_to_str:N \l_@@_head_tl }
      }
  }
%    \end{macrocode}
% \end{macro}
%
% |hskip=26| for \tn{hfil}, \tn{hfill}, \tn{hss}, \tn{hfilneg},
% \tn{hskip}.
%    \begin{macrocode}
\@@_new_tex_cmd:nn { hskip }                                % 26
  { \@@_mode_non_vertical:n { \@@_do_append_glue: } }
%    \end{macrocode}
%
% |vskip=27| for \tn{vfil}, \tn{vfill}, \tn{vss}, \tn{vfilneg},
% \tn{vskip}.
%    \begin{macrocode}
\@@_new_tex_cmd:nn { vskip }                                % 27
  { \@@_mode_vertical:n { \@@_do_append_glue: } }
%    \end{macrocode}
%
% |mskip=28| for \tn{mskip}~(5).
%    \begin{macrocode}
\@@_new_tex_cmd:nn { mskip }                                % 28
  { \@@_mode_math:n { \@@_do_append_glue: } }
%    \end{macrocode}
%
% \begin{macro}{\@@_do_append_kern:}
%   See \cs{@@_do_append_glue:}.  This function is used for the
%   primitives \tn{kern} and \tn{mkern} only.
%    \begin{macrocode}
\cs_new_protected:Npn \@@_do_append_kern:
  {
    \@@_prev_input_gpush:N \l_@@_head_tl
    \@@_print_action:
    \token_if_eq_meaning:NNTF \l_@@_head_token \tex_kern:D
      { \@@_scan_dimen:nN { 2 } \c_false_bool }
      { \@@_scan_dimen:nN { 3 } \c_false_bool }
    \@@_prev_input_gpop:N \l_@@_head_tl
    \tl_use:N \l_@@_head_tl \scan_stop:
    \@@_print_action:x { \tl_to_str:N \l_@@_head_tl }
  }
%    \end{macrocode}
% \end{macro}
%
% |kern=29| for \tn{kern}~(1).
%    \begin{macrocode}
\@@_new_tex_cmd:nn { kern }                                 % 29
  { \@@_do_append_kern: }
%    \end{macrocode}
%
% |mkern=30| for \tn{mkern}~(99).
%    \begin{macrocode}
\@@_new_tex_cmd:nn { mkern }                                % 30
  { \@@_mode_math:n { \@@_do_append_kern: } }
%    \end{macrocode}
%
% |leader_ship=31| for \tn{shipout}~(99), \tn{leaders}~(100),
% \tn{cleaders}~(101), \tn{xleaders}~(102).
%    \begin{macrocode}
\@@_new_tex_cmd:nn { leader_ship }                          % 31
  {
    \@@_prev_input_gpush:N \l_@@_head_tl
    \@@_print_action:
    \@@_do_box:N \c_true_bool
  }
%    \end{macrocode}
%
% \subsubsection{From 32 to 47}
%
% ^^A todo: implement \@@_begin_insert_or_adjust:
%
% \begin{itemize}
% \item |halign=32|
% \item |valign=33|
% \item |no_align=34|
% \item |vrule=35|
% \item |hrule=36|
% \item |insert=37|
% \item |vadjust=38|
% \item |ignore_spaces=39|
% \item |after_assignment=40|
% \item |after_group=41|
% \item |break_penalty=42|
% \item |start_par=43|
% \item |ital_corr=44|
% \item |accent=45|
% \item |math_accent=46|
% \item |discretionary=47|
% \end{itemize}
%
%    \begin{macrocode}
\@@_new_tex_cmd:nn { halign }                               % 32
  { \@@_not_implemented:n { halign } }
\@@_new_tex_cmd:nn { valign }                               % 33
  { \@@_not_implemented:n { valign } }
\@@_new_tex_cmd:nn { no_align }                             % 34
  { \@@_not_implemented:n { noalign } }
%    \end{macrocode}
%
%    \begin{macrocode}
\@@_new_tex_cmd:nn { vrule }                                % 35
  { \@@_mode_non_vertical:n { \@@_do_rule: } }
\@@_new_tex_cmd:nn { hrule }                                % 36
  { \@@_mode_vertical:n { \@@_do_rule: } }
\cs_new_protected:Npn \@@_do_rule:
  {
    \@@_prev_input_gpush:N \l_@@_head_tl
    \@@_print_action:
    \@@_scan_alt_rule:
    \@@_prev_input_gpop:N \l_@@_head_tl
    \tl_use:N \l_@@_head_tl \scan_stop:
    \@@_print_action:x { \tl_to_str:N \l_@@_head_tl }
  }
%    \end{macrocode}
%
%    \begin{macrocode}
\@@_new_tex_cmd:nn { insert }                               % 37
  { \@@_begin_insert_or_adjust: }
\@@_new_tex_cmd:nn { vadjust }                              % 38
  {
    \mode_if_vertical:TF
      { \@@_forbidden_case: } { \@@_begin_insert_or_adjust: }
  }
%    \end{macrocode}
%
% ^^A todo: implement the internal \pdfprimitive business.
%    \begin{macrocode}
\@@_new_tex_cmd:nn { ignore_spaces }                        % 39
  {
    \token_if_eq_meaning:NNTF \l_@@_head_token \tex_ignorespaces:D
      {
        \@@_print_action:
        \@@_get_x_non_blank:
        \@@_set_cmd:
        \@@_do_step:
      }
      { \@@_not_implemented:n { pdfprimitive } }
  }
%    \end{macrocode}
%
%    \begin{macrocode}
\@@_new_tex_cmd:nn { after_assignment }                     % 40
  {
    \tl_set_eq:NN \l_@@_tmpa_tl \l_@@_head_tl
    \@@_get_next:
    \gtl_gset_eq:NN \g_@@_after_assignment_gtl \l_@@_head_gtl
    \@@_print_action:x
      {
        Afterassignment:~\tl_to_str:N \l_@@_tmpa_tl
        \gtl_to_str:N \l_@@_head_gtl
      }
  }
%    \end{macrocode}
%
% Save the next token at the end of \cs{l_@@_after_group_gtl}, unless
% we are at the bottom group level, in which case, the token is ignored
% completely.
%    \begin{macrocode}
\@@_new_tex_cmd:nn { after_group }                          % 41
  {
    \tl_set_eq:NN \l_@@_tmpa_tl \l_@@_head_tl
    \@@_get_next:
    \int_compare:nNnTF \@@_currentgrouptype: = \c_zero
      {
        \@@_print_action:x
          {
            Aftergroup~(level~0~=>~dropped):~
            \tl_to_str:N \l_@@_tmpa_tl
            \gtl_to_str:N \l_@@_head_gtl
          }
      }
      {
        \gtl_concat:NNN \l_@@_after_group_gtl
          \l_@@_after_group_gtl \l_@@_head_gtl
        \@@_print_action:x
          {
            Aftergroup:~\tl_to_str:N \l_@@_tmpa_tl
            \gtl_to_str:N \l_@@_head_gtl
          }
      }
  }
%    \end{macrocode}
%
% See \cs{@@_do_append_glue:}.
%    \begin{macrocode}
\@@_new_tex_cmd:nn { break_penalty }                        % 42
  {
    \@@_prev_input_gpush:N \l_@@_head_tl
    \@@_print_action:
    \@@_scan_int:
    \@@_prev_input_gpop:N \l_@@_head_tl
    \tl_use:N \l_@@_head_tl \scan_stop:
    \@@_print_action:x { \tl_to_str:N \l_@@_head_tl }
  }
%    \end{macrocode}
%
%    \begin{macrocode}
\@@_new_tex_cmd:nn { start_par }                            % 43
  {
    \mode_if_vertical:TF
      {
        \token_if_eq_meaning:NNTF \l_@@_head_token \tex_noindent:D
          { \@@_new_graf:N \c_false_bool }
          { \@@_new_graf:N \c_true_bool }
      }
      {
        \int_compare:nNnT \l_@@_head_char_int = { 1 } % indent
          {
            \@@_hbox:w width \tex_parindent:D { }
            \gtl_gput_right:NV \g_@@_output_gtl \l_@@_head_tl
          }
        \@@_print_action:
      }
  }
%    \end{macrocode}
%
%    \begin{macrocode}
\@@_new_tex_cmd:nn { ital_corr }                            % 44
  {
    \mode_if_vertical:TF { \@@_forbidden_case: }
      { \l_@@_head_token \@@_print_action: }
  }
%    \end{macrocode}
%
% \begin{macro}{\@@_do_accent:}
%    \begin{macrocode}
\cs_new_protected:Npn \@@_do_accent:
  {
    \@@_prev_input_gpush:N \l_@@_head_tl
    \@@_print_action:
    \@@_scan_int:
    \@@_do_assignments:
    \bool_if:nTF
      {
        \token_if_eq_catcode_p:NN
          \l_@@_head_token \c_catcode_letter_token
        ||
        \token_if_eq_catcode_p:NN
          \l_@@_head_token \c_catcode_other_token
        ||
        \int_compare_p:nNn
          \l_@@_head_cmd_int = { \@@_tex_use:n { char_given } }
      }
      { \@@_prev_input:V \l_@@_head_tl }
      {
        \token_if_eq_meaning:NNTF \l_@@_head_token \tex_char:D
          {
            \@@_prev_input:V \l_@@_head_tl
            \@@_scan_int:
          }
          { \@@_break:w }
      }
    \@@_prev_input_gpop:N \l_@@_head_tl
    \gtl_gput_right:NV \g_@@_output_gtl \l_@@_head_tl
    \tl_use:N \l_@@_head_tl \scan_stop:
    \@@_print_action:x { \tl_to_str:N \l_@@_head_tl }
    \@@_break_point:
  }
%    \end{macrocode}
% \end{macro}
%
% \begin{macro}{\@@_do_math_accent:}
%   \TeX{} will complain if \cs{l_@@_head_tl} happens to start with
%   \tn{accent} (the user used \tn{accent} in math mode).
%    \begin{macrocode}
\cs_new_protected:Npn \@@_do_math_accent:
  {
    \@@_prev_input_gpush:N \l_@@_head_tl
    \@@_print_action:
    \@@_scan_int:
    \@@_scan_math:
    \@@_prev_input_gpop:N \l_@@_head_tl
    \gtl_gput_right:NV \g_@@_output_gtl \l_@@_head_tl
    \tl_use:N \l_@@_head_tl \scan_stop:
    \@@_print_action:x { \tl_to_str:N \l_@@_head_tl }
  }
%    \end{macrocode}
% \end{macro}
%
%    \begin{macrocode}
\@@_new_tex_cmd:nn { accent }                               % 45
  {
    \@@_mode_non_vertical:n
      {
        \mode_if_math:TF
          { \@@_do_math_accent: } { \@@_do_accent: }
      }
  }
%    \end{macrocode}
%
%    \begin{macrocode}
\@@_new_tex_cmd:nn { math_accent }                          % 46
  { \@@_mode_math:n { \@@_do_math_accent: } }
%    \end{macrocode}
%
%    \begin{macrocode}
\@@_new_tex_cmd:nn { discretionary }                        % 47
  { \@@_not_implemented:n { discretionary } }
%    \end{macrocode}
%
% \subsubsection{Maths: from 48 to 56}
%
% \begin{itemize}
% \item |eq_no=48|
% \item |left_right=49|
% \item |math_comp=50|
% \item |limit_switch=51|
% \item |above=52|
% \item |math_style=53|
% \item |math_choice=54|
% \item |non_script=55|
% \item |vcenter=56|
% \end{itemize}
%
%    \begin{macrocode}
\@@_new_tex_cmd:nn { eq_no }                                % 48
  { \@@_not_implemented:n { eqno } }
%    \end{macrocode}
%
%    \begin{macrocode}
\@@_new_tex_cmd:nn { left_right }                           % 49
  { \@@_not_implemented:n { left/right } }
%    \end{macrocode}
%
%    \begin{macrocode}
\@@_new_tex_cmd:nn { math_comp }                            % 50
  { \@@_not_implemented:n { math~comp } }
%    \end{macrocode}
%
%    \begin{macrocode}
\@@_new_tex_cmd:nn { limit_switch }                         % 51
  { \@@_not_implemented:n { limits } }
%    \end{macrocode}
%
%    \begin{macrocode}
\@@_new_tex_cmd:nn { above }                                % 52
  { \@@_not_implemented:n { above } }
%    \end{macrocode}
%
%    \begin{macrocode}
\@@_new_tex_cmd:nn { math_style }                           % 53
  { \@@_not_implemented:n { math~style } }
%    \end{macrocode}
%
%    \begin{macrocode}
\@@_new_tex_cmd:nn { math_choice }                          % 54
  { \@@_not_implemented:n { math~choice } }
%    \end{macrocode}
%
%    \begin{macrocode}
\@@_new_tex_cmd:nn { non_script }                           % 55
  { \@@_not_implemented:n { non~script } }
%    \end{macrocode}
%
%    \begin{macrocode}
\@@_new_tex_cmd:nn { vcenter }                              % 56
  { \@@_not_implemented:n { vcenter } }
%    \end{macrocode}
%
% \subsubsection{From 57 to 70}
%
% \begin{itemize}
% \item |case_shift=57|
% \item |message=58|
% \item |extension=59|
% \item |in_stream=60|
% \item |begin_group=61|
% \item |end_group=62|
% \item |omit=63|
% \item |ex_space=64|
% \item |no_boundary=65|
% \item |radical=66|
% \item |end_cs_name=67|
% \item |char_given=68|
% \item |math_given=69|
% \item |last_item=70|
% \end{itemize}
%
%    \begin{macrocode}
\@@_new_tex_cmd:nn { case_shift }                           % 57
  {
    \@@_prev_input_gpush:N \l_@@_head_tl
    \@@_scan_toks:NN \c_false_bool \c_false_bool
    \@@_prev_input_gpop:N \l_@@_tmpa_tl
    \exp_after:wN \@@_case_shift:Nn \l_@@_tmpa_tl
  }
\cs_new_protected:Npn \@@_case_shift:Nn #1#2
  {
    #1 { \@@_back_input:n {#2} }
    \@@_print_action:x
      { \token_to_meaning:N #1 ~ \tl_to_str:n { {#2} } }
  }
%    \end{macrocode}
%
%    \begin{macrocode}
\@@_new_tex_cmd:nn { message }                              % 58
  {
    \@@_prev_input_gpush:N \l_@@_head_tl
    \@@_print_action:
    \@@_scan_toks_to_str:
    \@@_prev_input_gpop:N \l_@@_head_tl
    \tl_use:N \l_@@_head_tl
    \@@_print_action:x { \tl_to_str:N \l_@@_head_tl }
  }
%    \end{macrocode}
%
% Extensions are implemented in a later section.
%    \begin{macrocode}
\@@_new_tex_cmd:nn { extension }                            % 59
  {
    \@@_prev_input_gpush:N \l_@@_head_tl
    \@@_print_action:
    \@@_scan_extension_operands:
    \@@_prev_input_gpop:N \l_@@_head_tl
    \tl_use:N \l_@@_head_tl \scan_stop:
    \@@_print_action:x { \tl_to_str:N \l_@@_head_tl }
  }
%    \end{macrocode}
%
%    \begin{macrocode}
\@@_new_tex_cmd:nn { in_stream }                            % 60
  {
    \@@_prev_input_gpush:N \l_@@_head_tl
    \@@_print_action:
    \token_if_eq_meaning:NNTF \l_@@_head_token \tex_openin:D
      {
        \@@_scan_int:
        \@@_scan_optional_equals:
        \@@_scan_file_name:
      }
      { \@@_scan_int: }
    \@@_prev_input_gpop:N \l_@@_head_tl
    \tl_use:N \l_@@_head_tl \scan_stop:
    \@@_print_action:x { \tl_to_str:N \l_@@_head_tl }
  }
%    \end{macrocode}
%
%    \begin{macrocode}
\@@_new_tex_cmd:nn { begin_group }                          % 61
  {
    \gtl_gput_right:NV \g_@@_output_gtl \l_@@_head_tl
    \l_@@_head_token
    \gtl_clear:N \l_@@_after_group_gtl
    \@@_print_action:
  }
\@@_new_tex_cmd:nn { end_group }                            % 62
  {
    \gtl_gput_right:NV \g_@@_output_gtl \l_@@_head_tl
    \@@_back_input_gtl:N \l_@@_after_group_gtl
    \l_@@_head_token
    \@@_print_action:
  }
%    \end{macrocode}
%
%    \begin{macrocode}
\@@_new_tex_cmd:nn { omit }                                 % 63
  { \@@_not_implemented:n { omit } }
%    \end{macrocode}
%
%    \begin{macrocode}
\@@_new_tex_cmd:nn { ex_space }                             % 64
  {
    \@@_mode_non_vertical:n
      { \l_@@_head_token \@@_print_action: }
  }
%    \end{macrocode}
%
%    \begin{macrocode}
\@@_new_tex_cmd:nn { no_boundary }                          % 65
  {
    \@@_mode_non_vertical:n
      { \l_@@_head_token \@@_print_action: }
  }
%    \end{macrocode}
%
%    \begin{macrocode}
\@@_new_tex_cmd:nn { radical }                              % 66
  {
    \@@_mode_math:n
      {
        \@@_prev_input_gpush:N \l_@@_head_tl
        \@@_print_action:
        \@@_scan_int:
        \@@_scan_math:
        \@@_prev_input_gpop:N \l_@@_head_tl
        \gtl_gput_right:NV \g_@@_output_gtl \l_@@_head_tl
        \tl_use:N \l_@@_head_tl \scan_stop:
        \@@_print_action:x { \tl_to_str:N \l_@@_head_tl }
      }
  }
%    \end{macrocode}
%
%    \begin{macrocode}
\@@_new_tex_cmd:nn { end_cs_name }                          % 67
  {
    \@@_tex_error:nV { extra-endcsname } \l_@@_head_tl
    \@@_print_action:
  }
%    \end{macrocode}
%
% See |the_char| and |other_char|.
%    \begin{macrocode}
\@@_new_tex_cmd:nn { char_given }                           % 68
  {
    \@@_mode_non_vertical:n
      {
        \mode_if_math:TF
          { \@@_char_in_mmode:V \l_@@_head_char_int }
          { \@@_char:V \l_@@_head_char_int }
      }
  }
%    \end{macrocode}
%
% See |math_char_num|.
%    \begin{macrocode}
\@@_new_tex_cmd:nn { math_given }                           % 69
  {
    \@@_mode_math:n
      { \@@_mathchar:x { \int_use:N \l_@@_head_char_int } }
  }
%    \end{macrocode}
%
%    \begin{macrocode}
\@@_new_tex_cmd:nn { last_item }                            % 70
  { \@@_forbidden_case: }
%    \end{macrocode}
%
% \subsubsection{Extensions}
% ^^A todo: move this section.
%
% \begin{macro}{\@@_scan_extension_operands:}
%    \begin{macrocode}
\cs_new_protected:Npn \@@_scan_extension_operands:
  {
    \int_case:nnF \l_@@_head_char_int
      {
        { 0 } % openout
          {
            \@@_scan_int:
            \@@_scan_optional_equals:
            \@@_scan_file_name:
          }
        { 1 } % write
          {
            \@@_scan_int:
            \@@_scan_toks:NN \c_false_bool \c_false_bool
          }
        { 2 } % closeout
          { \@@_scan_int: }
        { 3 } % special
          { \@@_scan_toks_to_str: }
        { 4 } % immediate
          { \@@_scan_immediate_operands: }
        { 5 } % setlanguage
          {
            \mode_if_horizontal:TF
              { \@@_scan_int: }
              { \@@_error:nnnnn { invalid-mode } { } { } { } { } }
          }
        { 6 } % pdfliteral
          {
            \@@_scan_keyword:nF { dDiIrReEcCtT }
              { \@@_scan_keyword:n { pPaAgGeE } }
            \@@_scan_pdf_ext_toks:
          }
        { 7 } % pdfobj
          {
            \@@_scan_keyword:nTF
              { rReEsSeErRvVeEoObBjJnNuUmM }
              { \@@_skip_optional_space: }
              {
                \@@_scan_keyword:nF { uUsSeEoObBjJnNuUmM }
                  { \@@_scan_int: }
                \@@_scan_keyword:nT { sStTrReEaAmM }
                  {
                    \@@_scan_keyword:nT { aAtTtTrR }
                      { \@@_scan_pdf_ext_toks: }
                  }
                \@@_scan_keyword:n { fFiIlLeE }
                \@@_scan_pdf_ext_toks:
              }
          }
        { 8 } % pdfrefobj
          { \@@_scan_int: }
        { 9 } % pdfxform
          {
            \@@_scan_keyword:nT { aAtTtTrR }
              { \@@_scan_pdf_ext_toks: }
            \@@_scan_keyword:nTF { rReEsSoOuUrRcCeEsS }
              { \@@_scan_pdf_ext_toks: }
            \@@_scan_int:
          }
        { 10 } % pdfrefxform
          { \@@_scan_int: }
        { 11 } % pdfximage
          { \@@_scan_image: }
        { 12 } % pdfrefximage
          { \@@_scan_int: }
        { 13 } % pdfannot
          {
            \@@_scan_keyword:nTF
              { rReEsSeErRvVeEoObBjJnNuUmM }
              { \@@_scan_optional_space: }
              {
                \@@_scan_keyword:nT { uUsSeEoObBjJnNuUmM }
                  { \@@_scan_int: }
                \@@_scan_alt_rule:
                \@@_scan_pdf_ext_toks:
              }
          }
        { 14 } % pdfstartlink
          {
            \mode_if_vertical:TF
              { \@@_error:nnnnn { invalid-mode } { } { } { } { } }
              {
                \@@_scan_rule_attr:
                \@@_scan_action:
              }
          }
        { 15 } % pdfendlink
          {
            \mode_if_vertical:T
              { \@@_error:nnnnn { invalid-mode } { } { } { } { } }
          }
        { 16 } % pdfoutline
          {
            \@@_scan_keyword:nT { aAtTtTrR }
              { \@@_scan_pdf_ext_toks: }
            \@@_scan_action:
            \@@_scan_keyword:nT { cCoOuUnNtT }
              { \@@_scan_int: }
            \@@_scan_pdf_ext_toks:
          }
        { 17 } % pdfdest
          { \@@_scan_pdfdest_operands: }
        { 18 } % pdfthread
          { \@@_scan_rule_attr: \@@_scan_thread_id: }
        { 19 } % pdfstartthread
          { \@@_scan_rule_attr: \@@_scan_thread_id: }
        { 20 } % pdfendthread
          { }
        { 21 } % pdfsavepos
          { }
        { 22 } % pdfinfo
          { \@@_scan_pdf_ext_toks: }
        { 23 } % pdfcatalog
          {
            \@@_scan_pdf_ext_toks:
            \@@_scan_keyword:n { oOpPeEnNaAcCtTiIoOnN }
              { \@@_scan_action: }
          }
        { 24 } % pdfnames
          { \@@_scan_pdf_ext_toks: }
        { 25 } % pdffontattr
          {
            \@@_scan_font_ident:
            \@@_scan_pdf_ext_toks:
          }
        { 26 } % pdfincludechars
          {
            \@@_scan_font_ident:
            \@@_scan_pdf_ext_toks:
          }
        { 27 } % pdfmapfile
          { \@@_scan_pdf_ext_toks: }
        { 28 } % pdfmapline
          { \@@_scan_pdf_ext_toks: }
        { 29 } % pdftrailer
          { \@@_scan_pdf_ext_toks: }
        { 30 } % pdfresettimer
          { }
        { 31 } % pdffontexpand
          {
            \@@_scan_font_ident:
            \@@_scan_optional_equals:
            \@@_scan_int:
            \@@_scan_int:
            \@@_scan_int:
            \@@_scan_keyword:nT { aAuUtToOeExXpPaAnNdD }
              { \@@_skip_optional_space: }
          }
        { 32 } % pdfsetrandomseed
          { \@@_scan_int: }
        { 33 } % pdfsnaprefpoint
          { }
        { 34 } % pdfsnapy
          { \@@_scan_normal_glue: }
        { 35 } % pdfsnapycomp
          { \@@_scan_int: }
        { 36 } % pdfglyphtounicode
          {
            \@@_scan_pdf_ext_toks:
            \@@_scan_pdf_ext_toks:
          }
        { 37 } % pdfcolorstack
          { \@@_scan_pdfcolorstack_operands: }
        { 38 } % pdfsetmatrix
          { \@@_scan_pdf_ext_toks: }
        { 39 } % pdfsave
          { }
        { 40 } % pdfrestore
          { }
        { 41 } % pdfnobuiltintounicode
          { \@@_scan_font_ident: }
      }
      { } % no other cases.
  }
%    \end{macrocode}
% \end{macro}
%
% \begin{macro}{\@@_scan_pdfcolorstack_operands:}
%    \begin{macrocode}
\cs_new_protected:Npn \@@_scan_pdfcolorstack_operands:
  {
    \@@_scan_int:
    \@@_scan_keyword:nF { sSeEtT }
      {
        \@@_scan_keyword:nF { pPuUsShH }
          {
            \@@_scan_keyword:nF { pPoOpP }
              {
                \@@_scan_keyword:nF { cCuUrRrReEnNtT }
                  {
                    \@@_error:nnnnn { color-stack-action-missing }
                      { } { } { } { }
                  }
              }
          }
      }
  }
%    \end{macrocode}
% \end{macro}
%
% \begin{macro}{\@@_scan_rule_attr:}
%    \begin{macrocode}
\cs_new_protected:Npn \@@_scan_rule_attr:
  {
    \@@_scan_alt_rule:
    \@@_scan_keyword:nT { aAtTtTrR }
      { \@@_scan_pdf_ext_toks: }
  }
%    \end{macrocode}
% \end{macro}
%
% \begin{macro}{\@@_scan_action:}
%    \begin{macrocode}
\cs_new_protected:Npn \@@_scan_action:
  {
    \@@_scan_keyword:nTF { uUsSeErR }
      { \@@_scan_pdf_ext_toks: }
      {
        \@@_scan_keyword:nF { gGoOtToO }
          {
            \@@_scan_keyword:nF { tThHrReEaAdD }
              { \@@_error:nnnnn { action-type-missing } { } { } { } { } }
          }
      }
    \@@_scan_keyword:nT { fFiIlLeE }
      { \@@_scan_pdf_ext_toks: }
    \@@_scan_keyword:nTF { pPaAgGeE }
      {
        \@@_scan_int:
        \@@_scan_pdf_ext_toks:
      }
      {
        \@@_scan_keyword:nTF { nNaAmMeE }
          { \@@_scan_pdf_ext_toks: }
          {
            \@@_scan_keyword:nTF { nNuUmM }
              { \@@_scan_int: }
              { \@@_error:nnnnn { identifier-type-missing } { } { } { } { } }
          }
      }
    \@@_scan_keyword:nTF { nNeEwWwWiInNdDoOwW }
      { \@@_skip_optional_space: }
      {
        \@@_scan_keyword:nT { nNoOnNeEwWwWiInNdDoOwW }
          { \@@_skip_optional_space: }
      }
  }
%    \end{macrocode}
% \end{macro}
%
% \begin{macro}{\@@_scan_image:}
%   Used by \tn{pdfximage}.
%    \begin{macrocode}
\cs_new_protected:Npn \@@_scan_image:
  {
    \@@_scan_rule_attr:
    \@@_scan_keyword:nTF { nNaAmMeEdD }
      { \@@_scan_pdf_ext_toks: }
      {
        \@@_scan_keyword:nT { pPaAgGeE }
          { \@@_scan_int: }
      }
    \@@_scan_keyword:nT { cCoOlLoOrRsSpPaAcCeE }
      { \@@_scan_int: }
    \@@_scan_pdf_ext_toks:
  }
%    \end{macrocode}
% \end{macro}
%
% \begin{macro}{\@@_scan_immediate_operands:}
%    \begin{macrocode}
\cs_new_protected:Npn \@@_scan_immediate_operands:
  {
    \@@_get_x_next:
    \@@_set_cmd:
    \int_compare:nNnTF
      \l_@@_head_cmd_int = { \@@_tex_use:n { extension } }
      {
        \int_compare:nNnTF
          \l_@@_head_char_int < { 3 } % openout, write, closeout
          { \@@_scan_immediate_operands_aux: }
          {
            \int_case:nnF \l_@@_head_char_int
              {
                { 7 } { \@@_scan_extension_operands_aux: } % pdfobj
                { 9 } { \@@_scan_extension_operands_aux: } % pdfxform
                { 11 } { \@@_scan_extension_operands_aux: } %pdfximage
              }
              { \@@_scan_immediate_operands_bad: }
          }
      }
      { \@@_scan_immediate_operands_bad: }
  }
\cs_new_protected:Npn \@@_scan_immediate_operands_aux:
  {
    \@@_prev_input:V \l_@@_head_tl
    \@@_scan_extension_operands:
  }
\cs_new_protected:Npn \@@_scan_immediate_operands_bad:
  {
    \@@_back_input:
    \@@_prev_input_gpop:N \l_@@_head_tl
    \@@_print_action:x { \tl_to_str:N \l_@@_head_tl ignored }
    \@@_prev_input_gpush:
  }

%    \end{macrocode}
% \end{macro}
%
% \begin{macro}{\@@_scan_pdfdest_operands:}
%    \begin{macrocode}
\cs_new_protected:Npn \@@_scan_pdfdest_operands:
  {
    \@@_scan_keyword:nTF { nNuUmM }
      { \@@_scan_int: }
      {
        \@@_scan_keyword:nTF { nNaAmMeE }
          { \@@_scan_pdf_ext_toks: }
          { \@@_error:nnnnn { identifier-type-missing } { } { } { } { } }
      }
    \@@_scan_keyword:nTF { xXyYzZ }
      {
        \@@_scan_keyword:nT { zZoOoOmM }
          { \@@_scan_int: }
      }
      {
        \@@_scan_keyword:nF { fFiItTbBhH }
          {
            \@@_scan_keyword:nF { fFiItTbBvV }
              {
                \@@_scan_keyword:nF { fFiItTbB }
                  {
                    \@@_scan_keyword:nF { fFiItThHhH }
                      {
                        \@@_scan_keyword:nF { fFiItTvV }
                          {
                            \@@_scan_keyword:nTF
                              { fFiItTrR }
                              {
                                \@@_skip_optional_space:
                                \@@_scan_alt_rule:
                                \use_none:n
                              }
                              {
                                \@@_scan_keyword:nF
                                  { fFiItT }
                                  {
                                    \@@_error:nnnnn { destination-type-missing }
                                      { } { } { } { }
                                  }
                              }
                          }
                      }
                  }
              }
          }
      }
    \@@_skip_optional_space:
  }
%    \end{macrocode}
% \end{macro}
%
% \subsubsection{Assignments}
%
% Quoting \texttt{tex.web}: ``Every prefix, and every command code that
% might or might not be prefixed, calls the action procedure
% |prefixed_command|.  This routine accumulates a sequence of prefixes
% until coming to a non-prefix, then it carries out the command.''
% We define all those commands in one go, from
% |max_non_prefixed_command+1=71| to |max_command=102|.
%    \begin{macrocode}
\cs_set_protected:Npn \@@_tmp:w
  {
    \@@_prev_input_gpush:
    \@@_prefixed_command:
  }
\int_step_inline:nnnn
  { \@@_tex_use:n { max_non_prefixed_command } + 1 }
  { 1 }
  { \@@_tex_use:n { max_command } }
  { \cs_new_eq:cN { @@_cmd_#1: } \@@_tmp:w }
%    \end{macrocode}
%
% \begin{macro}{\@@_prefixed_command:}
%   Accumulated prefix codes so far are stored as the last item of
%   the previous-input sequence.
%    \begin{macrocode}
\cs_new_protected:Npn \@@_prefixed_command:
  {
    \int_while_do:nNnn
      \l_@@_head_cmd_int = { \@@_tex_use:n { prefix } }
      {
        \@@_prev_input:V \l_@@_head_tl
        \@@_get_x_non_relax:
        \@@_set_cmd:
        \int_compare:nNnF \l_@@_head_cmd_int
          > { \@@_tex_use:n { max_non_prefixed_command } }
          {
            \@@_prev_input_gpop:N \l_@@_tmpa_tl
            \@@_error:nxxxx { erroneous-prefixes }
              { \tl_to_str:N \l_@@_tmpa_tl }
              { \tl_to_str:N l_@@_head_tl }
              { } { }
            \@@_back_input:
            \@@_omit_after_assignment:w
          }
      }
    % ^^A todo: Discard non-\global prefixes if they are irrelevant
    % ^^A todo: Adjust for the setting of \globaldefs
    \cs_if_exist_use:cF
      { @@_prefixed_ \int_use:N \l_@@_head_cmd_int : }
      {
        \@@_error:nnnnn { internal } { prefixed } { } { } { }
        \@@_omit_after_assignment:w
      }
    \@@_after_assignment:
  }
%    \end{macrocode}
% \end{macro}
% We now need to implement prefixed commands, for command codes in the
% range $[71,102]$, with the exception of |prefix=93|, which would have
% been collected by the \cs{@@_prefixed_command:} loop.
%
% \begin{macro}{\@@_after_assignment:}
% \begin{macro}{\@@_omit_after_assignment:w}
%    \begin{macrocode}
\cs_new_protected:Npn \@@_after_assignment:
  {
    \@@_back_input_gtl:N \g_@@_after_assignment_gtl
    \gtl_gclear:N \g_@@_after_assignment_gtl
  }
\cs_new_protected:Npn \@@_omit_after_assignment:w
    #1 \@@_after_assignment: { }
%    \end{macrocode}
% \end{macro}
% \end{macro}
%
% \begin{macro}{\@@_prefixed_new:nn}
%    \begin{macrocode}
\cs_new_protected:Npn \@@_prefixed_new:nn #1#2
  {
    \cs_new_protected:cpn
      { @@_prefixed_ \@@_tex_use:n {#1} : } {#2}
  }
%    \end{macrocode}
% \end{macro}
%
% \begin{macro}{\@@_assign_token:n}
%    \begin{macrocode}
\cs_new_protected:Npn \@@_assign_token:n #1
  {
    \@@_prev_input_gpop:N \l_@@_head_tl
    #1
    \tl_use:N \l_@@_head_tl \scan_stop:
    \@@_print_assigned_token:
  }
%    \end{macrocode}
% \end{macro}
%
% \begin{macro}{\@@_assign_register:}
%    \begin{macrocode}
\cs_new_protected:Npn \@@_assign_register:
  {
    \@@_prev_input_gpop:N \l_@@_head_tl
    \tl_use:N \l_@@_head_tl \scan_stop:
    \@@_print_assigned_register:
  }
%    \end{macrocode}
% \end{macro}
%
% \begin{macro}{\@@_assign_value:nn}
%    \begin{macrocode}
\cs_new_protected:Npn \@@_assign_value:nn #1#2
  {
    \tl_if_empty:nF {#1}
      {
        \@@_prev_input_gpush:N \l_@@_head_tl
        \@@_print_action:x { \tl_to_str:N \l_@@_head_tl }
        #1
        \@@_prev_input_gpop:N \l_@@_head_tl
      }
    \@@_prev_input:V \l_@@_head_tl
    \tl_set_eq:NN \l_@@_defined_tl \l_@@_head_tl
    \@@_scan_optional_equals:
    #2
    \@@_assign_register:
  }
%    \end{macrocode}
% \end{macro}
%
% \begin{macro}{\@@_assign_toks:}
%    \begin{macrocode}
\@@_prefixed_new:nn { toks_register }                       % 71
  {
    \int_compare:nNnT \l_@@_head_char_int = \c_zero
      { % \toks
        \@@_prev_input_gpush:N \l_@@_head_tl
        \@@_print_action:
        \@@_scan_int:
        \@@_prev_input_gpop:N \l_@@_head_tl
      }
    \@@_assign_toks:
  }
\@@_prefixed_new:nn { assign_toks }                         % 72
  { \@@_assign_toks: }
\cs_new_protected:Npn \@@_assign_toks:
  {
    \@@_prev_input_silent:V \l_@@_head_tl
    \@@_print_action:
    \tl_set_eq:NN \l_@@_defined_tl \l_@@_head_tl
    \@@_scan_optional_equals:
    \@@_get_x_non_relax:
    \@@_set_cmd:
    \int_compare:nNnTF
      \l_@@_head_cmd_int = { \@@_tex_use:n { toks_register } }
      {
        \@@_prev_input:V \l_@@_head_tl
        \int_compare:nNnT \l_@@_head_char_int = \c_zero
          { \@@_scan_int: }
      }
      {
        \int_compare:nNnTF
          \l_@@_head_cmd_int = { \@@_tex_use:n { assign_toks } }
          { \@@_prev_input:V \l_@@_head_tl }
          {
            \@@_back_input:
            \@@_scan_toks:NN \c_false_bool \c_false_bool
          }
      }
    \@@_assign_register:
  }
%    \end{macrocode}
% \end{macro}
%
%    \begin{macrocode}
\@@_prefixed_new:nn { assign_int }                          % 73
  { \@@_assign_value:nn { } { \@@_scan_int: } }
\@@_prefixed_new:nn { assign_dimen }                        % 74
  { \@@_assign_value:nn { } { \@@_scan_normal_dimen: } }
\@@_prefixed_new:nn { assign_glue }                         % 75
  { \@@_assign_value:nn { } { \@@_scan_normal_glue: } }
\@@_prefixed_new:nn { assign_mu_glue }                      % 76
  { \@@_assign_value:nn { } { \@@_scan_mu_glue: } }
\@@_prefixed_new:nn { assign_font_dimen }                   % 77
  {
    \@@_assign_value:nn
      { \@@_scan_int: \@@_scan_font_ident: }
      { \@@_scan_normal_dimen: }
  }
\@@_prefixed_new:nn { assign_font_int }                     % 78
  {
    \@@_assign_value:nn
      { \@@_scan_font_int: } { \@@_scan_int: }
  }
\@@_prefixed_new:nn { set_aux }                             % 79
  { % prevdepth = 1, spacefactor = 102
    \int_compare:nNnTF \l_@@_head_char_int = \c_one
      { \@@_assign_value:nn { } { \@@_scan_normal_dimen: } }
      { \@@_assign_value:nn { } { \@@_scan_int: } }
  }
\@@_prefixed_new:nn { set_prev_graf }                       % 80
  { \@@_assign_value:nn { } { \@@_scan_int: } }
\@@_prefixed_new:nn { set_page_dimen }                      % 81
  { \@@_assign_value:nn { } { \@@_scan_normal_dimen: } }
\@@_prefixed_new:nn { set_page_int }                        % 82
  { \@@_assign_value:nn { } { \@@_scan_int: } }
\@@_prefixed_new:nn { set_box_dimen }                       % 83
  {
    \@@_assign_value:nn
      { \@@_scan_int: } { \@@_scan_normal_dimen: }
  }
%    \end{macrocode}
%
%    \begin{macrocode}
\@@_prefixed_new:nn { set_shape }                           % 84
  {
    \@@_assign_value:nn { \@@_scan_int: }
      {
        \prg_replicate:nn
          {
            \tl_if_head_eq_meaning:VNT
              \l_@@_defined_tl \tex_parshape:D { \c_two * }
            \tl_tail:N \l_@@_defined_tl
          }
          { \@@_scan_int: }
      }
  }
%    \end{macrocode}
%
%    \begin{macrocode}
\@@_prefixed_new:nn { def_code }                            % 85
  {
    \@@_assign_value:nn
      { \@@_scan_int: } { \@@_scan_int: }
  }
\@@_prefixed_new:nn { def_family }                          % 86
  {
    \@@_assign_value:nn
      { \@@_scan_int: } { \@@_scan_font_ident: }
  }
\@@_prefixed_new:nn { set_font }                            % 87
  {
    \@@_prev_input_gpop:N \l_@@_tmpa_tl
    \tl_put_left:NV \l_@@_head_tl \l_@@_tmpa_tl
    \tl_use:N \l_@@_head_tl \scan_stop:
    \gtl_gput_right:NV \g_@@_output_gtl \l_@@_head_tl
    \@@_print_action:
  }
\@@_prefixed_new:nn { def_font }                            % 88
  {
    \@@_prev_input_silent:V \l_@@_head_tl
    \@@_set_action_text:x { \tl_to_str:N \l_@@_head_tl }
    \@@_scan_r_token:
    \@@_print_action:x
      { \g_@@_action_text_str \tl_to_str:N \l_@@_defined_tl }
    \@@_scan_optional_equals:
    \@@_scan_file_name:
    \bool_gset_true:N \g_@@_name_in_progress_bool
    \@@_scan_keyword:nTF { aAtT }
      { \@@_scan_normal_dimen: }
      {
        \@@_scan_keyword:nT { sScCaAlLeEdD }
          { \@@_scan_int: }
      }
    \bool_gset_false:N \g_@@_name_in_progress_bool
    \@@_assign_token:n { }
  }
%    \end{macrocode}
%
% |register=89|, |advance=90|, |multiply=91|, |divide=92| are
% implemented elsewhere.  |prefix=93| is never needed (see explanation
% above).
%
% |let|, |futurelet|
%    \begin{macrocode}
\@@_prefixed_new:nn { let }                                 % 94
  {
    \@@_prev_input_gpush:N \l_@@_head_tl
    \token_if_eq_meaning:NNTF \l_@@_head_token \tex_let:D
      { % |let|
        \@@_scan_r_token:
        \@@_prev_input_get:N \l_@@_tmpa_tl
        \@@_print_action:x { \tl_to_str:N \l_@@_tmpa_tl }
        \@@_get_next:
        \bool_while_do:nn
          { \token_if_eq_catcode_p:NN \l_@@_head_token \c_space_token }
          { \@@_get_next: }
        \tl_if_eq:NNT \l_@@_head_tl \c_@@_eq_tl
          { \@@_get_next: }
        \token_if_eq_catcode:NNT \l_@@_head_token \c_space_token
          { \@@_get_next: }
      }
      { % |futurelet|
        \@@_scan_r_token:
        \@@_prev_input_get:N \l_@@_tmpa_tl
        \@@_print_action:x { \tl_to_str:N \l_@@_tmpa_tl }
        \@@_get_next:
        \gtl_set_eq:NN \l_@@_tmpb_gtl \l_@@_head_gtl
        \@@_get_next:
        \@@_back_input:
        \gtl_set_eq:NN \l_@@_head_gtl \l_@@_tmpb_gtl
        \@@_back_input:
      }
    \@@_prev_input_gpop:N \l_@@_tmpa_tl
    \tl_put_right:Nn \l_@@_tmpa_tl { = ~ \l_@@_head_token }
    \@@_prev_input_gpop:N \l_@@_head_tl
    \@@_exp_args:Nx \use:n
      {
        \exp_not:V \l_@@_head_tl
        \tex_let:D \tl_tail:N \l_@@_tmpa_tl
      }
    \@@_print_assigned_token:
  }
%    \end{macrocode}
%
% ^^A todo: print the temporary relax assignment?
%    \begin{macrocode}
\@@_prefixed_new:nn { shorthand_def }                       % 95
  {
    \@@_prev_input_silent:V \l_@@_head_tl
    \tl_set:Nx \l_@@_prev_action_tl
      { \tl_to_str:N \l_@@_head_tl }
    \@@_scan_r_token:
    \@@_print_action:x
      { \l_@@_prev_action_tl \tl_to_str:N \l_@@_defined_tl }
    \exp_after:wN \cs_set_eq:NN \l_@@_defined_tl \scan_stop:
    \@@_scan_optional_equals:
    \@@_scan_int:
    \@@_assign_token:n { }
  }
%    \end{macrocode}
%
% \begin{macro}[EXP]{\@@_read_to_cs_safe:nTF, \@@_read_to_cs_safe:fTF}
%   After \tn{read} or \tn{readline}, find an int, the mandatory
%   keyword |to|, and an assignable token.  The \tn{read} and
%   \tn{readline} primitives throw a fatal error in \tn{nonstopmode}
%   and in \tn{batchmode} when trying to read from a stream that is
%   outside $[0,15]$ or that is not open (according to \tn{ifeof}).  We
%   detect this situation using \cs{@@_read_to_cs_safe:nTF} after
%   grabbing all arguments of the primitives.  If reading is unsafe,
%   let the user know that \TeX{} would have thrown a fatal error.
%    \begin{macrocode}
\@@_prefixed_new:nn { read_to_cs }                          % 96
  {
    \@@_prev_input_silent:V \l_@@_head_tl
    \@@_print_action:x { \tl_to_str:N \l_@@_head_tl }
    \@@_scan_int:
    \@@_scan_to:
    \@@_scan_r_token:
    \@@_prev_input_get:N \l_@@_tmpa_tl
    \@@_read_to_cs_safe:fTF
      { \@@_tl_first_int:N \l_@@_tmpa_tl }
      { \@@_assign_token:n { } }
      {
        \@@_prev_input_gpop:N \l_@@_head_tl
        \@@_tex_fatal_error:nV { cannot-read } \l_@@_head_tl
      }
  }
\prg_new_conditional:Npnn \@@_read_to_cs_safe:n #1 { TF }
  {
    \int_compare:nNnTF { \etex_interactionmode:D } > { 1 }
      { \prg_return_true: }
      {
        \int_compare:nNnTF {#1} < { 0 }
          { \prg_return_false: }
          {
            \int_compare:nNnTF {#1} > { 15 }
              { \prg_return_false: }
              {
                \tex_ifeof:D #1 \exp_stop_f:
                  \prg_return_false:
                \else:
                  \prg_return_true:
                \fi:
              }
          }
      }
  }
\cs_generate_variant:Nn \@@_read_to_cs_safe:nTF { f }
%    \end{macrocode}
% \end{macro}
%
%    \begin{macrocode}
\@@_prefixed_new:nn { def }                                 % 97
  {
    \@@_prev_input_get:N \l_@@_tmpa_tl
    \tl_set:NV \l_@@_defining_tl \l_@@_tmpa_tl
    \tl_put_right:NV \l_@@_defining_tl \l_@@_head_tl
    \@@_prev_input_gpush:N \l_@@_head_tl
    \int_compare:nNnTF \l_@@_head_char_int < \c_two
      { % def/gdef
        \@@_scan_r_token:
        \tl_put_right:NV \l_@@_defining_tl \l_@@_defined_tl
        \@@_scan_toks:NN \c_true_bool \c_false_bool
      }
      { % edef/xdef
        \@@_scan_r_token:
        \tl_put_right:NV \l_@@_defining_tl \l_@@_defined_tl
        \@@_scan_toks:NN \c_true_bool \c_true_bool
      }
    \@@_prev_input_gpop:N \l_@@_head_tl
    \@@_prev_input:V \l_@@_head_tl
    \@@_assign_token:n
      { \tl_set_eq:NN \l_@@_head_tl \l_@@_defining_tl }
  }
%    \end{macrocode}
%
% \tn{setbox} is a bit special: directly put it in
% the previous-input sequence with the prefixes; the box code will take
% care of things, and expects a single item containing what it needs to
% do.
%    \begin{macrocode}
\@@_prefixed_new:nn { set_box }                             % 98
  {
    \@@_prev_input:V \l_@@_head_tl
    \@@_scan_int:
    \@@_scan_optional_equals:
    \bool_if:NTF \g_@@_set_box_allowed_bool
      { \@@_do_box:N \c_false_bool }
      {
        \@@_error:nnnnn { improper-setbox } { } { } { } { }
        \@@_prev_input_gpop:N \l_@@_tmpa_tl
        \@@_omit_after_assignment:w
      }
  }
%    \end{macrocode}
%
% ^^A todo: \hyphenation{...{...}...} recovers weirdly (no nesting).
% \tn{hyphenation} and \tn{patterns}
%    \begin{macrocode}
\@@_prefixed_new:nn { hyph_data }                           % 99
  {
    \@@_prev_input:V \l_@@_head_tl
    \@@_scan_toks:NN \c_false_bool \c_false_bool
    \@@_assign_token:n { }
  }
%    \end{macrocode}
%
%    \begin{macrocode}
\@@_prefixed_new:nn { set_interaction }                     % 100
  {
    \@@_prev_input_gpop:N \l_@@_tmpa_tl
    \tl_put_left:NV \l_@@_head_tl \l_@@_tmpa_tl
    \tl_use:N \l_@@_head_tl \scan_stop:
    \@@_print_assignment:x { \tl_to_str:N \l_@@_head_tl }
  }
%    \end{macrocode}
%
% ^^A todo: print the temporary nullfont assignment?
%    \begin{macrocode}
\@@_prefixed_new:nn { letterspace_font }                    % 101
  {
    \@@_prev_input_silent:V \l_@@_head_tl
    \@@_set_action_text:x { \tl_to_str:N \l_@@_head_tl }
    \@@_scan_r_token:
    \@@_print_action:x
      { \g_@@_action_text_str \tl_to_str:N \l_@@_defined_tl }
    \exp_after:wN \cs_set_eq:NN \l_@@_defined_tl \@@_nullfont:
    \@@_scan_optional_equals:
    \@@_scan_font_ident:
    \@@_scan_int:
    \@@_assign_token:n { }
  }
%    \end{macrocode}
%
% ^^A todo: print the temporary nullfont assignment?
%    \begin{macrocode}
\@@_prefixed_new:nn { pdf_copy_font }                       % 102
  {
    \@@_prev_input_silent:V \l_@@_head_tl
    \@@_set_action_text:x { \tl_to_str:N \l_@@_head_tl }
    \@@_scan_r_token:
    \@@_print_action:x
      { \g_@@_action_text_str \tl_to_str:N \l_@@_defined_tl }
    \exp_after:wN \cs_set_eq:NN \l_@@_defined_tl \@@_nullfont:
    \@@_scan_optional_equals:
    \@@_scan_font_ident:
    \@@_assign_token:n { }
  }
%    \end{macrocode}
%
% Changes to numeric registers (\tn{count}, \tn{dimen}, \tn{skip},
% \tn{muskip}, and commands with a built-in number).
%    \begin{macrocode}
\@@_prefixed_new:nn { register }                            % 89
  { \@@_do_register:N \c_zero }
\@@_prefixed_new:nn { advance }                             % 90
  { \@@_do_operation:N \c_one }
\@@_prefixed_new:nn { multiply }                            % 91
  { \@@_do_operation:N \c_two }
\@@_prefixed_new:nn { divide }                              % 92
  { \@@_do_operation:N \c_three }
%    \end{macrocode}
%
% \begin{macro}{\@@_do_operation:N, \@@_do_operation_fail:w}
%   \begin{macrocode}
\cs_new_protected:Npn \@@_do_operation:N #1
  {
    \@@_prev_input_silent:V \l_@@_head_tl
    \@@_print_action:
    \@@_get_x_next:
    \@@_set_cmd:
    \int_compare:nNnTF
      \l_@@_head_cmd_int > { \@@_tex_use:n { assign_mu_glue } }
      {
        \int_compare:nNnTF
          \l_@@_head_cmd_int = { \@@_tex_use:n { register } }
          { \@@_do_register:N #1 }
          { \@@_do_operation_fail:w }
      }
      {
        \int_compare:nNnTF
          \l_@@_head_cmd_int < { \@@_tex_use:n { assign_int } }
          { \@@_do_operation_fail:w }
          {
            \@@_prev_input:V \l_@@_head_tl
            \exp_args:NNf \@@_do_register_set:Nn #1
              {
                \int_eval:n
                  {
                    \l_@@_head_cmd_int
                    - \@@_tex_use:n { assign_toks }
                  }
              }
          }
      }
  }
\cs_new_protected:Npn \@@_do_operation_fail:w
  {
    \@@_error:nnnnn { after-advance } { } { } { } { }
    \@@_prev_input_gpop:N \l_@@_tmpa_tl
    \@@_omit_after_assignment:w
  }
%    \end{macrocode}
% \end{macro}
%
% \begin{macro}{\@@_do_register:N, \@@_do_register_aux:Nn}
%    \begin{macrocode}
\cs_new_protected:Npn \@@_do_register:N #1
  {
    \exp_args:NNV \@@_do_register_aux:Nn #1
      \l_@@_head_char_int
  }
\cs_new_protected:Npn \@@_do_register_aux:Nn #1#2
  {
    \int_compare:nNnTF { \tl_tail:n {#2} } = \c_zero
      {
        \@@_prev_input_gpush:N \l_@@_head_tl
        \@@_print_assignment:
        \@@_scan_int:
        \@@_prev_input_gpop:N \l_@@_head_tl
        \@@_prev_input_silent:V \l_@@_head_tl
      }
      {
        \@@_prev_input_silent:V \l_@@_head_tl
        \@@_print_assignment:
      }
    \tl_set_eq:NN \l_@@_defined_tl \l_@@_head_tl
    \exp_args:NNf \@@_do_register_set:Nn #1
      { \int_eval:n { #2 / 1 000 000 } }
  }
%    \end{macrocode}
% \end{macro}
%
% \begin{macro}{\@@_do_register_set:Nn}
%    \begin{macrocode}
\cs_new_protected:Npn \@@_do_register_set:Nn #1#2
  {
    \int_compare:nNnTF {#1} = \c_zero
      { % truly register command
        \@@_scan_optional_equals:
      }
      { % \advance, \multiply, \divide
        \@@_scan_keyword:nF { bByY }
          { \@@_prev_input_silent:n { by } }
      }
    \int_compare:nNnTF {#1} < \c_two
      {
        \int_case:nnF {#2}
          {
            { 1 } { \@@_scan_int:          } % count
            { 2 } { \@@_scan_normal_dimen: } % dim
            { 3 } { \@@_scan_normal_glue:  } % glue
            { 4 } { \@@_scan_mu_glue:      } % muglue
          }
          { \@@_error:nxxxx { internal } { do-reg=#2 } { } { } { } }
      }
      { \@@_scan_int: }
    \@@_assign_register:
  }
%    \end{macrocode}
% \end{macro}
%
% The following is used for instance when making accents.
%    \begin{macrocode}
\cs_new_protected:Npn \@@_do_assignments:
  {
    \@@_get_x_non_relax:
    \@@_set_cmd:
    \int_compare:nNnT
      \l_@@_head_cmd_int
      > { \@@_tex_use:n { max_non_prefixed_command } }
      {
        \bool_gset_false:N \g_@@_set_box_allowed_bool
        \@@_prev_input_gpush:
        \@@_prefixed_command:
        \bool_gset_true:N \g_@@_set_box_allowed_bool
        \@@_do_assignments:
      }
  }
%    \end{macrocode}
%
% \subsection{Expandable primitives}
%
% This section implements expandable primitives, which have the
% following command codes:
% \begin{itemize}
% \item |undefined_cs=103| for undefined control sequences (not quite a
%   primitive).
% \item |expand_after=104| for \tn{expandafter} and \tn{unless}.
% \item |no_expand=105| for \tn{noexpand} and \tn{pdfprimitive}.
% \item |input=106| for \tn{input}, \tn{endinput} and \tn{scantokens}.
% \item |if_test=107| for the conditionals, \tn{if}, \tn{ifcat},
%   \tn{ifnum}, \tn{ifdim}, \tn{ifodd}, \tn{ifvmode}, \tn{ifhmode},
%   \tn{ifmmode}, \tn{ifinner}, \tn{ifvoid}, \tn{ifhbox}, \tn{ifvbox},
%   \tn{ifx}, \tn{ifeof}, \tn{iftrue}, \tn{iffalse}, \tn{ifcase},
%   \tn{ifdefined}, \tn{ifcsname}, \tn{iffontchar}, \tn{ifincsname},
%   \tn{ifpdfprimitive}, \tn{ifpdfabsnum}, and \tn{ifpdfabsdim}.
% \item |fi_or_else=108| for \tn{fi}, \tn{else} and \tn{or}.
% \item |cs_name=109| for \tn{csname}.
% \item |convert=110| for \tn{number}, \tn{romannumeral}, \tn{string},
%   \tn{meaning}, \tn{fontname}, \tn{eTeXrevision}, \tn{pdftexrevision},
%   \tn{pdftexbanner}, \tn{pdffontname}, \tn{pdffontobjnum},
%   \tn{pdffontsize}, \tn{pdfpageref}, \tn{pdfxformname},
%   \tn{pdfescapestring}, \tn{pdfescapename}, \tn{leftmarginkern},
%   \tn{rightmarginkern}, \tn{pdfstrcmp}, \tn{pdfcolorstackinit},
%   \tn{pdfescapehex}, \tn{pdfunescapehex}, \tn{pdfcreationdate},
%   \tn{pdffilemoddate}, \tn{pdffilesize}, \tn{pdfmdfivesum},
%   \tn{pdffiledump}, \tn{pdfmatch}, \tn{pdflastmatch},
%   \tn{pdfuniformdeviate}, \tn{pdfnormaldeviate}, \tn{pdfinsertht},
%   \tn{pdfximagebbox}, \tn{jobname}, and in
%   \LuaTeX{} \tn{directlua}, \tn{expanded}, \tn{luaescapestring}.
% \item |the=111| for \tn{the}, \tn{unexpanded}, and \tn{detokenize}.
% \item |top_bot_mark=112| \tn{topmark}, \tn{firstmark}, \tn{botmark},
%   \tn{splitfirstmark}, \tn{splitbotmark}, \tn{topmarks},
%   \tn{firstmarks}, \tn{botmarks}, \tn{splitfirstmarks}, and
%   \tn{splitbotmarks}.
% \item |call=113| for macro calls, implemented by \cs{@@_macro_call:}.
% \item |end_template=117| for \TeX{}'s end template.
% \end{itemize}
%
% Let \TeX{} trigger an error.
%    \begin{macrocode}
\@@_new_tex_expandable:nn { undefined_cs }                  % 103
  { \tl_use:N \l_@@_head_tl \@@_print_expansion: }
%    \end{macrocode}
%
% \begin{macro}{\@@_expandafter:, \@@_unless:, \@@_unless_bad:}
%    \begin{macrocode}
\@@_new_tex_expandable:nn { expand_after }                  % 104
  {
    \token_if_eq_meaning:NNTF \l_@@_head_token \tex_expandafter:D
      { \@@_expandafter: } { \@@_unless: }
  }
\cs_new_protected:Npn \@@_expandafter:
  {
    \gtl_set_eq:NN \l_@@_tmpb_gtl \l_@@_head_gtl
    \@@_get_next:
    \gtl_concat:NNN \l_@@_head_gtl
      \l_@@_tmpb_gtl \l_@@_head_gtl
    \@@_prev_input_gpush_gtl:N \l_@@_head_gtl
    \@@_print_expansion:x { \gtl_to_str:N \l_@@_head_gtl }
    \@@_get_next:
    \@@_token_if_expandable:NTF \l_@@_head_token
      { \@@_expand_do:N \prg_do_nothing: }
      { \@@_back_input: }
    \@@_prev_input_gpop:N \l_@@_head_gtl
    \@@_set_action_text:x
      { back_input: ~ \gtl_to_str:N \l_@@_head_gtl }
    \gtl_pop_left:N \l_@@_head_gtl
    \@@_back_input:
    \@@_print_expansion:
  }
\cs_new_protected:Npn \@@_unless:
  {
    \@@_get_token:
    \int_compare:nNnTF
      \l_@@_head_cmd_int = { \@@_tex_use:n { if_test } }
      {
        \token_if_eq_meaning:NNTF \l_@@_head_token \tex_ifcase:D
          { \@@_unless_bad: }
          {
            \tl_put_left:Nn \l_@@_head_tl { \reverse_if:N }
            % \int_add:Nn \l_@@_head_char_int { 32 }
            \@@_expand_nonmacro:
          }
      }
      { \@@_unless_bad: }
  }
\cs_new_protected:Npn \@@_unless_bad:
  {
    \@@_error:nnnnn { bad-unless } { } { } { } { }
    \@@_back_input:
  }
%    \end{macrocode}
% \end{macro}
%
% \begin{macro}{\@@_noexpand:N, \@@_noexpand_after:, \@@_pdfprimitive:}
%   Currently not fully implemented.
%
%   The argument of \cs{@@_noexpand:N} is \cs{prg_do_nothing:} when
%   \tn{noexpand} is hit by \tn{expandafter}; otherwise it is one of
%   various loop commands (\cs{@@_get_x_next:},
%   \cs{@@_get_x_or_protected:}, \cs{@@_get_token_xdef:},
%   \cs{@@_get_token_x:}) that would call \cs{@@_get_next:} and possibly
%   expand the token more.  For these cases we simply stop after
%   \cs{@@_get_next:} and if the token is expandable we pretend its
%   meaning is \tn{relax}.
%
%   The case of \tn{expandafter} (so \cs{prg_do_nothing:}) is tougher.
%   Do nothing if the next token is an explicit non-active character
%   (begin-group and end-group characters are detected by
%   \cs{l_@@_head_tl}, the rest by testing if the token is definable).
%   Otherwise the token must be marked with \cs{notexpanded:} (even if
%   the token is currently a non-expandable primitive, as its meaning
%   can be changed by the code skipped over by \tn{expandafter}).  That
%   \cs{notexpanded:} marker should be removed if the token is taken as
%   the argument of a macro, but we fail to do that.  We set the
%   \cs{notexpanded:\ldots{}} command to be a special \tn{relax} marker
%   to make it quickly recognizable in \cs{@@_get_next:}.  This is
%   incidentally the same meaning used by \TeX{} for expandable commands.
%    \begin{macrocode}
\@@_new_tex_expandable:nn { no_expand }                     % 105
  {
    \token_if_eq_meaning:NNTF \l_@@_head_token \tex_noexpand:D
      { \@@_noexpand:N }
      { \@@_pdfprimitive: }
  }
\cs_new_protected:Npn \@@_noexpand:N #1
  {
    \@@_get_token:
    \cs_if_eq:NNTF #1 \prg_do_nothing:
      {
        \tl_if_empty:NTF \l_@@_head_tl
          { \@@_back_input: }
          {
            \exp_after:wN \@@_token_if_definable:NTF \l_@@_head_tl
              { \@@_noexpand_after: }
              { \@@_back_input: }
          }
      }
      {
        \@@_back_input:
        \@@_get_next:
        \@@_token_if_expandable:NT \l_@@_head_token
          { \cs_set_eq:NN \l_@@_head_token \@@_special_relax: }
      }
  }
\cs_new_protected:Npn \@@_noexpand_after:
  {
    \group_begin:
      \@@_set_escapechar:n { 92 }
      \exp_args:NNc
    \group_end:
    \@@_noexpand_after:N
      { notexpanded: \exp_after:wN \token_to_str:N \l_@@_head_tl }
  }
\cs_new_protected:Npn \@@_noexpand_after:N #1
  {
    \cs_gset_eq:NN #1 \@@_special_relax:
    \@@_back_input:n {#1}
  }
\cs_new_protected:Npn \@@_pdfprimitive:
  { \@@_not_implemented:n { pdfprimitive } }
%    \end{macrocode}
% \end{macro}
%
% \begin{macro}{\@@_endinput:, \@@_scantokens:, \@@_input:}
%    \begin{macrocode}
\@@_new_tex_expandable:nn { input }                       % 106
  {
    \int_case:nnF \l_@@_head_char_int
      {
        { 1 } { \@@_endinput: } % \endinput
        { 2 } { \@@_scantokens: } % \scantokens
      }
      { % 0=\input
        \bool_if:NTF \g_@@_name_in_progress_bool
          { \@@_insert_relax: } { \@@_input: }
      }
  }
\cs_new_protected:Npn \@@_endinput:
  {
    \group_begin:
      \msg_warning:nn { unravel } { endinput-ignored }
    \group_end:
    \@@_print_expansion:
  }
\cs_new_protected:Npn \@@_scantokens:
  {
    \@@_prev_input_gpush:
    \@@_scan_toks:NN \c_false_bool \c_false_bool
    \@@_prev_input_gpop:N \l_@@_tmpa_tl
    \tl_set_rescan:Nno \l_@@_head_tl { } \l_@@_tmpa_tl
    \@@_back_input:V \l_@@_head_tl
    \@@_print_expansion:x { \tl_to_str:N \l_@@_tmpa_tl }
  }
\cs_new_protected:Npn \@@_input:
  {
    \@@_prev_input_gpush:N \l_@@_head_tl
    \@@_scan_file_name:
    \@@_prev_input_gpop:N \l_@@_head_tl
    \tl_set:Nx \l_@@_tmpa_tl { \tl_tail:N \l_@@_head_tl }
    \@@_file_get:nN \l_@@_tmpa_tl \l_@@_tmpa_tl
    \@@_back_input:V \l_@@_tmpa_tl
    \@@_print_expansion:x { \tl_to_str:N \l_@@_head_tl }
  }
%    \end{macrocode}
% \end{macro}
%
% \begin{macro}{\@@_csname_loop:}
%    \begin{macrocode}
\@@_new_tex_expandable:nn { cs_name }                       % 109
  {
    \@@_prev_input_gpush:N \l_@@_head_tl
    \@@_print_expansion:
    \@@_csname_loop:
    \@@_prev_input_silent:V \l_@@_head_tl
    \@@_prev_input_gpop:N \l_@@_head_tl
    \@@_back_input_tl_o:
  }
\cs_new_protected:Npn \@@_csname_loop:
  {
    \@@_get_x_next:
    \@@_gtl_if_head_is_definable:NTF \l_@@_head_gtl
      {
        \cs_if_eq:NNF \l_@@_head_token \tex_endcsname:D
          {
            \@@_back_input:
            \@@_tex_error:nV { missing-endcsname } \l_@@_head_tl
            \tl_set:Nn \l_@@_head_tl { \tex_endcsname:D }
          }
      }
      {
        \@@_prev_input_silent:x
          { \@@_token_to_char:N \l_@@_head_token }
        \@@_csname_loop:
      }
  }
%    \end{macrocode}
% \end{macro}
%
%    \begin{macrocode}
\@@_new_tex_expandable:nn { convert }                       % 110
  {
    \@@_prev_input_gpush:N \l_@@_head_tl
    \@@_print_expansion:
    \int_case:nn \l_@@_head_char_int
      {
        0      \@@_scan_int:
        1      \@@_scan_int:
        2      \@@_convert_string:
        3      \@@_convert_meaning:
        4      \@@_scan_font_ident:
        8      \@@_scan_font_ident:
        9      \@@_scan_font_ident:
        { 10 } \@@_scan_font_ident:
        { 11 } \@@_scan_int:
        { 12 } \@@_scan_int:
        { 13 } \@@_scan_pdf_ext_toks:
        { 14 } \@@_scan_pdf_ext_toks:
        { 15 } \@@_scan_int:
        { 16 } \@@_scan_int:
        { 17 } \@@_scan_pdfstrcmp:
        { 18 } \@@_scan_pdfcolorstackinit:
        { 19 } \@@_scan_pdf_ext_toks:
        { 20 } \@@_scan_pdf_ext_toks:
        { 22 } \@@_scan_pdf_ext_toks:
        { 23 } \@@_scan_pdf_ext_toks:
        { 24 }
          {
            \@@_scan_keyword:n { fFiIlLeE }
            \@@_scan_pdf_ext_toks:
          }
        { 25 } \@@_scan_pdffiledump:
        { 26 } \@@_scan_pdfmatch:
        { 27 } \@@_scan_int:
        { 28 } \@@_scan_int:
        { 30 } \@@_scan_int:
        { 31 } \@@_scan_pdfximagebbox:
        { 33 } \@@_scan_directlua:
        { 34 } \@@_scan_pdf_ext_toks:
        { 35 } \@@_scan_pdf_ext_toks:
        { 40 }
          {
            \@@_scan_int:
            \@@_prev_input_silent:n { ~ }
            \@@_scan_int:
          }
      }
    \@@_prev_input_gpop:N \l_@@_head_tl
    \@@_back_input_tl_o:
  }
\cs_new_protected:Npn \@@_convert_string:
  {
    \@@_get_next:
    \tl_if_empty:NTF \l_@@_head_tl
      { \@@_prev_input:x { \gtl_to_str:N \l_@@_head_gtl } }
      { \@@_prev_input:V \l_@@_head_tl }
  }
\cs_new_protected:Npn \@@_convert_meaning:
  {
    \@@_get_next:
    \tl_if_empty:NTF \l_@@_head_tl
      { \@@_prev_input:n { \l_@@_head_token } }
      { \@@_prev_input:V \l_@@_head_tl }
  }
\cs_new_protected:Npn \@@_scan_pdfstrcmp:
  {
    \@@_scan_toks_to_str:
    \@@_scan_toks_to_str:
  }
\cs_new_protected:Npn \@@_scan_pdfximagebbox:
  { \@@_scan_int: \@@_scan_int: }
\cs_new_protected:Npn \@@_scan_pdfcolorstackinit:
  {
    \@@_scan_keyword:nTF { pPaAgGeE }
      { \bool_set_true:N \l_@@_tmpa_bool }
      { \bool_set_false:N \l_@@_tmpb_bool }
    \@@_scan_keyword:nF { dDiIrReEcCtT }
      { \@@_scan_keyword:n { pPaAgGeE } }
    \@@_scan_toks_to_str:
  }
\cs_new_protected:Npn \@@_scan_pdffiledump:
  {
    \@@_scan_keyword:nT { oOfFfFsSeEtT } \@@_scan_int:
    \@@_scan_keyword:nT { lLeEnNgGtThH } \@@_scan_int:
    \@@_scan_pdf_ext_toks:
  }
\cs_new_protected:Npn \@@_scan_pdfmatch:
  {
    \@@_scan_keyword:n { iIcCaAsSeE }
    \@@_scan_keyword:nT { sSuUbBcCoOuUnNtT }
      { \@@_scan_int: }
    \@@_scan_pdf_ext_toks:
    \@@_scan_pdf_ext_toks:
  }
\sys_if_engine_luatex:T
  {
    \cs_new_protected:Npn \@@_scan_directlua:
      {
        \@@_get_x_non_relax:
        \token_if_eq_catcode:NNTF \l_@@_head_token \c_group_begin_token
          { \@@_back_input: }
          {
            \@@_scan_int:
            \@@_get_x_non_relax:
          }
        \@@_scan_pdf_ext_toks:
     }
  }
%    \end{macrocode}
%
% \begin{macro}{\@@_get_the:N}
%   |#1| is \cs{@@_get_token_xdef:} in \tn{edef} or \tn{xdef},
%   \cs{@@_get_token_x:} in \tn{message} and the like, and can be other
%   commands.
%    \begin{macrocode}
\@@_new_tex_expandable:nn { the }                           % 111
  { \@@_get_the:N }
\cs_new_protected:Npn \@@_get_the:N #1
  {
    \@@_prev_input_gpush:N \l_@@_head_tl
    \@@_print_expansion:
    \int_if_odd:nTF \l_@@_head_char_int
      { % \unexpanded, \detokenize
        \@@_scan_toks:NN \c_false_bool \c_false_bool
        \@@_prev_input_gpop:N \l_@@_head_tl
        \@@_set_action_text:x { \tl_to_str:N \l_@@_head_tl }
      }
      { % \the
        \@@_get_x_next:
        \@@_scan_something_internal:n { 5 }
        \@@_prev_input_gpop:N \l_@@_head_tl
        \@@_set_action_text:x
          {
            \tl_head:N \l_@@_head_tl
            => \tl_tail:N \l_@@_head_tl
          }
        \tl_set:Nx \l_@@_head_tl
          { \exp_not:N \exp_not:n { \tl_tail:N \l_@@_head_tl } }
      }
    \cs_if_eq:NNTF #1 \@@_get_token_xdef:
      {
        \tl_put_right:NV \l_@@_defining_tl \l_@@_head_tl
        \@@_prev_input:V \l_@@_head_tl
      }
      {
        \cs_if_eq:NNTF #1 \@@_get_token_x:
          {
            \@@_exp_args:NNx \gtl_set:Nn \l_@@_tmpb_gtl { \l_@@_head_tl }
            \@@_prev_input_gtl:N \l_@@_tmpb_gtl
          }
          {
            \tl_set:Nx \l_@@_tmpa_tl { \exp_args:NV \exp_not:o \l_@@_head_tl }
            \@@_back_input:V \l_@@_tmpa_tl
          }
        \@@_print_expansion:
      }
    #1
  }
%    \end{macrocode}
% \end{macro}
%
%    \begin{macrocode}
\@@_new_tex_expandable:nn { top_bot_mark }                  % 112
  { \@@_back_input_tl_o: }
%    \end{macrocode}
%
%    \begin{macrocode}
\@@_new_tex_expandable:nn { end_template }                  % 117
  {
    \@@_not_implemented:n { end-template } { } { } { }
    \@@_back_input_tl_o:
  }
%    \end{macrocode}
%
%
% \subsubsection{Conditionals}
%
% ^^A todo: simply use \cs{@@_input_gpop:N} right away.
% \begin{macro}{\@@_pass_text:}
% \begin{macro}{\@@_pass_text_done:w}
%    \begin{macrocode}
\cs_new_protected:Npn \@@_pass_text:
  {
    \@@_input_if_empty:TF
      { \@@_pass_text_empty: }
      {
        \@@_input_get:N \l_@@_tmpb_gtl
        \if_true:
          \if_case:w \gtl_head_do:NN \l_@@_tmpb_gtl \c_one
            \exp_after:wN \@@_pass_text_done:w
          \fi:
          \@@_input_gpop:N \l_@@_tmpb_gtl
          \exp_after:wN \@@_pass_text:
        \else:
          \use:c { fi: }
          \int_set_eq:NN \l_@@_if_nesting_int \c_one
          \@@_input_gpop:N \l_@@_tmpb_gtl
          \exp_after:wN \@@_pass_text_nested:
        \fi:
      }
  }
\cs_new_protected:Npn \@@_pass_text_done:w
  {
    \@@_get_next:
    \token_if_eq_meaning:NNT \l_@@_head_token \fi: { \if_true: }
    \else:
  }
%    \end{macrocode}
% \end{macro}
% \end{macro}
%
% \begin{macro}{\@@_pass_text_nested:}
%   Again, if there is no more input we are in trouble.  The
%   construction otherwise essentially results in
%   \begin{quote}
%     \cs{if_true:} \cs{if_true:} \cs{else:} \meta{head} \\
%     \cs{int_decr:N} \cs{l_@@_if_nesting_int} \cs{use_none:nnnnn} \cs{fi:} \\
%     \cs{use_none:nnn} \cs{fi:} \\
%     \cs{int_incr:N} \cs{l_@@_if_nesting_int} \cs{fi:}
%   \end{quote}
%   If the \meta{head} is a primitive |\if...|, then the \cs{if_true:}
%   \cs{else:} ends with the second \cs{fi:}, and the nesting integer is
%   incremented before appropriately closing the \cs{if_true:}.  If it
%   is a normal token or \tn{or} or \tn{else}, \cs{use_none:nnn} cleans
%   up, leaving the appropriate number of \cs{fi:}.  Finally, if it is
%   \cs{fi:}, the nesting integer is decremented before removing most
%   \cs{fi:}.
%    \begin{macrocode}
\cs_new_protected:Npn \@@_pass_text_nested:
  {
    \@@_input_if_empty:TF
      { \@@_pass_text_empty: }
      {
        \@@_input_get:N \l_@@_tmpb_gtl
        \if_true:
          \if_true:
            \gtl_head_do:NN \l_@@_tmpb_gtl \else:
            \int_decr:N \l_@@_if_nesting_int
            \use_none:nnnnn
          \fi:
          \use_none:nnn
        \fi:
        \int_incr:N \l_@@_if_nesting_int
        \fi:
        \@@_input_gpop:N \l_@@_unused_gtl
        \int_compare:nNnTF \l_@@_if_nesting_int = \c_zero
          { \@@_pass_text: }
          { \@@_pass_text_nested: }
      }
  }
%    \end{macrocode}
% \end{macro}
%
% \begin{macro}{\@@_pass_text_empty:}
%    \begin{macrocode}
\cs_new_protected:Npn \@@_pass_text_empty:
  {
    \@@_error:nnnnn { runaway-if } { } { } { } { }
    \@@_exit:w
  }
%    \end{macrocode}
% \end{macro}
%
% \begin{macro}{\@@_cond_push:, \@@_cond_pop:}
%    \begin{macrocode}
\cs_new_protected:Npn \@@_cond_push:
  {
    \tl_gput_left:Nx \g_@@_if_limit_tl
      { { \int_use:N \g_@@_if_limit_int } }
    \int_gincr:N \g_@@_if_depth_int
    \int_gzero:N \g_@@_if_limit_int
  }
\cs_new_protected:Npn \@@_cond_pop:
  {
    \int_gset:Nn \g_@@_if_limit_int
      { \tl_head:N \g_@@_if_limit_tl }
    \tl_gset:Nx \g_@@_if_limit_tl
      { \tl_tail:N \g_@@_if_limit_tl }
    \int_gdecr:N \g_@@_if_depth_int
  }
%    \end{macrocode}
% \end{macro}
%
% \begin{macro}{\@@_change_if_limit:nn}
%    \begin{macrocode}
\cs_new_protected:Npn \@@_change_if_limit:nn #1#2
  {
    \int_compare:nNnTF {#2} = \g_@@_if_depth_int
      { \int_gset:Nn \g_@@_if_limit_int {#1} }
      {
        \tl_clear:N \l_@@_tmpa_tl
        \prg_replicate:nn { \g_@@_if_depth_int - #2 - \c_one }
          {
            \tl_put_right:Nx \l_@@_tmpa_tl
              { { \tl_head:N \g_@@_if_limit_tl } }
            \tl_gset:Nx \g_@@_if_limit_tl
              { \tl_tail:N \g_@@_if_limit_tl }
          }
        \tl_gset:Nx \g_@@_if_limit_tl
          { \l_@@_tmpa_tl {#1} \tl_tail:N \g_@@_if_limit_tl }
      }
  }
%    \end{macrocode}
% \end{macro}
%
%    \begin{macrocode}
\@@_new_tex_expandable:nn { if_test }                       % 107
  {
    \@@_cond_push:
    \exp_args:NV \@@_cond_aux:n \g_@@_if_depth_int
  }
%    \end{macrocode}
%
% \begin{macro}{\@@_cond_aux:nn}
%    \begin{macrocode}
\cs_new_protected:Npn \@@_cond_aux:n #1
  {
    \int_case:nnF \l_@@_head_char_int
      {
        { 12 } { \@@_test_ifx:n {#1} }
        { 16 } { \@@_test_case:n {#1} }
        { 21 } { \@@_test_pdfprimitive:n {#1} } % ^^A todo and \unless
      }
      {
        \@@_prev_input_gpush:N \l_@@_head_tl
        \@@_print_expansion:
        \int_case:nn \l_@@_head_char_int
          {
            {  0 } { \@@_test_two_chars: } % if
            {  1 } { \@@_test_two_chars: } % ifcat
            {  2 } % ifnum
              { \@@_test_two_vals:N \@@_scan_int: }
            {  3 } % ifdim
              { \@@_test_two_vals:N \@@_scan_normal_dimen: }
            {  4 } { \@@_scan_int: } % ifodd
            % {  5 } { } % ifvmode
            % {  6 } { } % ifhmode
            % {  7 } { } % ifmmode
            % {  8 } { } % ifinner
            {  9 } { \@@_scan_int: } % ifvoid
            { 10 } { \@@_scan_int: } % ifhbox
            { 11 } { \@@_scan_int: } % ifvbox
            { 13 } { \@@_scan_int: } % ifeof
            % { 14 } { } % iftrue
            % { 15 } { } % iffalse
            { 17 } { \@@_test_ifdefined: } % ifdefined
            { 18 } { \@@_test_ifcsname: } % ifcsname
            { 19 } % iffontchar
              { \@@_scan_font_ident: \@@_scan_int: }
            % { 20 } { } % ifincsname % ^^A todo: something?
            { 22 } % ifpdfabsnum
              { \@@_test_two_vals:N \@@_scan_int: }
            { 23 } % ifpdfabsdim
              { \@@_test_two_vals:N \@@_scan_normal_dimen: }
          }
        \@@_prev_input_gpop:N \l_@@_head_tl
        \@@_set_action_text:x { \tl_to_str:N \l_@@_head_tl }
        \l_@@_head_tl \scan_stop:
          \exp_after:wN \@@_cond_true:n
        \else:
          \exp_after:wN \@@_cond_false:n
        \fi:
        {#1}
      }
  }
%    \end{macrocode}
% \end{macro}
%
% \begin{macro}{\@@_cond_true:n}
%    \begin{macrocode}
\cs_new_protected:Npn \@@_cond_true:n #1
  {
    \@@_change_if_limit:nn { 3 } {#1} % wait for else/fi
    \@@_print_expansion:x { \g_@@_action_text_str = true }
  }
%    \end{macrocode}
% \end{macro}
%
% \begin{macro}
%   {\@@_cond_false:n, \@@_cond_false_loop:n, \@@_cond_false_common:}
%    \begin{macrocode}
\cs_new_protected:Npn \@@_cond_false:n #1
  {
    \@@_cond_false_loop:n {#1}
    \@@_cond_false_common:
    \@@_print_expansion:x
      {
        \g_@@_action_text_str = false ~
        => ~ skipped ~ to ~ \iow_char:N\\fi
      }
  }
\cs_new_protected:Npn \@@_cond_false_loop:n #1
  {
    \@@_pass_text:
    \int_compare:nNnTF \g_@@_if_depth_int = {#1}
      {
        \token_if_eq_meaning:NNT \l_@@_head_token \or:
          {
            \@@_error:nnnnn { extra-or } { } { } { } { }
            \@@_cond_false_loop:n {#1}
          }
      }
      {
        \token_if_eq_meaning:NNT \l_@@_head_token \fi:
          { \@@_cond_pop: }
        \@@_cond_false_loop:n {#1}
      }
  }
\cs_new_protected:Npn \@@_cond_false_common:
  {
    \token_if_eq_meaning:NNTF \l_@@_head_token \fi:
      { \@@_cond_pop: }
      { \int_gset:Nn \g_@@_if_limit_int { 2 } } % wait for fi
  }
%    \end{macrocode}
% \end{macro}
%
% \begin{macro}{\@@_test_two_vals:N}
%    \begin{macrocode}
\cs_new_protected:Npn \@@_test_two_vals:N #1
  {
    #1
    \@@_get_x_non_blank:
    \@@_tl_if_in:ooTF { < = > } \l_@@_head_tl { }
      {
        \@@_error:nnnnn { missing-equals } { } { } { } { }
        \@@_back_input:
        \tl_set:Nn \l_@@_head_tl { = }
      }
    \@@_prev_input:V \l_@@_head_tl
    #1
  }
%    \end{macrocode}
% \end{macro}
%
% ^^A todo: fix interaction with noexpand and active chars
% ^^A todo: fix non-brace begin-group and end-group tokens
% \begin{macro}{\@@_test_two_chars:, \@@_test_two_chars_aux:}
%    \begin{macrocode}
\cs_new_protected:Npn \@@_test_two_chars:
  {
    \@@_test_two_chars_aux:
    \@@_prev_input:V \l_@@_head_tl
    \@@_test_two_chars_aux:
    \@@_prev_input:V \l_@@_head_tl
  }
\cs_new_protected:Npn \@@_test_two_chars_aux:
  {
    \@@_get_x_next:
    \gtl_if_tl:NF \l_@@_head_gtl
      {
        \tl_set:Nx \l_@@_head_tl
          {
            \gtl_if_head_is_group_begin:NTF \l_@@_head_gtl
              { \c_group_begin_token } { \c_group_end_token }
          }
      }
    \tl_put_left:Nn \l_@@_head_tl { \exp_not:N } % ^^A todo: prettify.
  }
%    \end{macrocode}
% \end{macro}
%
% \begin{macro}{\@@_test_ifx:n, \@@_test_ifx_str:NN, \@@_test_ifx_aux:NNN, \@@_test_ifx_aux:w}
%   The token equal to \tn{ifx} is pushed as a previous input to show an
%   action nicely, then retrieved as \cs{l_@@_tmpa_tl} after getting the
%   next two tokens as |tmpb| and |head|.  Then we call
%   \cs{l_@@_tmpa_tl} followed by these two tokens.  A previous
%   implementation made sure to get these tokens from unpacking the
%   |gtl|, presumably (I should have documented, now I might be missing
%   something) to deal nicely with the master counter in case these
%   tokens are braces.  On the other hand we must take care of tokens
%   affected by \tn{noexpand} and whose current definition is
%   expandable, in which case the trustworthy \tn{meaning} is that of
%   the \cs{l_@@_head_token} or \cs{l_@@_tmpb_token} rather than that of
%   the token in \cs{l_@@_head_gtl} or \cs{l_@@_tmpb_gtl}.
%    \begin{macrocode}
\cs_new_protected:Npn \@@_test_ifx:n #1
  {
    \@@_prev_input_gpush:N \l_@@_head_tl
    \@@_print_expansion:
    \@@_get_next:
    \gtl_set_eq:NN \l_@@_tmpb_gtl \l_@@_head_gtl
    \cs_set_eq:NN \l_@@_tmpb_token \l_@@_head_token
    \@@_get_next:
    \@@_prev_input_gpop:N \l_@@_tmpa_tl
    \@@_set_action_text:x
      {
        Compare:~ \tl_to_str:N \l_@@_tmpa_tl
        \@@_test_ifx_str:NN \l_@@_tmpb_token \l_@@_tmpb_gtl
        \@@_test_ifx_str:NN \l_@@_head_token \l_@@_head_gtl
      }
    \@@_test_ifx_aux:NNN \l_@@_tmpb_token \l_@@_tmpb_gtl
      \@@_test_ifx_aux:w
      \exp_after:wN \@@_cond_true:n
    \else:
      \exp_after:wN \@@_cond_false:n
    \fi:
    {#1}
  }
\cs_new:Npn \@@_test_ifx_str:NN #1#2
  {
    \token_if_eq_meaning:NNT #1 \@@_special_relax:
      { \iow_char:N \\notexpanded: }
    \gtl_to_str:N #2
  }
\cs_new_protected:Npn \@@_test_ifx_aux:NNN #1#2#3
  {
    \token_if_eq_meaning:NNTF #1 \@@_special_relax:
      {
        \gtl_head_do:NN #2 \@@_token_if_expandable:NTF
          { #3 #1 } { \gtl_head_do:NN #2 #3 }
      }
      { \gtl_head_do:NN #2 #3 }
  }
\cs_new:Npn \@@_test_ifx_aux:w
  {
    \@@_test_ifx_aux:NNN \l_@@_head_token \l_@@_head_gtl
      \l_@@_tmpa_tl
  }
%    \end{macrocode}
% \end{macro}
%
% \begin{macro}{\@@_test_case:n, \@@_test_case_aux:nn}
%    \begin{macrocode}
\cs_new_protected:Npn \@@_test_case:n #1
  {
    \@@_prev_input_gpush:N \l_@@_head_tl
    \@@_print_expansion:
    \bool_if:NT \g_@@_internal_debug_bool { \iow_term:n { {\ifcase level~#1} } }
    \@@_scan_int:
    \@@_prev_input_get:N \l_@@_head_tl
    \tl_set:Nx \l_@@_head_tl { \tl_tail:N \l_@@_head_tl }
    % ^^A does text_case_aux use prev_input_seq?
    \exp_args:No \@@_test_case_aux:nn { \l_@@_head_tl } {#1}
    \@@_prev_input_gpop:N \l_@@_head_tl
    \@@_print_expansion:x { \tl_to_str:N \l_@@_head_tl }
  }
\cs_new_protected:Npn \@@_test_case_aux:nn #1#2
  {
    \int_compare:nNnTF {#1} = \c_zero
      { \@@_change_if_limit:nn { 4 } {#2} }
      {
        \@@_pass_text:
        \int_compare:nNnTF \g_@@_if_depth_int = {#2}
          {
            \token_if_eq_meaning:NNTF \l_@@_head_token \or:
              {
                \exp_args:Nf \@@_test_case_aux:nn
                  { \int_eval:n { #1 - 1 } } {#2}
              }
              { \@@_cond_false_common: }
          }
          {
            \token_if_eq_meaning:NNT \l_@@_head_token \fi:
              { \@@_cond_pop: }
            \@@_test_case_aux:nn {#1} {#2}
          }
      }
  }
%    \end{macrocode}
% \end{macro}
%
% \begin{macro}{\@@_test_ifdefined:}
%    \begin{macrocode}
\cs_new_protected:Npn \@@_test_ifdefined:
  {
    \@@_input_if_empty:TF
      { \@@_pass_text_empty: }
      {
        \@@_input_gpop:N \l_@@_tmpb_gtl
        \@@_set_action_text:x
          {
            Conditional:~ \tl_to_str:N \l_@@_head_tl
            \gtl_to_str:N \l_@@_tmpb_gtl
          }
        \@@_prev_input:x
          {
            \gtl_if_tl:NTF \l_@@_tmpb_gtl
              { \gtl_head:N \l_@@_tmpb_gtl }
              { \gtl_to_str:N \l_@@_tmpb_gtl }
          }
      }
  }
%    \end{macrocode}
% \end{macro}
%
% \begin{macro}{\@@_test_ifcsname:}
%    \begin{macrocode}
\cs_new_protected:Npn \@@_test_ifcsname:
  {
    \@@_csname_loop:
    \@@_prev_input:V \l_@@_head_tl
  }
%    \end{macrocode}
% \end{macro}
%
%    \begin{macrocode}
\@@_new_tex_expandable:nn { fi_or_else }                    % 108
  {
    \int_compare:nNnTF \l_@@_head_char_int > \g_@@_if_limit_int
      {
        \int_compare:nNnTF \g_@@_if_limit_int = \c_zero
          {
            \int_compare:nNnTF \g_@@_if_depth_int = \c_zero
              { \@@_error:nnnnn { extra-fi-or-else } { } { } { } { } }
              { \@@_insert_relax: }
          }
          { \@@_error:nnnnn { extra-fi-or-else } { } { } { } { } }
      }
      {
        \@@_set_action_text:
        \int_compare:nNnF \l_@@_head_char_int = \c_two
          {
            \@@_fi_or_else_loop:
            \@@_set_action_text:x
              {
                \g_@@_action_text_str \c_space_tl
                => ~ skipped ~ to ~ \tl_to_str:N \l_@@_head_tl
              }
          }
        % ^^A todo: in the terminal output the token itself is missing.
        \@@_print_expansion:
        \@@_cond_pop:
      }
  }
\cs_new_protected:Npn \@@_fi_or_else_loop:
  {
    \int_compare:nNnF \l_@@_head_char_int = \c_two
      {
        \@@_pass_text:
        \@@_set_cmd:
        \@@_fi_or_else_loop:
      }
  }
%    \end{macrocode}
%
% \subsection{User interaction}
%
% ^^A Not implemented yet: non-hash-doubling version.
%
% \subsubsection{Print}
%
% Let us start with the procedure which prints to the terminal: this
% will help me test the code while I'm writing it.
%
% \begin{macro}{\@@_print_normalize_null:}
% \begin{variable}{\l_@@_print_tl}
%   Change the null character to an explicit |^||^||@| in \LuaTeX{} to
%   avoid a bug whereby a null character ends a string prematurely.
%    \begin{macrocode}
\tl_new:N \l_@@_print_tl
\sys_if_engine_luatex:TF
  {
    \cs_new_protected:Npx \@@_print_normalize_null:
      {
        \tl_replace_all:Nnn \exp_not:N \l_@@_print_tl
          { \char_generate:nn { 0 } { 12 } }
          { \tl_to_str:n { ^ ^ @ } }
      }
  }
  { \cs_new_protected:Npn \@@_print_normalize_null: { } }
%    \end{macrocode}
% \end{variable}
% \end{macro}
%
% \begin{macro}{\@@_print:n, \@@_print:x, \@@_log:n}
%    \begin{macrocode}
\cs_new_protected:Npn \@@_print:n #1
  {
    \tl_set:Nn \l_@@_print_tl {#1}
    \@@_print_normalize_null:
    \@@_exp_args:Nx \iow_term:n { \l_@@_print_tl }
  }
\cs_new_protected:Npn \@@_print:x
  { \@@_exp_args:Nx \@@_print:n }
\cs_new_protected:Npn \@@_log:n #1
  {
    \tl_set:Nn \l_@@_print_tl {#1}
    \@@_print_normalize_null:
    \@@_exp_args:Nx \iow_log:n { \l_@@_print_tl }
  }
%    \end{macrocode}
% \end{macro}
%
% \begin{macro}{\@@_print_message:nn}
%   The message to be printed should come already detokenized, as~|#2|.
%   It will be wrapped to 80 characters per line, with~|#1| before each
%   line.  The message is properly suppressed (or sent only to the log)
%   according to \cs{g_@@_online_int}.
%    \begin{macrocode}
\cs_new_protected:Npn \@@_print_message:nn #1 #2
  {
    \int_compare:nNnF \g_@@_online_int < \c_zero
      {
        \int_compare:nNnTF \g_@@_online_int = \c_zero
          { \iow_wrap:nnnN { #1 #2 } { #1 } { } \@@_log:n }
          { \iow_wrap:nnnN { #1 #2 } { #1 } { } \@@_print:n }
      }
  }
%    \end{macrocode}
% \end{macro}
%
% \begin{macro}{\@@_set_action_text:x}
%    \begin{macrocode}
\cs_new_protected:Npn \@@_set_action_text:x #1
  {
    \group_begin:
      \@@_set_escapechar:n { 92 }
      \str_gset:Nx \g_@@_action_text_str {#1}
    \group_end:
  }
%    \end{macrocode}
% \end{macro}
%
% \begin{macro}{\@@_set_action_text:}
%    \begin{macrocode}
\cs_new_protected:Npn \@@_set_action_text:
  {
    \@@_set_action_text:x
      {
        \tl_to_str:N \l_@@_head_tl
        \tl_if_single_token:VT \l_@@_head_tl
          { = ~ \token_to_meaning:N \l_@@_head_token }
      }
  }
%    \end{macrocode}
% \end{macro}
%
% \begin{macro}{\@@_print_state:}
%    \begin{macrocode}
\cs_new_protected:Npn \@@_print_state:
  {
    \group_begin:
      \@@_set_escapechar:n { 92 }
      \tl_use:N \g_@@_before_print_state_tl
      \int_compare:nNnT \g_@@_online_int > \c_zero
        {
          \@@_print_state_output:
          \@@_print_state_prev:
          \@@_print_state_input:
        }
    \group_end:
  }
%    \end{macrocode}
% \end{macro}
%
% \begin{macro}{\@@_print_state_output:, \@@_print_state_output:n}
%   Unless empty, print |#1| with each line starting with \verb*+<|~+.
%   The \cs{@@_str_truncate_left:nn} function trims |#1| if needed, to
%   fit in a maximum of \cs{g_@@_max_output_int} characters.
%    \begin{macrocode}
\cs_new_protected:Npn \@@_print_state_output:
  {
    \@@_exp_args:Nx \@@_print_state_output:n
      { \gtl_to_str:N \g_@@_output_gtl }
  }
\cs_new_protected:Npn \@@_print_state_output:n #1
  {
    \tl_if_empty:nF {#1}
      {
        \@@_print_message:nn { <| ~ }
          { \@@_str_truncate_left:nn {#1} { \g_@@_max_output_int } }
      }
  }
%    \end{macrocode}
% \end{macro}
%
% \begin{macro}{\@@_print_state_prev:}
%   Never trim~|##1|.
%    \begin{macrocode}
\cs_new_protected:Npn \@@_print_state_prev:
  {
    \seq_set_map:NNn \l_@@_tmpa_seq \g_@@_prev_input_seq
      { \@@_to_str:n {##1} }
    \seq_remove_all:Nn \l_@@_tmpa_seq { }
    \seq_if_empty:NTF \l_@@_tmpa_seq
      { \@@_print_message:nn { || ~ } { } }
      {
        \seq_map_inline:Nn \l_@@_tmpa_seq
          {
            \@@_print_message:nn { || ~ } {##1}
          }
      }
  }
%    \end{macrocode}
% \end{macro}
%
% \begin{macro}{\@@_print_state_input:, \@@_print_state_input:n}
%   Print |#1| with each line starting with \verb*+|>~+.  The
%   \cs{@@_str_truncate_right:nn} function trims |#1| if needed, to fit
%   in a maximum of \cs{g_@@_max_input_int} characters.
%    \begin{macrocode}
\cs_new_protected:Npn \@@_print_state_input:
  {
    \@@_exp_args:Nx \@@_print_state_input:n
      { \@@_input_to_str: }
  }
\cs_new_protected:Npn \@@_print_state_input:n #1
  {
    \@@_print_message:nn { |> ~ }
      { \@@_str_truncate_right:nn {#1} { \g_@@_max_input_int } }
  }
%    \end{macrocode}
% \end{macro}
%
% \begin{macro}{\@@_print_meaning:}
%    \begin{macrocode}
\cs_new_protected:Npn \@@_print_meaning:
  {
    \@@_input_if_empty:TF
      { \@@_print_message:nn { } { Empty~input! } }
      {
        \@@_input_get:N \l_@@_tmpb_gtl
        \@@_print_message:nn { }
          {
            \gtl_head_do:NN \l_@@_tmpb_gtl \token_to_str:N
            = \gtl_head_do:NN \l_@@_tmpb_gtl \token_to_meaning:N
          }
      }
  }
%    \end{macrocode}
% \end{macro}
%
% \begin{macro}{\@@_print_action:, \@@_print_action:x, \@@_print_assignment:, \@@_print_assignment:x, \@@_print_expansion:, \@@_print_expansion:x, \@@_print_action_aux:N}
%   Some of these commands are currently synonyms but we may decide to
%   make some options act differently on them.
%    \begin{macrocode}
\cs_new_protected:Npn \@@_print_action:
  { \@@_print_action_aux:N \g_@@_trace_other_bool }
\cs_new_protected:Npn \@@_print_action:x #1
  {
    \@@_set_action_text:x {#1}
    \@@_print_action:
  }
\cs_new_protected:Npn \@@_print_assignment:
  { \@@_print_action_aux:N \g_@@_trace_assigns_bool }
\cs_new_protected:Npn \@@_print_assignment:x #1
  {
    \@@_set_action_text:x {#1}
    \@@_print_assignment:
  }
\cs_new_protected:Npn \@@_print_expansion:
  { \@@_print_action_aux:N \g_@@_trace_expansion_bool }
\cs_new_protected:Npn \@@_print_expansion:x #1
  {
    \@@_set_action_text:x {#1}
    \@@_print_expansion:
  }
\cs_new_protected:Npn \@@_print_action_aux:N #1
  {
    \int_gdecr:N \g_@@_nonstop_int
    \int_gincr:N \g_@@_step_int
    \bool_if:NT #1
      {
        \@@_print:x
          {
            [=====
            \bool_if:NT \g_@@_number_steps_bool
              { ~ Step ~ \int_to_arabic:n { \g_@@_step_int } ~ }
            =====]~
            \int_compare:nNnTF
              { \str_count:N \g_@@_action_text_str }
              > { \g_@@_max_action_int }
              {
                \str_range:Nnn \g_@@_action_text_str
                  { 1 } { \g_@@_max_action_int - 3 } ...
              }
              { \g_@@_action_text_str }
          }
        \@@_print_state:
        \@@_prompt:
      }
  }
%    \end{macrocode}
% \end{macro}
%
% \begin{macro}{\@@_print_assigned_token:, \@@_print_assigned_register:}
%    \begin{macrocode}
\cs_new_protected:Npn \@@_print_assigned_token:
  {
    \@@_after_assignment: % ^^A todo: simplify
    \@@_print_assignment:x
      {
        Set~ \exp_after:wN \token_to_str:N \l_@@_defined_tl
        = \exp_after:wN \token_to_meaning:N \l_@@_defined_tl
      }
    \@@_omit_after_assignment:w
  }
\cs_new_protected:Npn \@@_print_assigned_register:
  {
    \@@_after_assignment: % ^^A todo: simplify
    \@@_exp_args:Nx \@@_print_assignment:x
      {
        \exp_not:n
          {
            Set~ \exp_after:wN \token_to_str:N \l_@@_defined_tl
            \tl_if_single:NT \l_@@_defined_tl
              { ( \exp_after:wN \token_to_meaning:N \l_@@_defined_tl ) }
          }
        = \exp_not:N \tl_to_str:n { \@@_the:w \l_@@_defined_tl }
      }
    \@@_omit_after_assignment:w
  }
%    \end{macrocode}
% \end{macro}
%
% \begin{macro}{\@@_print_welcome:}
%   Welcome message.
%    \begin{macrocode}
\cs_new_protected:Npn \@@_print_welcome:
  {
    \@@_print_message:nn { }
      {
        \bool_if:NTF \g_@@_welcome_message_bool
          {
            \\
            ========~ Welcome~ to~ the~ unravel~ package~ ========\\
            \iow_indent:n
              {
                "<|"~ denotes~ the~ output~ to~ TeX's~ stomach. \\
                "||"~ denotes~ tokens~ waiting~ to~ be~ used. \\
                "|>"~ denotes~ tokens~ that~ we~ will~ act~ on. \\
                Press~<enter>~to~continue;~'h'~<enter>~for~help. \\
              }
          }
          { [=====~Start~=====] }
      }
    \@@_print_state:
    \@@_prompt:
  }
%    \end{macrocode}
% \end{macro}
%
% \begin{macro}{\@@_print_outcome:}
%   Final message.
%    \begin{macrocode}
\cs_new_protected:Npn \@@_print_outcome:
  { \@@_print_message:nn { } { [=====~End~=====] } }
%    \end{macrocode}
% \end{macro}
%
% \subsubsection{Prompt}
%
% \begin{macro}{\@@_prompt:}
%    \begin{macrocode}
\cs_new_protected:Npn \@@_prompt:
  {
    \int_compare:nNnF \g_@@_nonstop_int > \c_zero
      {
        \group_begin:
          \@@_set_escapechar:n { -1 }
          \int_set:Nn \tex_endlinechar:D { -1 }
          \tl_use:N \g_@@_before_prompt_tl
          \@@_prompt_aux:
        \group_end:
      }
  }
\cs_new_protected:Npn \@@_prompt_aux:
  {
    \int_compare:nNnT { \etex_interactionmode:D } = { 3 }
      {
        \bool_if:NTF \g_@@_explicit_prompt_bool
          { \ior_str_get:Nc \c_@@_prompt_ior }
          { \ior_str_get:Nc \c_@@_noprompt_ior }
              { Your~input }
        \exp_args:Nv \@@_prompt_treat:n { Your~input }
      }
  }
\cs_new_protected:Npn \@@_prompt_treat:n #1
  {
    \tl_if_empty:nF {#1}
      {
        \@@_exp_args:Nx \str_case:nnF { \tl_head:n {#1} }
          {
            { m } { \@@_print_meaning: \@@_prompt_aux: }
            { q }
              {
                \int_gset:Nn \g_@@_online_int { -1 }
                \int_gzero:N \g_@@_nonstop_int
              }
            { x }
              {
                \group_end:
                \exp_after:wN \@@_exit:w \@@_exit:w
              }
            { X } { \tex_batchmode:D \tex_end:D }
            { s } { \@@_prompt_scan_int:nn {#1}
              \@@_prompt_silent_steps:n }
            { o } { \@@_prompt_scan_int:nn {#1}
              { \int_gset:Nn \g_@@_online_int } }
            { C }
              {
                \@@_exp_args:Nx \use:n
                  {
                    \tl_gset_rescan:Nnn \exp_not:N \g_@@_tmpc_tl
                      { \exp_not:N \ExplSyntaxOn } { \tl_tail:n {#1} }
                  }
                \tl_gput_left:Nn \g_@@_tmpc_tl
                  { \tl_gclear:N \g_@@_tmpc_tl }
                \group_insert_after:N \g_@@_tmpc_tl
              }
            { | } { \@@_prompt_scan_int:nn {#1}
              \@@_prompt_vert:n }
            { a } { \@@_prompt_all: }
          }
          { \@@_prompt_help: }
      }
  }
\cs_new_protected:Npn \@@_prompt_scan_int:nn #1
  {
    \tex_afterassignment:D \@@_prompt_scan_int_after:wn
    \l_@@_prompt_tmpa_int =
      \tl_if_head_eq_charcode:fNF { \use_none:n #1 } - { 0 }
      \use_ii:nn #1 \scan_stop:
  }
\cs_new_protected:Npn \@@_prompt_scan_int_after:wn #1 \scan_stop: #2
  {
    #2 \l_@@_prompt_tmpa_int
    \tl_if_blank:nF {#1} { \@@_prompt_treat:n {#1} }
  }
\cs_new_protected:Npn \@@_prompt_help:
  {
    \@@_print:n { "m":~meaning~of~first~token }
    \@@_print:n { "q":~semi-quiet~(same~as~"o-1") }
    \@@_print:n { "x"/"X":~exit~this~instance~of~unravel/TeX }
    \@@_print:n { "s<num>":~do~<num>~steps~silently }
    \@@_print:n
      { "o<num>":~1~=>~log~and~terminal,~0~=>~only~log,~-1~=>~neither.}
    \@@_print:n { "C<code>":~run~some~expl3~code~immediately }
    \@@_print:n { "|<num>":~silent~steps~until~<num>~fewer~"||" }
    \@@_print:n { "a":~print~state~again,~without~truncating }
    \@@_prompt_aux:
  }
\cs_new_protected:Npn \@@_prompt_silent_steps:n #1
  {
    \int_compare:nNnF {#1} < \c_zero
      {
        \int_gset:Nn \g_@@_online_int { -1 }
        \tl_gset:Nn \g_@@_before_prompt_tl
          {
            \int_gset_eq:NN \g_@@_online_int \c_one
            \tl_gclear:N \g_@@_before_prompt_tl
          }
        \int_gset:Nn \g_@@_nonstop_int {#1}
      }
  }
\cs_new_protected:Npn \@@_prompt_vert:n #1
  {
    \int_compare:nNnTF {#1} < { 0 }
      { \@@_prompt_vert:Nn > {#1} }
      { \@@_prompt_vert:Nn < {#1} }
  }
\cs_new_protected:Npn \@@_prompt_vert:Nn #1#2
  {
    \int_gset:Nn \g_@@_online_int { -1 }
    \tl_gset:Nf \g_@@_before_print_state_tl
      {
        \exp_args:NNf \exp_stop_f: \int_compare:nNnTF
          { \int_eval:n { \@@_prev_input_count: - #2 } }
          #1 { \@@_prev_input_count: }
          {
            \int_gset:Nn \g_@@_nonstop_int
              { \int_max:nn { \g_@@_nonstop_int } { 2 } }
          }
          {
            \int_gset_eq:NN \g_@@_online_int \c_one
            \tl_gclear:N \g_@@_before_print_state_tl
          }
      }
  }
\cs_new_protected:Npn \@@_prompt_all:
  {
    \tl_gset:Nx \g_@@_tmpc_tl
      {
        \exp_not:n
          {
            \tl_gclear:N \g_@@_tmpc_tl
            \int_gset_eq:NN \g_@@_max_output_int \c_max_int
            \int_gset_eq:NN \g_@@_max_input_int \c_max_int
            \@@_print_state:
            \int_gdecr:N \g_@@_nonstop_int
            \@@_prompt:
          }
        \@@_prompt_all_aux:N \g_@@_max_output_int
        \@@_prompt_all_aux:N \g_@@_max_input_int
      }
    \group_insert_after:N \g_@@_tmpc_tl
  }
\cs_new:Npn \@@_prompt_all_aux:N #1
  { \exp_not:n { \int_gset:Nn #1 } { \int_use:N #1 } }
%    \end{macrocode}
% \end{macro}
%
% ^^A todo: I suspect a bug if given 'a' twice in a row
%
% \subsubsection{Errors}
%
% \begin{macro}{\@@_not_implemented:n}
%    \begin{macrocode}
\cs_new_protected:Npn \@@_not_implemented:n #1
  { \@@_error:nnnnn { not-implemented } {#1} { } { } { } }
%    \end{macrocode}
% \end{macro}
%
% \begin{macro}{\@@_error:nnnnn, \@@_error:nxxxx}
%   Errors within a group to make sure that none of the \pkg{l3msg}
%   variables (or others) that may be currently in use in the code being
%   debugged are modified.
%    \begin{macrocode}
\cs_new_protected:Npn \@@_error:nnnnn #1#2#3#4#5
  {
    \group_begin:
    \msg_error:nnnnnn { unravel } {#1} {#2} {#3} {#4} {#5}
    \group_end:
  }
\cs_new_protected:Npn \@@_error:nxxxx #1#2#3#4#5
  {
    \group_begin:
    \msg_error:nnxxxx { unravel } {#1} {#2} {#3} {#4} {#5}
    \group_end:
  }
%    \end{macrocode}
% \end{macro}
%
% \begin{macro}{\@@_tex_msg_new:nnn}
%   This stores a \TeX{} error message.
%    \begin{macrocode}
\cs_new_protected:Npn \@@_tex_msg_new:nnn #1#2#3
  {
    \cs_new:cpn { @@_tex_msg_error_#1: } {#2}
    \cs_new:cpn { @@_tex_msg_help_#1: } {#3}
  }
%    \end{macrocode}
% \end{macro}
%
% \begin{macro}{\@@_tex_error:nn, \@@_tex_error:nV}
%   Throw the |tex-error| message, with arguments: |#2| which triggered
%   the error, \TeX{}'s error message, and \TeX{}'s help text.
%    \begin{macrocode}
\cs_new_protected:Npn \@@_tex_error:nn #1#2
  {
    \group_begin:
    \msg_error:nnxxx { unravel } { tex-error }
      { \tl_to_str:n {#2} }
      { \use:c { @@_tex_msg_error_#1: } }
      { \use:c { @@_tex_msg_help_#1: } }
    \group_end:
  }
\cs_generate_variant:Nn \@@_tex_error:nn { nV }
%    \end{macrocode}
% \end{macro}
%
% \begin{macro}{\@@_tex_fatal_error:nn, \@@_tex_fatal_error:nV}
%   Throw the |tex-fatal| error message, with arguments: |#2| which
%   triggered the fatal error, \TeX{}'s error message, and \TeX{}'s
%   help text.
%    \begin{macrocode}
\cs_new_protected:Npn \@@_tex_fatal_error:nn #1#2
  {
    \@@_error:nxxxx { tex-fatal }
      { \tl_to_str:n {#2} }
      { \use:c { @@_tex_msg_error_#1: } }
      { \use:c { @@_tex_msg_help_#1: } }
      { }
  }
\cs_generate_variant:Nn \@@_tex_fatal_error:nn { nV }
%    \end{macrocode}
% \end{macro}
%
% \subsection{Keys}
%
% Each key needs to be defined twice: for its default setting and for
% its setting applying to a single \cs{unravel}.  This is due to the
% fact that we cannot use grouping to keep settings local to a single
% \cs{unravel} since the \meta{code} argument of \cs{unravel} may open
% or close groups.
%    \begin{macrocode}
\keys_define:nn { unravel/defaults }
  {
    explicit-prompt  .bool_gset:N = \g_@@_default_explicit_prompt_bool ,
    internal-debug   .bool_gset:N = \g_@@_default_internal_debug_bool ,
    max-action       .int_gset:N  = \g_@@_default_max_action_int ,
    max-output       .int_gset:N  = \g_@@_default_max_output_int ,
    max-input        .int_gset:N  = \g_@@_default_max_input_int ,
    number-steps     .bool_gset:N = \g_@@_default_number_steps_bool ,
    online           .int_gset:N  = \g_@@_default_online_int ,
    trace-assigns    .bool_gset:N = \g_@@_default_trace_assign_bool ,
    trace-expansion  .bool_gset:N = \g_@@_default_trace_expansion_bool ,
    trace-other      .bool_gset:N = \g_@@_default_trace_other_bool ,
    welcome-message  .bool_gset:N = \g_@@_default_welcome_message_bool ,
  }
\keys_define:nn { unravel }
  {
    explicit-prompt  .bool_gset:N = \g_@@_explicit_prompt_bool ,
    internal-debug   .bool_gset:N = \g_@@_internal_debug_bool ,
    max-action       .int_gset:N  = \g_@@_max_action_int ,
    max-output       .int_gset:N  = \g_@@_max_output_int ,
    max-input        .int_gset:N  = \g_@@_max_input_int ,
    number-steps     .bool_gset:N = \g_@@_number_steps_bool ,
    online           .int_gset:N  = \g_@@_online_int ,
    trace-assigns    .bool_gset:N = \g_@@_trace_assigns_bool ,
    trace-expansion  .bool_gset:N = \g_@@_trace_expansion_bool ,
    trace-other      .bool_gset:N = \g_@@_trace_other_bool ,
    welcome-message  .bool_gset:N = \g_@@_welcome_message_bool ,
  }
%    \end{macrocode}
%
% The |machine| and |trace| options are somewhat special so it is
% clearer to define them separately.  The code is identical for
% |unravel/defaults| and |unravel| keys.  To be sure of which options
% are set, use |.meta:nn| and give the path explicitly.
%    \begin{macrocode}
\tl_map_inline:nn { { /defaults } { } }
  {
    \keys_define:nn { unravel #1 }
      {
        machine        .meta:nn =
          { unravel #1 }
          {
            explicit-prompt = false ,
            internal-debug  = false ,
            max-action      = \c_max_int ,
            max-output      = \c_max_int ,
            max-input       = \c_max_int ,
            number-steps    = false ,
            welcome-message = false ,
          } ,
        mute          .meta:nn =
          { unravel #1 }
          {
            trace-assigns = false ,
            trace-expansion = false ,
            trace-other = false ,
            welcome-message = false ,
            online = -1 ,
          }
      }
  }
%    \end{macrocode}
%
% \subsection{Main command}
%
% \begin{macro}{\unravel}
%   Simply call an underlying code-level command.
%    \begin{macrocode}
\NewDocumentCommand \unravel { O { } +m } { \unravel:nn {#1} {#2} }
%    \end{macrocode}
% \end{macro}
%
% \begin{macro}{\unravelsetup}
%   Simply call an underlying code-level command.
%    \begin{macrocode}
\NewDocumentCommand \unravelsetup { m } { \unravel_setup:n {#1} }
%    \end{macrocode}
% \end{macro}
%
% \begin{macro}{\unravel_setup:n}
%   Set keys, updating both default values and current values.
%    \begin{macrocode}
\cs_new_protected:Npn \unravel_setup:n #1
  {
    \keys_set:nn { unravel/defaults } {#1}
    \keys_set:nn { unravel } {#1}
  }
%    \end{macrocode}
% \end{macro}
%
% \begin{macro}{\unravel:nn, \@@:nn, \@@_unravel_marker:}
%   The command starts with \cs{@@_unravel_marker:} to detect nesting of
%   \tn{unravel} in \tn{unravel} and avoid re-initializing important
%   variables.
%   Initialize and setup keys.  Initialize and setup other variables
%   including the input.  Welcome the user.  Then comes the main loop:
%   until the input is exhausted, print the current status and do one
%   step.  The main loop is exited by skipping to the first
%   \cs{@@_exit_point:}, while some abort procedures jump to the second
%   (and last) one instead.  If the main loop finished correctly, print
%   its outcome and finally test that everything is all right.
%    \begin{macrocode}
\cs_new_protected:Npn \unravel:nn { \@@_unravel_marker: \@@:nn }
\cs_new_eq:NN \@@_unravel_marker: \@@_special_relax:
\cs_new_protected:Npn \@@:nn #1#2
  {
    \@@_init_key_vars:
    \keys_set:nn { unravel } {#1}
    \@@_init_vars:
    \@@_input_gset:n {#2}
    \@@_print_welcome:
    \@@_main_loop:
    \@@_exit_point:
    \@@_print_outcome:
    \@@_final_test:
    \@@_exit_point:
  }
\cs_new_protected:Npn \unravel_get:nnN #1#2#3
  {
    \unravel:nn {#1} {#2}
    \tl_set:Nx #3 { \gtl_left_tl:N \g_@@_output_gtl }
  }
%    \end{macrocode}
% \end{macro}
%
% \begin{macro}{\@@_init_key_vars:}
%   Give variables that are affected by keys their default values (also
%   controlled by keys).
%    \begin{macrocode}
\cs_new_protected:Npn \@@_init_key_vars:
  {
    \bool_gset_eq:NN \g_@@_explicit_prompt_bool \g_@@_default_explicit_prompt_bool
    \bool_gset_eq:NN \g_@@_internal_debug_bool \g_@@_default_internal_debug_bool
    \bool_gset_eq:NN \g_@@_number_steps_bool \g_@@_default_number_steps_bool
    \int_gset_eq:NN  \g_@@_online_int \g_@@_default_online_int
    \bool_gset_eq:NN \g_@@_trace_assigns_bool \g_@@_default_trace_assigns_bool
    \bool_gset_eq:NN \g_@@_trace_expansion_bool \g_@@_default_trace_expansion_bool
    \bool_gset_eq:NN \g_@@_trace_other_bool \g_@@_default_trace_other_bool
    \bool_gset_eq:NN \g_@@_welcome_message_bool \g_@@_default_welcome_message_bool
    \int_gset_eq:NN \g_@@_max_action_int \g_@@_default_max_action_int
    \int_gset_eq:NN \g_@@_max_output_int \g_@@_default_max_output_int
    \int_gset_eq:NN \g_@@_max_input_int  \g_@@_default_max_input_int
    \int_gzero:N \g_@@_nonstop_int
  }
%    \end{macrocode}
% \end{macro}
%
% \begin{macro}{\@@_init_vars:}
%   Give initial values to variables used during the processing.  These
%   have no reason to be modified by the user: neither directly nor
%   through keys.
%    \begin{macrocode}
\cs_new_protected:Npn \@@_init_vars:
  {
    \seq_gclear:N \g_@@_prev_input_seq
    \gtl_gclear:N \g_@@_output_gtl
    \int_gzero:N  \g_@@_step_int
    \tl_gclear:N  \g_@@_if_limit_tl
    \int_gzero:N  \g_@@_if_limit_int
    \int_gzero:N  \g_@@_if_depth_int
    \gtl_gclear:N \g_@@_after_assignment_gtl
    \bool_gset_true:N  \g_@@_set_box_allowed_bool
    \bool_gset_false:N \g_@@_name_in_progress_bool
    \gtl_clear:N \l_@@_after_group_gtl
  }
%    \end{macrocode}
% \end{macro}
%
% \begin{macro}{\@@_main_loop:}
%   Loop forever, getting the next token (with expansion) and
%   performing the corresponding command.
%    \begin{macrocode}
\cs_new_protected:Npn \@@_main_loop:
  {
    \@@_get_x_next:
    \@@_set_cmd:
    \@@_do_step:
    \@@_main_loop:
  }
%    \end{macrocode}
% \end{macro}
%
% ^^A todo: improve error message
% \begin{macro}{\@@_final_test:, \@@_final_bad:}
%   Make sure that the \cs{unravel} finished correctly.  The error
%   message is a bit primitive.
%    \begin{macrocode}
\cs_new_protected:Npn \@@_final_test:
  {
    \bool_if:nTF
      {
        \tl_if_empty_p:N \g_@@_if_limit_tl
        && \int_compare_p:nNn \g_@@_if_limit_int = \c_zero
        && \int_compare_p:nNn \g_@@_if_depth_int = \c_zero
        && \seq_if_empty_p:N \g_@@_prev_input_seq
      }
      { \@@_input_if_empty:TF { } { \@@_final_bad: } }
      { \@@_final_bad: }
  }
\cs_new_protected:Npn \@@_final_bad:
  {
    \@@_error:nnnnn { internal }
      { the-last-unravel-finished-badly } { } { } { }
  }
%    \end{macrocode}
% \end{macro}
%
% \subsection{Messages}
%
%    \begin{macrocode}
\msg_new:nnn { unravel } { unknown-primitive }
  { Internal~error:~the~primitive~'#1'~is~not~known. }
\msg_new:nnn { unravel } { extra-fi-or-else }
  { Extra~fi,~or,~or~else. }
\msg_new:nnn { unravel } { missing-dollar }
  { Missing~dollar~inserted. }
\msg_new:nnn { unravel } { unknown-expandable }
  { Internal~error:~the~expandable~command~'#1'~is~not~known. }
\msg_new:nnn { unravel } { missing-font-id }
  { Missing~font~identifier.~\iow_char:N\\nullfont~inserted. }
\msg_new:nnn { unravel } { missing-rparen }
  { Missing~right~parenthesis~inserted~for~expression. }
\msg_new:nnn { unravel } { missing-cs }
  { Missing~control~sequence.~\iow_char:N\\inaccessible~inserted. }
\msg_new:nnn { unravel } { missing-box }
  { Missing~box~inserted. }
\msg_new:nnn { unravel } { missing-to }
  { Missing~keyword~'to'~inserted. }
\msg_new:nnn { unravel } { improper-leaders }
  { Leaders~not~followed~by~proper~glue. }
\msg_new:nnn { unravel } { extra-close }
  { Extra~right~brace~or~\iow_char:N\\endgroup. }
\msg_new:nnn { unravel } { off-save }
  { Something~is~wrong~with~groups. }
\msg_new:nnn { unravel } { hrule-bad-mode }
  { \iow_char\\hrule~used~in~wrong~mode. }
\msg_new:nnn { unravel } { invalid-mode }
  { Invalid~mode~for~this~command. }
\msg_new:nnn { unravel } { color-stack-action-missing }
  { Missing~color~stack~action. }
\msg_new:nnn { unravel } { action-type-missing }
  { Missing~action~type. }
\msg_new:nnn { unravel } { identifier-type-missing }
  { Missing~identifier~type. }
\msg_new:nnn { unravel } { destination-type-missing }
  { Missing~destination~type. }
\msg_new:nnn { unravel } { erroneous-prefixes }
  { Prefixes~appplied~to~non-assignment~command. }
\msg_new:nnn { unravel } { improper-setbox }
  { \iow_char:N\\setbox~while~fetching~base~of~an~accent. }
\msg_new:nnn { unravel } { after-advance }
  {
    Missing~register~after~\iow_char:N\\advance,~
    \iow_char:N\\multiply,~or~\iow_char:N\\divide.
  }
\msg_new:nnn { unravel } { bad-unless }
  { \iow_char:N\\unless~not~followed~by~conditional. }
\msg_new:nnn { unravel } { runaway-if }
  { Runaway~\iow_char:N\\if... }
\msg_new:nnn { unravel } { runaway-macro-parameter }
  {
    Runaway~macro~parameter~\# #2~after \\\\
    \iow_indent:n {#1}
  }
\msg_new:nnn { unravel } { extra-or }
  { Extra~\iow_char:N\\or. }
\msg_new:nnn { unravel } { missing-equals }
  { Missing~equals~for~\iow_char:N\\ifnum~or~\iow_char:N\\ifdim. }
\msg_new:nnn { unravel } { internal }
  { Internal~error:~'#1'.~\ Please~report. }
\msg_new:nnn { unravel } { not-implemented }
  { The~following~feature~is~not~implemented:~'#1'. }
\msg_new:nnn { unravel } { endinput-ignored }
  { The~primitive~\iow_char:N\\endinput~was~ignored. }
\msg_new:nnn { unravel } { missing-something }
  { Something~is~missing,~sorry! }
\msg_new:nnn { unravel } { nested-unravel }
  { The~\iow_char:N\\unravel~command~may~not~be~nested. }
\msg_new:nnnn { unravel } { tex-error }
  { TeX~sees~"#1"~and~throws~an~error:\\\\ \iow_indent:n {#2} }
  {
    \tl_if_empty:nTF {#3}
      { TeX~provides~no~further~help~for~this~error. }
      { TeX's~advice~is:\\\\ \iow_indent:n {#3} }
  }
\msg_new:nnnn { unravel } { tex-fatal }
  { TeX~sees~"#1"~and~throws~a~fatal~error:\\\\ \iow_indent:n {#2} }
  {
    \tl_if_empty:nTF {#3}
      { TeX~provides~no~further~help~for~this~error. }
      { TeX's~advice~is:\\\\ \iow_indent:n {#3} }
  }
%    \end{macrocode}
%
% Some error messages from \TeX{} itself.
%    \begin{macrocode}
\@@_tex_msg_new:nnn { incompatible-mag }
  {
    Incompatible~magnification~
    ( \int_to_arabic:n { \@@_mag: } );~
    the~previous~value~will~be~retained
  }
  {
    I~can~handle~only~one~magnification~ratio~per~job.~So~I've~
    reverted~to~the~magnification~you~used~earlier~on~this~run.
  }
\@@_tex_msg_new:nnn { illegal-mag }
  {
    Illegal~magnification~has~been~changed~to~1000~
    ( \int_to_arabic:n { \@@_mag: } )
  }
  { The~magnification~ratio~must~be~between~1~and~32768. }
\@@_tex_msg_new:nnn { missing-number }
  { Missing~number,~treated~as~zero }
  {
    A~number~should~have~been~here;~I~inserted~`0'.~
    If~you~can't~figure~out~why~I~needed~to~see~a~number,~
    look~up~`weird~error'~in~the~index~to~The~TeXbook.
  }
\@@_tex_msg_new:nnn { the-cannot }
  { You~can't~use~`\tl_to_str:N\l_@@_head_tl'~after~\iow_char:N\\the }
  { I'm~forgetting~what~you~said~and~using~zero~instead. }
\@@_tex_msg_new:nnn { incompatible-units }
  { Incompatible~glue~units }
  { I'm~going~to~assume~that~1mu=1pt~when~they're~mixed. }
\@@_tex_msg_new:nnn { missing-mu }
  { Illegal~unit~of~measure~(mu~inserted) }
  {
    The~unit~of~measurement~in~math~glue~must~be~mu.~
    To~recover~gracefully~from~this~error,~it's~best~to~
    delete~the~erroneous~units;~e.g.,~type~`2'~to~delete~
    two~letters.~(See~Chapter~27~of~The~TeXbook.)
  }
\@@_tex_msg_new:nnn { missing-pt }
  { Illegal~unit~of~measure~(pt~inserted) }
  {
    Dimensions~can~be~in~units~of~em,~ex,~in,~pt,~pc,~
    cm,~mm,~dd,~cc,~nd,~nc,~bp,~or~sp;~but~yours~is~a~new~one!~
    I'll~assume~that~you~meant~to~say~pt,~for~printer's~points.~
    To~recover~gracefully~from~this~error,~it's~best~to~
    delete~the~erroneous~units;~e.g.,~type~`2'~to~delete~
    two~letters.~(See~Chapter~27~of~The~TeXbook.)
  }
\@@_tex_msg_new:nnn { missing-lbrace }
  { Missing~\iow_char:N\{~inserted }
  {
    A~left~brace~was~mandatory~here,~so~I've~put~one~in.~
    You~might~want~to~delete~and/or~insert~some~corrections~
    so~that~I~will~find~a~matching~right~brace~soon.~
    (If~you're~confused~by~all~this,~try~typing~`I\iow_char:N\}'~now.)
  }
\@@_tex_msg_new:nnn { extra-endcsname }
  { Extra~\token_to_str:c{endcsname} }
  { I'm~ignoring~this,~since~I~wasn't~doing~a~\token_to_str:c{csname}. }
\@@_tex_msg_new:nnn { missing-endcsname }
  { Missing~\token_to_str:c{endcsname}~inserted }
  {
    The~control~sequence~marked~<to~be~read~again>~should~
    not~appear~between~\token_to_str:c{csname}~and~
    \token_to_str:c{endcsname}.
  }
%    \end{macrocode}
%
% Fatal \TeX{} error messages.
%    \begin{macrocode}
\@@_tex_msg_new:nnn { cannot-read }
  { ***~(cannot~\iow_char:N\\read~from~terminal~in~nonstop~modes) }
  { }
\@@_tex_msg_new:nnn { file-error }
  { ***~(job~aborted,~file~error~in~nonstop~mode) }
  { }
\@@_tex_msg_new:nnn { interwoven-preambles }
  { (interwoven~alignment~preambles~are~not~allowed) }
  { }
%    \end{macrocode}
%
% Restore catcodes to their original values.
%    \begin{macrocode}
\@@_setup_restore:
%    \end{macrocode}
%
%    \begin{macrocode}
%</package>
%    \end{macrocode}
%
% \end{implementation}
%
% \endinput
