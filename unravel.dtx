% \iffalse
%% File: unravel.dtx Copyright (C) 2013 Bruno Le Floch
%%
%% It may be distributed and/or modified under the conditions of the
%% LaTeX Project Public License (LPPL), either version 1.3c of this
%% license or (at your option) any later version.  The latest version
%% of this license is in the file
%%
%%    http://www.latex-project.org/lppl.txt
%%
%% -----------------------------------------------------------------------
%
%<*driver>
%\fi
%\iffalse
\documentclass[full]{l3doc}
\usepackage{unravel}
\begin{document}
  \DocInput{unravel.dtx}
\end{document}
%</driver>
% \fi
%
% \title{The \textsf{unravel} package: \\
%   watching TeX digest tokens\thanks{This
%     file has version number 0.0a, last revised 2013/07/14.}}
% \author{Bruno Le Floch}
% \date{2013/07/14}
%
% \maketitle
% \tableofcontents
%
% \begin{documentation}
%
% \section{\pkg{unravel} documentation}
%
% \subsection{Future perhaps}
%
% \end{documentation}
%
% \begin{implementation}
%
% \section{\pkg{unravel} implementation}
%
% Some support packages are loaded first, then we declare the package's
% name, date, version, and purpose.
%
%    \begin{macrocode}
%<*package>
%    \end{macrocode}
%
%    \begin{macrocode}
\RequirePackage{expl3}[2013/07/01]
\RequirePackage{xtl}[2013/07/14]
\ProvidesExplPackage
  {unravel} {2013/07/14} {0.0a} {Watching TeX digest tokens}
%    \end{macrocode}
%
%    \begin{macrocode}
%<@@=unravel>
%    \end{macrocode}
%
% \subsection{Variables}
%
% \begin{variable}{\g_@@_input_xtl}
%   The user input, in various stages of expansion.
%    \begin{macrocode}
\xtl_new:N \g_@@_input_xtl
%    \end{macrocode}
% \end{variable}
%
% \begin{variable}{\g_@@_output_xtl}
%   Material that is ``typeset'' or otherwise sent further down \TeX{}'s
%   digestion.
%    \begin{macrocode}
\xtl_new:N \g_@@_output_xtl
%    \end{macrocode}
% \end{variable}
%
% \begin{variable}{\l_@@_head_xtl, \l_@@_head_tl, \l_@@_head_meaning_tl}
%   First token in the input, as an extended token list (general case)
%   or as a token list whenever this is possible.  Also, its meaning.
%    \begin{macrocode}
\xtl_new:N \l_@@_head_xtl
\tl_new:N  \l_@@_head_tl
\tl_new:N  \l_@@_head_meaning_tl
%    \end{macrocode}
% \end{variable}
%
% \begin{variable}{\l_@@_head_i_xtl, \l_@@_head_ii_xtl}
%   Extended token lists used when the first few tokens need to be inspected.
%    \begin{macrocode}
\xtl_new:N \l_@@_head_i_xtl
\xtl_new:N \l_@@_head_ii_xtl
%    \end{macrocode}
% \end{variable}
%
% \begin{variable}{\g_@@_step_int}
%   Current expansion step.
%    \begin{macrocode}
\int_new:N \g_@@_step_int
%    \end{macrocode}
% \end{variable}
%
% \begin{variable}{\l_@@_action_tl}
%   String that characterizes which action to perform, depending on the
%   meaning of the first token in the token list.
%    \begin{macrocode}
\tl_new:N \l_@@_action_tl
%    \end{macrocode}
% \end{variable}
%
% \begin{variable}{\l_@@_action_text_tl}
%   Text describing the action, displayed at each step.
%    \begin{macrocode}
\tl_new:N \l_@@_action_text_tl
%    \end{macrocode}
% \end{variable}
%
% \begin{variable}
%   {
%     \l_@@_if_nesting_int,
%     \l_@@_if_case_int,
%     \l_@@_if_a_int,
%     \l_@@_if_b_int
%   }
%    \begin{macrocode}
\int_new:N \l_@@_if_nesting_int
\int_new:N \l_@@_if_case_int
\int_new:N \l_@@_if_a_int
\int_new:N \l_@@_if_b_int
%    \end{macrocode}
% \end{variable}
%
% \begin{variable}{\l_@@_cmp_tl}
%   Comparison operator (for \tn{ifnum} or \tn{ifdim}).
%    \begin{macrocode}
\tl_new:N \l_@@_cmp_tl
%    \end{macrocode}
% \end{variable}
%
% \subsection{Variants and helper functions}
%
% Variants that we need.
%    \begin{macrocode}
\cs_generate_variant:Nn \tl_if_head_is_space:nTF { o }
\cs_if_exist:NF \exp_last_unbraced:NNn
  { \cs_new_eq:NN \exp_last_unbraced:NNn \use:nnn }
\cs_generate_variant:Nn \exp_last_unbraced:NNn { NNv }
\cs_generate_variant:Nn \xtl_gput_left:Nn { Nx , NV }
\cs_generate_variant:Nn \xtl_gput_right:Nn { Nx }
\cs_generate_variant:Nn \tl_if_head_is_N_type:nTF { V }
\cs_generate_variant:Nn \tl_if_head_is_space:nT { V }
%    \end{macrocode}
%
% \begin{macro}[EXP]{\@@_cs_case:Nnn}
%   Currently, \texttt{expl3} does not provide a case statement for the
%   meaning of control sequences.  However, \cs{tl_case:Nnn} in fact
%   does precisely this.
%    \begin{macrocode}
\cs_new_eq:NN \@@_cs_case:Nnn \tl_case:Nnn
%    \end{macrocode}
% \end{macro}
%
% \begin{macro}[int]{\@@_tl_pop_reg:NN}
% \begin{macro}[aux]{\@@_pop_reg_aux:w}
%   Retrieve an integer from a token list.
%    \begin{macrocode}
\cs_new_protected:Npn \@@_tl_pop_reg:NN #1#2
  {
    \cs_set_protected_nopar:Npn \@@_pop_reg_aux:w
      {
        \tex_edef:D #1
          { \etex_unexpanded:D \exp_after:wN { \if_false: } } \fi:
      }
    \tex_afterassignment:D \@@_pop_reg_aux:w
    \if_false: { { \fi: #2 = #1 } }
  }
\cs_new_protected_nopar:Npn \@@_pop_reg_aux:w { }
%    \end{macrocode}
% \end{macro}
% \end{macro}
%
% \begin{macro}[int]{\@@_tl_get_nsp:NN}
%    \begin{macrocode}
\cs_new_protected:Npn \@@_tl_get_nsp:NN #1#2
  {
    \tl_if_head_is_N_type:VTF #1
      {
        \exp_last_unbraced:Nx \token_if_expandable:NTF { \tl_head:N #1 }
          {
            \tl_set:Nx #1 { \exp_args:NV \exp_not:o #1 }
            \@@_tl_get_nsp:NN #1 #2
          }
          {
            \tl_set:Nx #2 { \tl_head:N #1 }
            \tl_set:Nx #1 { \tl_tail:N #1 }
          }
      }
      {
        \tl_if_head_is_space:VT #1
          {
            \tl_set:Nf #1 {#1}
            \@@_tl_get_nsp:NN #1 #2
          }
      }
  }
%    \end{macrocode}
% \end{macro}
%
% \begin{macro}[int]{\@@_strip_escape:w}
% \begin{macro}[aux]{\@@_strip_escape_aux:N, \@@_strip_escape_aux:w}
%   This is based on the 2013-07-19 (and earlier) version of
%   \cs{cs_to_str:N}.  There are three cases.  If the escape character
%   is printable, the charcode test is false, and
%   \cs{@@_strip_escape_aux:N} removes one character.  If the escape
%   character is a space, the charcode test is true, and if there is no
%   escape charcter, the test is unfinished after |\token_to_str:N \ |.
%   In both of those cases, \cs{@@_strip_escape_aux:w} inserts
%   |-\__int_value:w \fi: \c_zero|.  If the escape character was a
%   space, the test was true, and \cs{__int_value:w} converts
%   \cs{c_zero} to~|0|, hence the leading roman numeral expansion
%   removes a space from what follows (it is important that what follows
%   cannot start with a digit).  Otherwise, the test takes~|-| as its
%   second operand, is false, and the roman numeral expansion only sees
%   \cs{c_zero}, thus does not remove anything from what follows.
%    \begin{macrocode}
\cs_new_nopar:Npn \@@_strip_escape:w
  {
    \tex_romannumeral:D
      \if_charcode:w \token_to_str:N \ \@@_strip_escape_aux:w \fi:
      \@@_strip_escape_aux:N
  }
\cs_new:Npn \@@_strip_escape_aux:N #1 { \c_zero }
\cs_new:Npn \@@_strip_escape_aux:w #1#2
  { - \__int_value:w #1 \c_zero }
%    \end{macrocode}
% \end{macro}
% \end{macro}
%
% \subsection{Main command}
%
% \begin{macro}{\unravel}
%   Simply call an underlying internal command.
%    \begin{macrocode}
\cs_new_protected:Npn \unravel #1 { \@@_unravel:n {#1} }
%    \end{macrocode}
% \end{macro}
%
% \begin{macro}[int]{\@@_unravel:n}
%   Welcome the user, then initialize the input, output and step.  Until
%   the input is exhausted, print the current status and do one step.
%    \begin{macrocode}
\cs_new_protected:Npn \@@_unravel:n #1
  {
    \@@_print_welcome:
    \xtl_gset:Nn  \g_@@_input_xtl {#1}
    \tl_gclear:N \g_@@_output_tl
    \int_gzero:N \g_@@_step_int
    \bool_until_do:nn
      { \xtl_if_empty_p:N \g_@@_input_xtl }
      {
        \@@_print_state:
        \@@_step:
      }
    \@@_print_outcome:
  }
%    \end{macrocode}
% \end{macro}
%
% \begin{macro}[int]{\@@_step:}
%   Find the |action|, give a default value to the |action_text|, and
%   perform the action if the corresponding command |do_|\meta{action}
%   exists.  If that command does not exist, complain, remove the
%   offending first token of the input and place it in the output.
%    \begin{macrocode}
\cs_new_protected_nopar:Npn \@@_step:
  {
    \int_gincr:N \g_@@_step_int
    \@@_get_action:NN \g_@@_input_xtl \l_@@_action_tl
    \tl_set_eq:NN \l_@@_action_text_tl \l_@@_action_tl
    \cs_if_exist_use:cF { @@_do_ \tl_use:N \l_@@_action_tl : }
      {
        \msg_error:nnx { unravel } { unknown-action }
          { \l_@@_action_tl }
        \xtl_gpop_left:N \g_@@_input_xtl
        \xtl_gconcat:NNN \g_@@_output_xtl \g_@@_output_xtl \l_@@_head_xtl
      }
    \@@_print_action:
  }
%    \end{macrocode}
% \end{macro}
%
% \subsection{Printing}
%
% Let us start with the procedure which prints to the terminal: this
% will help me test the code while I'm writing it.
%
%^^A Not implemented yet: non-hash-doubling version.
%
% \subsubsection{Printing proper}
%
% \begin{macro}[int]{\@@_print:n, \@@_print:x}
%    \begin{macrocode}
\cs_new_eq:NN \@@_print:n \iow_term:n
\cs_generate_variant:Nn \@@_print:n { x }
%    \end{macrocode}
% \end{macro}
%
% \begin{macro}[int]{\@@_print_message:nn}
%   The message to be printed should come already detokenized, as~|#2|.
%   It will be wrapped to 80 characters per line, with~|#1| before each
%   line.
%    \begin{macrocode}
\cs_new_protected:Npn \@@_print_message:nn #1 #2
  { \iow_wrap:nnnN { #1 #2 } { #1 } { } \@@_print:n }
%    \end{macrocode}
% \end{macro}
%
% \subsubsection{Various messages}
%
% \begin{macro}[int]{\@@_print_welcome:}
%   Welcoming message.
%    \begin{macrocode}
\cs_new_protected_nopar:Npn \@@_print_welcome:
  {
    \@@_print_message:nn { }
      {
        \\
        ========~ Welcome~ to~ the~ unravel~ package~ ========\\
        \iow_indent:n
          {
            "<|"~ denotes~ the~ output~ to~ TeX's~ stomach. \\
            "|>"~ denotes~ tokens~ that~ we~ will~ act~ on. \\
          }
      }
  }
%    \end{macrocode}
% \end{macro}
%
% \begin{macro}[int]{\@@_print_state:}
%   We want to print both \cs{g_@@_input_xtl} and \cs{g_@@_output_xtl}.
%    \begin{macrocode}
\cs_new_protected:Npn \@@_print_state:
  {
    \@@_print:n { }
    \@@_print_message:nn { <| ~ } { \xtl_to_str:N \g_@@_output_xtl }
    \@@_print:n { }
    \@@_print_message:nn { |> ~ } { \xtl_to_str:N \g_@@_input_xtl }
  }
%    \end{macrocode}
% \end{macro}
%
% \begin{macro}[int]{\@@_print_action:}
%    \begin{macrocode}
\cs_new_protected:Npn \@@_print_action:
  {
    \@@_print_message:nn { }
      {
        \\
        Step~ \int_use:N \g_@@_step_int \
        =====~ \tl_to_str:N \l_@@_action_text_tl .
      }
  }
%    \end{macrocode}
% \end{macro}
%
% \begin{macro}[int]{\@@_print_outcome:}
%   Final message.
%    \begin{macrocode}
\cs_new_protected_nopar:Npn \@@_print_outcome:
  {
    \@@_print:n { }
    \@@_print_message:nn { <| ~ } { \xtl_to_str:N \g_@@_output_xtl }
    \@@_print_message:nn { }
      { \\ Step~ \int_use:N \g_@@_step_int \ was~ the~last! }
  }
%    \end{macrocode}
% \end{macro}
%
% \begin{macro}[int]{\@@_print_next:}
%   When we read the head of the token list \cs{g_@@_input_tl},
%   in Section~\ref{sec:head} %^^A todo
% \end{macro}
%
% \subsection{Examining the head}
% \label{sec:head}
%
% The user input is stored at all times in \cs{g_@@_input_xtl}.  One
% step is either to expand the leading token of \cs{g_@@_input_xtl} if
% it is expandable, or to perform one assignment, or enter a group,
% \emph{etc.}  In order to do that, we first have to carefully examine
% the leading token of \cs{g_@@_input_xtl}.
%
% Possible values of the action:
% \begin{itemize}
% \item |empty|
% \item |macro| if the token is a macro,
% \item |ifnum| or other expandable primitives,
% \item |def| or other non-expandable primitives,
% \item |toks_alt|, |char"_alt| or other ``registers'', followed by
%   |_alt| to distinguish it from the corresponding primitive.
% \item |begin_alt|, |end_alt|, |math_alt|, |alignment_alt|,
%   |macro_alt|, |superscript_alt|, |subscript_alt|, |blank_alt|,
%   |the_alt| for character tokens, both letters and others becoming
%   |the_alt|.
% \end{itemize}
%
% \begin{macro}[int]{\@@_get_action:NN}
%    \begin{macrocode}
\cs_new_protected:Npn \@@_get_action:NN #1#2
  {
    \xtl_if_empty:NTF #1
      { \tl_set:Nn #2 { empty } }
      {
        \xtl_get_left:NN #1 \l_@@_head_xtl
        \xtl_if_tl:NTF \l_@@_head_xtl
          {
            \tl_set:Nx \l_@@_head_tl { \xtl_head:N \l_@@_head_xtl }
            \tl_if_head_is_space:oTF \l_@@_head_tl
              { \tl_set:Nn #2 { blank_alt } }
              {
                \exp_after:wN \@@_get_action_normal:NN
                  \l_@@_head_tl #2
              }
          }
          {
            \xtl_if_head_is_group_end:NTF \l_@@_head_xtl
              { \tl_set:Nn #2 { end_alt } }
              { \tl_set:Nn #2 { begin_alt } }
          }
      }
  }
%    \end{macrocode}
% \end{macro}
%
% \begin{macro}[aux]
%   {\@@_get_action_normal:NN, \@@_get_action_normal:nNN}
% \begin{macro}[aux]
%   {\@@_first_word:w, \@@_strip_hex:w, \@@_strip_hex_end:w}
%   We still need to analyse the meaning of the head token to determine
%   if it is a primitive, if it is expandable, \emph{etc.}  This macro
%   sets the token list variable~|#2| to one of
%   \begin{itemize}
%   \item |macro| if the token is a macro,
%   \item |ifnum| or other expandable primitives,
%   \item |def| or other non-expandable primitives,
%   \item |toks_alt|, |char"_alt| or other ``registers'', followed by
%     |_alt| to distinguish it from the corresponding primitive.
%   \item |begin_alt|, |end_alt|, |math_alt|, |alignment_alt|,
%     |macro_alt|, |superscript_alt|, |subscript_alt|, |blank_alt|,
%     |the_alt| for character tokens, both letters and others becoming
%     |the_alt|.
%   \end{itemize}
%    \begin{macrocode}
\cs_new_protected:Npn \@@_get_action_normal:NN #1#2
  {
    \tl_set:Nx \l_@@_head_meaning_tl { \token_to_meaning:N #1 }
    \exp_args:Nx \@@_get_action_normal:nNN
      { \exp_after:wN \@@_strip_escape:w \l_@@_head_meaning_tl } #1 #2
  }
\cs_new_protected:Npn \@@_get_action_normal:nNN #1#2#3
  {
    \tl_set:Nx #3
      {
        \token_if_expandable:NTF #2
          {
            \token_if_macro:NTF #2
              { macro }
              {#1}
          }
          {
            \token_if_cs:NTF #2
              {
                \str_if_eq_x:nnTF {#1} { \@@_strip_hex:w #1 @ }
                  {#1}
                  { \@@_strip_hex:w #1 @ _alt }
              }
              {
                \exp_after:wN \@@_first_word:w
                  \l_@@_head_meaning_tl \q_stop
                _alt
              }
          }
      }
  }
\cs_new:Npn \@@_first_word:w #1 ~ #2 \q_stop {#1}
\cs_new:Npn \@@_strip_hex:w #1
  {
    \if_int_compare:w `#1 < `G ~ %
      \if_int_compare:w `#1 > `/ ~ %
        \if_charcode:w #1 @
          \exp_after:wN \@@_strip_hex_end:w
        \fi:
        \use_i_ii:nnn
      \fi:
    \fi:
    #1
    \@@_strip_hex:w
  }
\cs_new:Npn \@@_strip_hex_end:w #1#2 @ #3 {#2}
%    \end{macrocode}
% \end{macro}
% \end{macro}
%
% \subsection{Actions}
%
% \subsubsection{Character tokens}
%
% \begin{macro}[int]{\@@_do_the_alt:}
% \begin{macro}[aux]{\@@_do_the_aux:w}
%   ^^A todo: get right catcode.
%    \begin{macrocode}
\cs_new_protected_nopar:Npn \@@_do_the_alt:
  {
    \tl_set_eq:NN \l_@@_action_text_tl \l_@@_head_meaning_tl
    \xtl_gpop_left:N \g_@@_input_xtl
    \xtl_gput_right:Nx \g_@@_output_xtl
      { \exp_after:wN \@@_do_the_aux:w \l_@@_head_meaning_tl }
  }
\cs_new:Npn \@@_do_the_aux:w #1 ~ #2 ~ { }
%    \end{macrocode}
% \end{macro}
% \end{macro}
%
% \subsubsection{Assignments}
%
% \begin{macro}[int]{\@@_do_def:}
% \begin{macro}[aux]{\@@_do_def_extra_begin:, \@@_do_def_extra_end:}
%    \begin{macrocode}
\cs_new_protected_nopar:Npn \@@_do_def:
  {
    \tl_set:Nx \l_@@_action_text_tl { Assignment:~ \tl_to_str:N \l_@@_head_tl }
    \tl_set:Nx \l_@@_tmpa_tl
      { \xtl_left_text:NF \g_@@_input_xtl { } }
    \tl_if_empty:NTF \l_@@_tmpa_tl
      {
        \int_compare:nNnTF
          { \xtl_extra_end:N \g_@@_input_xtl } > \c_zero
          { \@@_do_def_extra_end: }
          { \@@_do_def_extra_begin: }
      }
      {
        \xtl_gpop_left_text:N \g_@@_input_xtl
        \tl_use:N \l_@@_tmpa_tl
      }
  }
\cs_new_protected_nopar:Npn \@@_do_def_extra_end:
  {
    \tl_set:Nx \l_@@_tmpa_tl
      { \xtl_left_tl:N \g_@@_input_xtl }
    \xtl_gpop_left_tl:N \g_@@_input_xtl
    \xtl_gpop_left:N \g_@@_input_xtl
    % ^^A Error on purpose.
    \if_false: { \fi: \tl_use:N \l_@@_tmpa_tl }
  }
\cs_new_protected_nopar:Npn \@@_do_def_extra_begin:
  {
    \msg_error:nn { unravel } { runaway-def }
    \xtl_gclear:N \g_@@_input_xtl
  }
\cs_set_eq:NN \@@_do_gdef: \@@_do_def:
%    \end{macrocode}
% \end{macro}
% \end{macro}
%
% \subsubsection{One step of expansion}
%
% \begin{macro}[int]{\@@_do_one_expansion:}
%    \begin{macrocode}
\cs_new_protected_nopar:Npn \@@_do_one_expansion:
  {
    \tl_set:Nx \l_@@_tmpa_tl { \xtl_left_tl:N \g_@@_input_xtl }
    \xtl_gpop_left_tl:N \g_@@_input_xtl
    \xtl_gput_left:Nx \g_@@_input_xtl
     { \exp_args:NV \exp_not:o \l_@@_tmpa_tl }
  }
%    \end{macrocode}
% \end{macro}
%
% \begin{macro}[int]{\@@_do_macro:}
%   Macros are simply expanded once.
%    \begin{macrocode}
\cs_new_protected_nopar:Npn \@@_do_macro:
  {
    \tl_set:Nx \l_@@_action_text_tl { Macro:~ \tl_to_str:N \l_@@_head_tl }
    \@@_do_one_expansion:
  }
%    \end{macrocode}
% \end{macro}
%
% \begin{macro}[int]{\@@_do_csname:, \@@_do_expandafter:}
%    \begin{macrocode}
\cs_new_eq:NN \@@_do_csname: \@@_do_one_expansion:
\cs_new_eq:NN \@@_do_expandafter: \@@_do_one_expansion:
%    \end{macrocode}
% \end{macro}
%
% \subsubsection{Conditionals}
%
% \begin{macro}[int]{\@@_if:n, \@@_if_true:, \@@_if_false:}
% \begin{macro}[aux]{\@@_if_loop:, \@@_if_loop_token:N}
%   ^^A todo: look at tex.web: this does not catch \iffalse\else\else\fi.
%    \begin{macrocode}
\cs_new_protected:Npn \@@_if:n #1
  {
    \int_set:Nn \l_@@_if_case_int {#1}
    \if_case:w \c_zero
    \int_compare:nNnF \l_@@_if_case_int = \c_zero
      {
        \int_zero:N \l_@@_if_nesting_int
        \@@_if_loop:
      }
  }
\cs_new_protected_nopar:Npn \@@_if_true:
  {
    \tl_put_right:Nx \l_@@_action_text_tl { ~ : ~ true }
    \@@_if:n \c_zero
  }
\cs_new_protected_nopar:Npn \@@_if_false:
  {
    \tl_put_right:Nx \l_@@_action_text_tl { ~ : ~ false }
    \@@_if:n \c_minus_one
  }
\cs_new_protected_nopar:Npn \@@_if_loop:
  {
    \xtl_if_head_is_N_type:NTF \g_@@_input_xtl
      {
        \tl_set:Nx \l_@@_tmpa_tl { \xtl_head:N \g_@@_input_xtl }
        \xtl_gpop_left:N \g_@@_input_xtl
        \tl_if_eq:NNTF \l_@@_tmpa_tl \c_space_tl
          { \@@_if_loop: }
          { \exp_after:wN \@@_if_loop_token:N \l_@@_tmpa_tl }
      }
      {
        \xtl_if_empty:NTF \g_@@_input_tl
          { \msg_error:nn { unravel } { runaway-if } }
          {
            \xtl_gpop_left:N \g_@@_input_xtl
            \@@_if_loop:
          }
      }
  }
\cs_new_protected:Npn \@@_if_loop_token:N #1
  {
    \@@_cs_case:Nnn #1
      {
        \else:
          { \int_compare:nNnF \l_@@_if_nesting_int = \c_zero { \@@_if_loop: } }
        \or:
          {
            \int_compare:nNnTF \l_@@_if_nesting_int = \c_zero
              {
                \int_decr:N \l_@@_if_case_int
                \int_compare:nNnF \l_@@_if_case_int = \c_zero { \@@_if_loop: }
              }
              { \@@_if_loop: }
          }
        \fi:
          {
            \int_compare:nNnTF \l_@@_if_nesting_int = \c_zero
              { \fi: }
              {
                \int_decr:N \l_@@_if_nesting_int
                \@@_if_loop:
              }
          }
      }
      {
        \if_false: #1 \fi: \use_none:nnn \fi:
        \int_incr:N \l_@@_if_nesting_int
        \@@_if_loop:
      }
  }
%    \end{macrocode}
% \end{macro}
% \end{macro}
%
% \begin{macro}[int]{\@@_do_iftrue:}
%    \begin{macrocode}
\cs_new_protected_nopar:Npn \@@_do_iftrue:
  {
    \tl_set:Nx \l_@@_action_text_tl { Compare:~ \iow_char:N \\ iftrue }
    \xtl_gpop_left:N \g_@@_input_xtl
    \@@_if_true:
  }
%    \end{macrocode}
% \end{macro}
%
% \begin{macro}[int]{\@@_do_iffalse:}
%    \begin{macrocode}
\cs_new_protected_nopar:Npn \@@_do_iffalse:
  {
    \tl_set:Nx \l_@@_action_text_tl { Compare:~ \iow_char:N \\ iffalse }
    \xtl_gpop_left:N \g_@@_input_xtl
    \@@_if_false:
  }
%    \end{macrocode}
% \end{macro}
%
% \begin{macro}[int]{\@@_do_else:}
%    \begin{macrocode}
\cs_new_protected_nopar:Npn \@@_do_else:
  {
    \tl_set:Nx \l_@@_action_text_tl { Compare:~ \iow_char:N \\ else }
    \xtl_gpop_left:N \g_@@_input_xtl
    \@@_if:n { \c_minus_one }
  }
%    \end{macrocode}
% \end{macro}
%
% \begin{macro}[int]{\@@_do_or:}
%   ^^A todo: this is wrong!! Takes the else branch in all cases.
%    \begin{macrocode}
 % \cs_new_protected_nopar:Npn \@@_do_or:
 %   {
 %     \tl_set:Nx \l_@@_action_text_tl { Compare:~ \iow_char:N \\ or }
 %     \xtl_gpop_left:N \g_@@_input_xtl
 %     \@@_if:n { \c_minus_one }
 %   }
%    \end{macrocode}
% \end{macro}
%
% \begin{macro}[int]{\@@_do_ifx:}
% \begin{macro}[aux, EXP]{\@@_do_ifx_aux:w}
%    \begin{macrocode}
\cs_new_protected_nopar:Npn \@@_do_ifx:
  {
    \xtl_gpop_left:N \g_@@_input_xtl
    \xtl_gpop_left:NN \g_@@_input_xtl \l_@@_head_i_xtl
    \xtl_if_empty:NTF \g_@@_input_xtl
      {
        \msg_error:nn { unravel } { runaway-if }
      }
      {
        \xtl_gpop_left:NN \g_@@_input_xtl \l_@@_head_ii_xtl
        \tl_set:Nx \l_@@_action_text_tl
          {
            Compare:~ \iow_char:N \\ ifx
            \c_space_tl \xtl_to_str:N \l_@@_head_i_xtl
            \c_space_tl \xtl_to_str:N \l_@@_head_ii_xtl
          }
        \xtl_head_do:NN \l_@@_head_ii_xtl \@@_do_ifx_aux:w
          \exp_after:wN \@@_if_true:
        \else:
          \exp_after:wN \@@_if_false:
        \fi:
      }
  }
\cs_new_nopar:Npn \@@_do_ifx_aux:w
  { \xtl_head_do:NN \l_@@_head_i_xtl \if_meaning:w }
%    \end{macrocode}
% \end{macro}
% \end{macro}
%
% \begin{macro}[int]{\@@_do_ifcase:}
%    \begin{macrocode}
\cs_new_protected_nopar:Npn \@@_do_ifcase:
  {
    \xtl_gpop_left:N \g_@@_input_xtl
    \tl_set:Nx \l_@@_tmpa_tl { \xtl_left_tl:N \g_@@_input_xtl }
    \@@_tl_pop_reg:NN \l_@@_tmpa_tl \l_@@_if_case_int
    \tl_set:Nx \l_@@_action_text_tl
      { Compare:~ \iow_char:N \\ ifcase ~ \int_use:N \l_@@_if_case_int }
    \xtl_gpop_left_tl:N \g_@@_input_xtl
    \xtl_gput_left:NV \g_@@_input_xtl \l_@@_tmpa_tl
    \@@_if:n { \l_@@_if_case_int }
  }
%    \end{macrocode}
% \end{macro}
%
% \begin{macro}[int]{\@@_do_ifnum:}
%    \begin{macrocode}
\cs_new_protected_nopar:Npn \@@_do_ifnum:
  {
    \xtl_gpop_left:N \g_@@_input_xtl
    \tl_set:Nx \l_@@_tmpa_tl { \xtl_left_tl:N \g_@@_input_xtl }
    \@@_tl_pop_reg:NN \l_@@_tmpa_tl \l_@@_if_a_int
    \tl_clear:N \l_@@_cmp_tl
    \@@_tl_get_nsp:NN \l_@@_tmpa_tl \l_@@_cmp_tl
    \str_case:onn { \l_@@_cmp_tl }
      { { < } { }   { = } { }   { > } { } }
      {
        \msg_error:nn { unravel } { missing-cmp }
        \tl_put_left:NV \l_@@_tmpa_tl \l_@@_cmp_tl
        \tl_set:Nn \l_@@_cmp_tl { = }
      }
    \@@_tl_pop_reg:NN \l_@@_tmpa_tl \l_@@_if_b_int
    \xtl_gpop_left_tl:N \g_@@_input_xtl
    \xtl_gput_left:NV \g_@@_input_xtl \l_@@_tmpa_tl
    \tl_set:Nx \l_@@_action_text_tl
      {
        Compare:~ \iow_char:N \\ ifnum
        \c_space_tl \int_use:N \l_@@_if_a_int
        \c_space_tl \tl_use:N \l_@@_cmp_tl
        \c_space_tl \int_use:N \l_@@_if_b_int
      }
    \int_compare:nNnTF \l_@@_if_a_int \l_@@_cmp_tl \l_@@_if_b_int
      { \@@_if_true: } { \@@_if_false: }
  }
%    \end{macrocode}
% \end{macro}
%
% \subsection{Messages}
%
%    \begin{macrocode}
\msg_new:nnn { unravel } { unknown-action }
  { Internal~error:~the~action~'#1'~is~not~known. }
%    \end{macrocode}
%
%    \begin{macrocode}
%</package>
%    \end{macrocode}
%
% \end{implementation}
%
% \endinput
